\documentclass[11pt]{article}

    \usepackage[breakable]{tcolorbox}
    \usepackage{parskip} % Stop auto-indenting (to mimic markdown behaviour)
   
	 \renewcommand{\familydefault}{\sfdefault}

    \usepackage{iftex}
    \ifPDFTeX
    	\usepackage{helvet}
    \else
    	\usepackage{helvet}
    \fi

    % Basic figure setup, for now with no caption control since it's done
    % automatically by Pandoc (which extracts ![](path) syntax from Markdown).
    \usepackage{graphicx}
    % Maintain compatibility with old templates. Remove in nbconvert 6.0
    \let\Oldincludegraphics\includegraphics
    % Ensure that by default, figures have no caption (until we provide a
    % proper Figure object with a Caption API and a way to capture that
    % in the conversion process - todo).
    \usepackage{caption}
    \DeclareCaptionFormat{nocaption}{}
    \captionsetup{format=nocaption,aboveskip=0pt,belowskip=0pt}

    \usepackage[Export]{adjustbox} % Used to constrain images to a maximum size
    \adjustboxset{max size={0.9\linewidth}{0.9\paperheight}}
    \usepackage{float}
    \floatplacement{figure}{H} % forces figures to be placed at the correct location
    \usepackage{xcolor} % Allow colors to be defined
    \usepackage{enumerate} % Needed for markdown enumerations to work
    \usepackage{geometry} % Used to adjust the document margins
    \usepackage{amsmath} % Equations
    \usepackage{amssymb} % Equations
    \usepackage{textcomp} % defines textquotesingle
    % Hack from http://tex.stackexchange.com/a/47451/13684:
    \AtBeginDocument{%
        \def\PYZsq{\textquotesingle}% Upright quotes in Pygmentized code
    }
    \usepackage{upquote} % Upright quotes for verbatim code
    \usepackage{eurosym} % defines \euro
    \usepackage[mathletters]{ucs} % Extended unicode (utf-8) support
    \usepackage{fancyvrb} % verbatim replacement that allows latex
    \usepackage{grffile} % extends the file name processing of package graphics 
                         % to support a larger range
    \makeatletter % fix for grffile with XeLaTeX
    \def\Gread@@xetex#1{%
      \IfFileExists{"\Gin@base".bb}%
      {\Gread@eps{\Gin@base.bb}}%
      {\Gread@@xetex@aux#1}%
    }
    \makeatother

    % The hyperref package gives us a pdf with properly built
    % internal navigation ('pdf bookmarks' for the table of contents,
    % internal cross-reference links, web links for URLs, etc.)
    \usepackage{hyperref}
    % The default LaTeX title has an obnoxious amount of whitespace. By default,
    % titling removes some of it. It also provides customization options.
    \usepackage{titling}
    \usepackage{longtable} % longtable support required by pandoc >1.10
    \usepackage{booktabs}  % table support for pandoc > 1.12.2
    \usepackage[inline]{enumitem} % IRkernel/repr support (it uses the enumerate* environment)
    \usepackage[normalem]{ulem} % ulem is needed to support strikethroughs (\sout)
                                % normalem makes italics be italics, not underlines
    \usepackage{mathrsfs}
    

    
    % Colors for the hyperref package
    \definecolor{urlcolor}{rgb}{0,.145,.698}
    \definecolor{linkcolor}{rgb}{.71,0.21,0.01}
    \definecolor{citecolor}{rgb}{.12,.54,.11}

    % ANSI colors
    \definecolor{ansi-black}{HTML}{3E424D}
    \definecolor{ansi-black-intense}{HTML}{282C36}
    \definecolor{ansi-red}{HTML}{E75C58}
    \definecolor{ansi-red-intense}{HTML}{B22B31}
    \definecolor{ansi-green}{HTML}{00A250}
    \definecolor{ansi-green-intense}{HTML}{007427}
    \definecolor{ansi-yellow}{HTML}{DDB62B}
    \definecolor{ansi-yellow-intense}{HTML}{B27D12}
    \definecolor{ansi-blue}{HTML}{208FFB}
    \definecolor{ansi-blue-intense}{HTML}{0065CA}
    \definecolor{ansi-magenta}{HTML}{D160C4}
    \definecolor{ansi-magenta-intense}{HTML}{A03196}
    \definecolor{ansi-cyan}{HTML}{60C6C8}
    \definecolor{ansi-cyan-intense}{HTML}{258F8F}
    \definecolor{ansi-white}{HTML}{C5C1B4}
    \definecolor{ansi-white-intense}{HTML}{A1A6B2}
    \definecolor{ansi-default-inverse-fg}{HTML}{FFFFFF}
    \definecolor{ansi-default-inverse-bg}{HTML}{000000}

    % commands and environments needed by pandoc snippets
    % extracted from the output of `pandoc -s`
    \providecommand{\tightlist}{%
      \setlength{\itemsep}{0pt}\setlength{\parskip}{0pt}}
    \DefineVerbatimEnvironment{Highlighting}{Verbatim}{commandchars=\\\{\}}
    % Add ',fontsize=\small' for more characters per line
    \newenvironment{Shaded}{}{}
    \newcommand{\KeywordTok}[1]{\textcolor[rgb]{0.00,0.44,0.13}{\textbf{{#1}}}}
    \newcommand{\DataTypeTok}[1]{\textcolor[rgb]{0.56,0.13,0.00}{{#1}}}
    \newcommand{\DecValTok}[1]{\textcolor[rgb]{0.25,0.63,0.44}{{#1}}}
    \newcommand{\BaseNTok}[1]{\textcolor[rgb]{0.25,0.63,0.44}{{#1}}}
    \newcommand{\FloatTok}[1]{\textcolor[rgb]{0.25,0.63,0.44}{{#1}}}
    \newcommand{\CharTok}[1]{\textcolor[rgb]{0.25,0.44,0.63}{{#1}}}
    \newcommand{\StringTok}[1]{\textcolor[rgb]{0.25,0.44,0.63}{{#1}}}
    \newcommand{\CommentTok}[1]{\textcolor[rgb]{0.38,0.63,0.69}{\textit{{#1}}}}
    \newcommand{\OtherTok}[1]{\textcolor[rgb]{0.00,0.44,0.13}{{#1}}}
    \newcommand{\AlertTok}[1]{\textcolor[rgb]{1.00,0.00,0.00}{\textbf{{#1}}}}
    \newcommand{\FunctionTok}[1]{\textcolor[rgb]{0.02,0.16,0.49}{{#1}}}
    \newcommand{\RegionMarkerTok}[1]{{#1}}
    \newcommand{\ErrorTok}[1]{\textcolor[rgb]{1.00,0.00,0.00}{\textbf{{#1}}}}
    \newcommand{\NormalTok}[1]{{#1}}
    
    % Additional commands for more recent versions of Pandoc
    \newcommand{\ConstantTok}[1]{\textcolor[rgb]{0.53,0.00,0.00}{{#1}}}
    \newcommand{\SpecialCharTok}[1]{\textcolor[rgb]{0.25,0.44,0.63}{{#1}}}
    \newcommand{\VerbatimStringTok}[1]{\textcolor[rgb]{0.25,0.44,0.63}{{#1}}}
    \newcommand{\SpecialStringTok}[1]{\textcolor[rgb]{0.73,0.40,0.53}{{#1}}}
    \newcommand{\ImportTok}[1]{{#1}}
    \newcommand{\DocumentationTok}[1]{\textcolor[rgb]{0.73,0.13,0.13}{\textit{{#1}}}}
    \newcommand{\AnnotationTok}[1]{\textcolor[rgb]{0.38,0.63,0.69}{\textbf{\textit{{#1}}}}}
    \newcommand{\CommentVarTok}[1]{\textcolor[rgb]{0.38,0.63,0.69}{\textbf{\textit{{#1}}}}}
    \newcommand{\VariableTok}[1]{\textcolor[rgb]{0.10,0.09,0.49}{{#1}}}
    \newcommand{\ControlFlowTok}[1]{\textcolor[rgb]{0.00,0.44,0.13}{\textbf{{#1}}}}
    \newcommand{\OperatorTok}[1]{\textcolor[rgb]{0.40,0.40,0.40}{{#1}}}
    \newcommand{\BuiltInTok}[1]{{#1}}
    \newcommand{\ExtensionTok}[1]{{#1}}
    \newcommand{\PreprocessorTok}[1]{\textcolor[rgb]{0.74,0.48,0.00}{{#1}}}
    \newcommand{\AttributeTok}[1]{\textcolor[rgb]{0.49,0.56,0.16}{{#1}}}
    \newcommand{\InformationTok}[1]{\textcolor[rgb]{0.38,0.63,0.69}{\textbf{\textit{{#1}}}}}
    \newcommand{\WarningTok}[1]{\textcolor[rgb]{0.38,0.63,0.69}{\textbf{\textit{{#1}}}}}
    
    
    % Define a nice break command that doesn't care if a line doesn't already
    % exist.
    \def\br{\hspace*{\fill} \\* }
    % Math Jax compatibility definitions
    \def\gt{>}
    \def\lt{<}
    \let\Oldtex\TeX
    \let\Oldlatex\LaTeX
    \renewcommand{\TeX}{\textrm{\Oldtex}}
    \renewcommand{\LaTeX}{\textrm{\Oldlatex}}
    % Document parameters
    % Document title
    \title{IDS.147\_PSet\_3}
    
    
    
    
    
% Pygments definitions
\makeatletter
\def\PY@reset{\let\PY@it=\relax \let\PY@bf=\relax%
    \let\PY@ul=\relax \let\PY@tc=\relax%
    \let\PY@bc=\relax \let\PY@ff=\relax}
\def\PY@tok#1{\csname PY@tok@#1\endcsname}
\def\PY@toks#1+{\ifx\relax#1\empty\else%
    \PY@tok{#1}\expandafter\PY@toks\fi}
\def\PY@do#1{\PY@bc{\PY@tc{\PY@ul{%
    \PY@it{\PY@bf{\PY@ff{#1}}}}}}}
\def\PY#1#2{\PY@reset\PY@toks#1+\relax+\PY@do{#2}}

\expandafter\def\csname PY@tok@w\endcsname{\def\PY@tc##1{\textcolor[rgb]{0.73,0.73,0.73}{##1}}}
\expandafter\def\csname PY@tok@c\endcsname{\let\PY@it=\textit\def\PY@tc##1{\textcolor[rgb]{0.25,0.50,0.50}{##1}}}
\expandafter\def\csname PY@tok@cp\endcsname{\def\PY@tc##1{\textcolor[rgb]{0.74,0.48,0.00}{##1}}}
\expandafter\def\csname PY@tok@k\endcsname{\let\PY@bf=\textbf\def\PY@tc##1{\textcolor[rgb]{0.00,0.50,0.00}{##1}}}
\expandafter\def\csname PY@tok@kp\endcsname{\def\PY@tc##1{\textcolor[rgb]{0.00,0.50,0.00}{##1}}}
\expandafter\def\csname PY@tok@kt\endcsname{\def\PY@tc##1{\textcolor[rgb]{0.69,0.00,0.25}{##1}}}
\expandafter\def\csname PY@tok@o\endcsname{\def\PY@tc##1{\textcolor[rgb]{0.40,0.40,0.40}{##1}}}
\expandafter\def\csname PY@tok@ow\endcsname{\let\PY@bf=\textbf\def\PY@tc##1{\textcolor[rgb]{0.67,0.13,1.00}{##1}}}
\expandafter\def\csname PY@tok@nb\endcsname{\def\PY@tc##1{\textcolor[rgb]{0.00,0.50,0.00}{##1}}}
\expandafter\def\csname PY@tok@nf\endcsname{\def\PY@tc##1{\textcolor[rgb]{0.00,0.00,1.00}{##1}}}
\expandafter\def\csname PY@tok@nc\endcsname{\let\PY@bf=\textbf\def\PY@tc##1{\textcolor[rgb]{0.00,0.00,1.00}{##1}}}
\expandafter\def\csname PY@tok@nn\endcsname{\let\PY@bf=\textbf\def\PY@tc##1{\textcolor[rgb]{0.00,0.00,1.00}{##1}}}
\expandafter\def\csname PY@tok@ne\endcsname{\let\PY@bf=\textbf\def\PY@tc##1{\textcolor[rgb]{0.82,0.25,0.23}{##1}}}
\expandafter\def\csname PY@tok@nv\endcsname{\def\PY@tc##1{\textcolor[rgb]{0.10,0.09,0.49}{##1}}}
\expandafter\def\csname PY@tok@no\endcsname{\def\PY@tc##1{\textcolor[rgb]{0.53,0.00,0.00}{##1}}}
\expandafter\def\csname PY@tok@nl\endcsname{\def\PY@tc##1{\textcolor[rgb]{0.63,0.63,0.00}{##1}}}
\expandafter\def\csname PY@tok@ni\endcsname{\let\PY@bf=\textbf\def\PY@tc##1{\textcolor[rgb]{0.60,0.60,0.60}{##1}}}
\expandafter\def\csname PY@tok@na\endcsname{\def\PY@tc##1{\textcolor[rgb]{0.49,0.56,0.16}{##1}}}
\expandafter\def\csname PY@tok@nt\endcsname{\let\PY@bf=\textbf\def\PY@tc##1{\textcolor[rgb]{0.00,0.50,0.00}{##1}}}
\expandafter\def\csname PY@tok@nd\endcsname{\def\PY@tc##1{\textcolor[rgb]{0.67,0.13,1.00}{##1}}}
\expandafter\def\csname PY@tok@s\endcsname{\def\PY@tc##1{\textcolor[rgb]{0.73,0.13,0.13}{##1}}}
\expandafter\def\csname PY@tok@sd\endcsname{\let\PY@it=\textit\def\PY@tc##1{\textcolor[rgb]{0.73,0.13,0.13}{##1}}}
\expandafter\def\csname PY@tok@si\endcsname{\let\PY@bf=\textbf\def\PY@tc##1{\textcolor[rgb]{0.73,0.40,0.53}{##1}}}
\expandafter\def\csname PY@tok@se\endcsname{\let\PY@bf=\textbf\def\PY@tc##1{\textcolor[rgb]{0.73,0.40,0.13}{##1}}}
\expandafter\def\csname PY@tok@sr\endcsname{\def\PY@tc##1{\textcolor[rgb]{0.73,0.40,0.53}{##1}}}
\expandafter\def\csname PY@tok@ss\endcsname{\def\PY@tc##1{\textcolor[rgb]{0.10,0.09,0.49}{##1}}}
\expandafter\def\csname PY@tok@sx\endcsname{\def\PY@tc##1{\textcolor[rgb]{0.00,0.50,0.00}{##1}}}
\expandafter\def\csname PY@tok@m\endcsname{\def\PY@tc##1{\textcolor[rgb]{0.40,0.40,0.40}{##1}}}
\expandafter\def\csname PY@tok@gh\endcsname{\let\PY@bf=\textbf\def\PY@tc##1{\textcolor[rgb]{0.00,0.00,0.50}{##1}}}
\expandafter\def\csname PY@tok@gu\endcsname{\let\PY@bf=\textbf\def\PY@tc##1{\textcolor[rgb]{0.50,0.00,0.50}{##1}}}
\expandafter\def\csname PY@tok@gd\endcsname{\def\PY@tc##1{\textcolor[rgb]{0.63,0.00,0.00}{##1}}}
\expandafter\def\csname PY@tok@gi\endcsname{\def\PY@tc##1{\textcolor[rgb]{0.00,0.63,0.00}{##1}}}
\expandafter\def\csname PY@tok@gr\endcsname{\def\PY@tc##1{\textcolor[rgb]{1.00,0.00,0.00}{##1}}}
\expandafter\def\csname PY@tok@ge\endcsname{\let\PY@it=\textit}
\expandafter\def\csname PY@tok@gs\endcsname{\let\PY@bf=\textbf}
\expandafter\def\csname PY@tok@gp\endcsname{\let\PY@bf=\textbf\def\PY@tc##1{\textcolor[rgb]{0.00,0.00,0.50}{##1}}}
\expandafter\def\csname PY@tok@go\endcsname{\def\PY@tc##1{\textcolor[rgb]{0.53,0.53,0.53}{##1}}}
\expandafter\def\csname PY@tok@gt\endcsname{\def\PY@tc##1{\textcolor[rgb]{0.00,0.27,0.87}{##1}}}
\expandafter\def\csname PY@tok@err\endcsname{\def\PY@bc##1{\setlength{\fboxsep}{0pt}\fcolorbox[rgb]{1.00,0.00,0.00}{1,1,1}{\strut ##1}}}
\expandafter\def\csname PY@tok@kc\endcsname{\let\PY@bf=\textbf\def\PY@tc##1{\textcolor[rgb]{0.00,0.50,0.00}{##1}}}
\expandafter\def\csname PY@tok@kd\endcsname{\let\PY@bf=\textbf\def\PY@tc##1{\textcolor[rgb]{0.00,0.50,0.00}{##1}}}
\expandafter\def\csname PY@tok@kn\endcsname{\let\PY@bf=\textbf\def\PY@tc##1{\textcolor[rgb]{0.00,0.50,0.00}{##1}}}
\expandafter\def\csname PY@tok@kr\endcsname{\let\PY@bf=\textbf\def\PY@tc##1{\textcolor[rgb]{0.00,0.50,0.00}{##1}}}
\expandafter\def\csname PY@tok@bp\endcsname{\def\PY@tc##1{\textcolor[rgb]{0.00,0.50,0.00}{##1}}}
\expandafter\def\csname PY@tok@fm\endcsname{\def\PY@tc##1{\textcolor[rgb]{0.00,0.00,1.00}{##1}}}
\expandafter\def\csname PY@tok@vc\endcsname{\def\PY@tc##1{\textcolor[rgb]{0.10,0.09,0.49}{##1}}}
\expandafter\def\csname PY@tok@vg\endcsname{\def\PY@tc##1{\textcolor[rgb]{0.10,0.09,0.49}{##1}}}
\expandafter\def\csname PY@tok@vi\endcsname{\def\PY@tc##1{\textcolor[rgb]{0.10,0.09,0.49}{##1}}}
\expandafter\def\csname PY@tok@vm\endcsname{\def\PY@tc##1{\textcolor[rgb]{0.10,0.09,0.49}{##1}}}
\expandafter\def\csname PY@tok@sa\endcsname{\def\PY@tc##1{\textcolor[rgb]{0.73,0.13,0.13}{##1}}}
\expandafter\def\csname PY@tok@sb\endcsname{\def\PY@tc##1{\textcolor[rgb]{0.73,0.13,0.13}{##1}}}
\expandafter\def\csname PY@tok@sc\endcsname{\def\PY@tc##1{\textcolor[rgb]{0.73,0.13,0.13}{##1}}}
\expandafter\def\csname PY@tok@dl\endcsname{\def\PY@tc##1{\textcolor[rgb]{0.73,0.13,0.13}{##1}}}
\expandafter\def\csname PY@tok@s2\endcsname{\def\PY@tc##1{\textcolor[rgb]{0.73,0.13,0.13}{##1}}}
\expandafter\def\csname PY@tok@sh\endcsname{\def\PY@tc##1{\textcolor[rgb]{0.73,0.13,0.13}{##1}}}
\expandafter\def\csname PY@tok@s1\endcsname{\def\PY@tc##1{\textcolor[rgb]{0.73,0.13,0.13}{##1}}}
\expandafter\def\csname PY@tok@mb\endcsname{\def\PY@tc##1{\textcolor[rgb]{0.40,0.40,0.40}{##1}}}
\expandafter\def\csname PY@tok@mf\endcsname{\def\PY@tc##1{\textcolor[rgb]{0.40,0.40,0.40}{##1}}}
\expandafter\def\csname PY@tok@mh\endcsname{\def\PY@tc##1{\textcolor[rgb]{0.40,0.40,0.40}{##1}}}
\expandafter\def\csname PY@tok@mi\endcsname{\def\PY@tc##1{\textcolor[rgb]{0.40,0.40,0.40}{##1}}}
\expandafter\def\csname PY@tok@il\endcsname{\def\PY@tc##1{\textcolor[rgb]{0.40,0.40,0.40}{##1}}}
\expandafter\def\csname PY@tok@mo\endcsname{\def\PY@tc##1{\textcolor[rgb]{0.40,0.40,0.40}{##1}}}
\expandafter\def\csname PY@tok@ch\endcsname{\let\PY@it=\textit\def\PY@tc##1{\textcolor[rgb]{0.25,0.50,0.50}{##1}}}
\expandafter\def\csname PY@tok@cm\endcsname{\let\PY@it=\textit\def\PY@tc##1{\textcolor[rgb]{0.25,0.50,0.50}{##1}}}
\expandafter\def\csname PY@tok@cpf\endcsname{\let\PY@it=\textit\def\PY@tc##1{\textcolor[rgb]{0.25,0.50,0.50}{##1}}}
\expandafter\def\csname PY@tok@c1\endcsname{\let\PY@it=\textit\def\PY@tc##1{\textcolor[rgb]{0.25,0.50,0.50}{##1}}}
\expandafter\def\csname PY@tok@cs\endcsname{\let\PY@it=\textit\def\PY@tc##1{\textcolor[rgb]{0.25,0.50,0.50}{##1}}}

\def\PYZbs{\char`\\}
\def\PYZus{\char`\_}
\def\PYZob{\char`\{}
\def\PYZcb{\char`\}}
\def\PYZca{\char`\^}
\def\PYZam{\char`\&}
\def\PYZlt{\char`\<}
\def\PYZgt{\char`\>}
\def\PYZsh{\char`\#}
\def\PYZpc{\char`\%}
\def\PYZdl{\char`\$}
\def\PYZhy{\char`\-}
\def\PYZsq{\char`\'}
\def\PYZdq{\char`\"}
\def\PYZti{\char`\~}
% for compatibility with earlier versions
\def\PYZat{@}
\def\PYZlb{[}
\def\PYZrb{]}
\makeatother


    % For linebreaks inside Verbatim environment from package fancyvrb. 
    \makeatletter
        \newbox\Wrappedcontinuationbox 
        \newbox\Wrappedvisiblespacebox 
        \newcommand*\Wrappedvisiblespace {\textcolor{red}{\textvisiblespace}} 
        \newcommand*\Wrappedcontinuationsymbol {\textcolor{red}{\llap{\tiny$\m@th\hookrightarrow$}}} 
        \newcommand*\Wrappedcontinuationindent {3ex } 
        \newcommand*\Wrappedafterbreak {\kern\Wrappedcontinuationindent\copy\Wrappedcontinuationbox} 
        % Take advantage of the already applied Pygments mark-up to insert 
        % potential linebreaks for TeX processing. 
        %        {, <, #, %, $, ' and ": go to next line. 
        %        _, }, ^, &, >, - and ~: stay at end of broken line. 
        % Use of \textquotesingle for straight quote. 
        \newcommand*\Wrappedbreaksatspecials {% 
            \def\PYGZus{\discretionary{\char`\_}{\Wrappedafterbreak}{\char`\_}}% 
            \def\PYGZob{\discretionary{}{\Wrappedafterbreak\char`\{}{\char`\{}}% 
            \def\PYGZcb{\discretionary{\char`\}}{\Wrappedafterbreak}{\char`\}}}% 
            \def\PYGZca{\discretionary{\char`\^}{\Wrappedafterbreak}{\char`\^}}% 
            \def\PYGZam{\discretionary{\char`\&}{\Wrappedafterbreak}{\char`\&}}% 
            \def\PYGZlt{\discretionary{}{\Wrappedafterbreak\char`\<}{\char`\<}}% 
            \def\PYGZgt{\discretionary{\char`\>}{\Wrappedafterbreak}{\char`\>}}% 
            \def\PYGZsh{\discretionary{}{\Wrappedafterbreak\char`\#}{\char`\#}}% 
            \def\PYGZpc{\discretionary{}{\Wrappedafterbreak\char`\%}{\char`\%}}% 
            \def\PYGZdl{\discretionary{}{\Wrappedafterbreak\char`\$}{\char`\$}}% 
            \def\PYGZhy{\discretionary{\char`\-}{\Wrappedafterbreak}{\char`\-}}% 
            \def\PYGZsq{\discretionary{}{\Wrappedafterbreak\textquotesingle}{\textquotesingle}}% 
            \def\PYGZdq{\discretionary{}{\Wrappedafterbreak\char`\"}{\char`\"}}% 
            \def\PYGZti{\discretionary{\char`\~}{\Wrappedafterbreak}{\char`\~}}% 
        } 
        % Some characters . , ; ? ! / are not pygmentized. 
        % This macro makes them "active" and they will insert potential linebreaks 
        \newcommand*\Wrappedbreaksatpunct {% 
            \lccode`\~`\.\lowercase{\def~}{\discretionary{\hbox{\char`\.}}{\Wrappedafterbreak}{\hbox{\char`\.}}}% 
            \lccode`\~`\,\lowercase{\def~}{\discretionary{\hbox{\char`\,}}{\Wrappedafterbreak}{\hbox{\char`\,}}}% 
            \lccode`\~`\;\lowercase{\def~}{\discretionary{\hbox{\char`\;}}{\Wrappedafterbreak}{\hbox{\char`\;}}}% 
            \lccode`\~`\:\lowercase{\def~}{\discretionary{\hbox{\char`\:}}{\Wrappedafterbreak}{\hbox{\char`\:}}}% 
            \lccode`\~`\?\lowercase{\def~}{\discretionary{\hbox{\char`\?}}{\Wrappedafterbreak}{\hbox{\char`\?}}}% 
            \lccode`\~`\!\lowercase{\def~}{\discretionary{\hbox{\char`\!}}{\Wrappedafterbreak}{\hbox{\char`\!}}}% 
            \lccode`\~`\/\lowercase{\def~}{\discretionary{\hbox{\char`\/}}{\Wrappedafterbreak}{\hbox{\char`\/}}}% 
            \catcode`\.\active
            \catcode`\,\active 
            \catcode`\;\active
            \catcode`\:\active
            \catcode`\?\active
            \catcode`\!\active
            \catcode`\/\active 
            \lccode`\~`\~ 	
        }
    \makeatother

    \let\OriginalVerbatim=\Verbatim
    \makeatletter
    \renewcommand{\Verbatim}[1][1]{%
        %\parskip\z@skip
        \sbox\Wrappedcontinuationbox {\Wrappedcontinuationsymbol}%
        \sbox\Wrappedvisiblespacebox {\FV@SetupFont\Wrappedvisiblespace}%
        \def\FancyVerbFormatLine ##1{\hsize\linewidth
            \vtop{\raggedright\hyphenpenalty\z@\exhyphenpenalty\z@
                \doublehyphendemerits\z@\finalhyphendemerits\z@
                \strut ##1\strut}%
        }%
        % If the linebreak is at a space, the latter will be displayed as visible
        % space at end of first line, and a continuation symbol starts next line.
        % Stretch/shrink are however usually zero for typewriter font.
        \def\FV@Space {%
            \nobreak\hskip\z@ plus\fontdimen3\font minus\fontdimen4\font
            \discretionary{\copy\Wrappedvisiblespacebox}{\Wrappedafterbreak}
            {\kern\fontdimen2\font}%
        }%
        
        % Allow breaks at special characters using \PYG... macros.
        \Wrappedbreaksatspecials
        % Breaks at punctuation characters . , ; ? ! and / need catcode=\active 	
        \OriginalVerbatim[#1,codes*=\Wrappedbreaksatpunct]%
    }
    \makeatother

    % Exact colors from NB
    \definecolor{incolor}{HTML}{303F9F}
    \definecolor{outcolor}{HTML}{D84315}
    \definecolor{cellborder}{HTML}{CFCFCF}
    \definecolor{cellbackground}{HTML}{F7F7F7}
    
    % prompt
    \makeatletter
    \newcommand{\boxspacing}{\kern\kvtcb@left@rule\kern\kvtcb@boxsep}
    \makeatother
    \newcommand{\prompt}[4]{
        \ttfamily\llap{{\color{#2}[#3]:\hspace{3pt}#4}}\vspace{-\baselineskip}
    }
    

    
    % Prevent overflowing lines due to hard-to-break entities
    \sloppy 
    % Setup hyperref package
    \hypersetup{
      breaklinks=true,  % so long urls are correctly broken across lines
      colorlinks=true,
      urlcolor=urlcolor,
      linkcolor=linkcolor,
      citecolor=citecolor,
      }
    % Slightly bigger margins than the latex defaults
    
    \geometry{verbose,tmargin=1in,bmargin=1in,lmargin=1in,rmargin=1in}
    
    % FancyHDR
	\usepackage{fancyhdr}
	\pagestyle{fancyplain}
	\rhead{Chun-Hei (Samuel) Lam \\ ID: 928931321}
	\chead{15.077 Statistical ML and DS \\ Problem Sheet 3}
	\lhead{}
	\setlength{\headheight}{15pt}
    
    % Document title
    \title{15.077 Statistical ML and DS}


\begin{document}
    
\begin{figure}
\includegraphics[scale=0.5]{MIT-Logo-(2).jpg}
\end{figure}

\begin{flushright}
\begin{tabular}{ll}
Module Code:     & 15.077                \\
Lecturer:        & Roy Welsch         \\
Coursework:      & PS.03                     \\
Due Date:        & 25 March, 2021      \\ \\
Student Name:    & Chun Hei (Samuel) LAM          \\
MIT ID:          & 928931321              
\end{tabular}
\end{flushright}

\vspace{2cm}

\begin{center}
    \huge 15.077 Statistical Learning and Data Science \\
    \vspace{0.7cm}
    \Large Problem Sheet 3 \\
    \vspace{0.5cm}
    \Large Comparison of Two Samples
    \normalsize
\end{center}

\vspace{1.5cm}

\noindent \textbf{Declaration:}
\begin{tcolorbox}
I pledge that the work submitted for this coursework is my own unassisted work unless stated otherwise.
\end{tcolorbox}
    
\noindent \textbf{Notes from Student:}
\begin{tcolorbox}
Acknowledgement to Harry Yu. Please look at the python notebook attached if there are places that cannot be displayed properly.
\end{tcolorbox}
\newpage

    \begin{tcolorbox}[breakable, size=fbox, boxrule=1pt, pad at break*=1mm,colback=cellbackground, colframe=cellborder]
\prompt{In}{incolor}{83}{\boxspacing}
\begin{Verbatim}[commandchars=\\\{\}]
\PY{k+kn}{import} \PY{n+nn}{numpy} \PY{k}{as} \PY{n+nn}{np}
\PY{k+kn}{from} \PY{n+nn}{scipy}\PY{n+nn}{.}\PY{n+nn}{special} \PY{k+kn}{import} \PY{n}{xlogy}
\PY{k+kn}{from} \PY{n+nn}{scipy} \PY{k+kn}{import} \PY{n}{stats}
\PY{k+kn}{import} \PY{n+nn}{pandas} \PY{k}{as} \PY{n+nn}{pd}
\PY{k+kn}{import} \PY{n+nn}{matplotlib}\PY{n+nn}{.}\PY{n+nn}{pyplot} \PY{k}{as} \PY{n+nn}{plt}
\PY{k+kn}{import} \PY{n+nn}{seaborn} \PY{k}{as} \PY{n+nn}{sns}
\PY{k+kn}{import} \PY{n+nn}{statsmodels}\PY{n+nn}{.}\PY{n+nn}{api} \PY{k}{as} \PY{n+nn}{sm}
\end{Verbatim}
\end{tcolorbox}

    \begin{tcolorbox}[breakable, size=fbox, boxrule=1pt, pad at break*=1mm,colback=cellbackground, colframe=cellborder]
\prompt{In}{incolor}{84}{\boxspacing}
\begin{Verbatim}[commandchars=\\\{\}]
\PY{k+kn}{import} \PY{n+nn}{plotly}\PY{n+nn}{.}\PY{n+nn}{express} \PY{k}{as} \PY{n+nn}{px}
\end{Verbatim}
\end{tcolorbox}

    \hypertarget{rice-10.48-lottery-for-the-military-draft}{%
\section{Rice 10.48: Lottery for the Military
Draft}\label{rice-10.48-lottery-for-the-military-draft}}

In 1970, Congress instituted a lottery for the military draft to support
the unpopular war in Vietnam. All 366 possible birth dates were placed
in plastic capsules in a rotating drum and were selected one by one.
Eligible males born on the first day drawn were first in line to be
drafted followed by those born on the second day drawn, etc. The results
were criticized by some who claimed that government incompetence at
running a fair lottery resulted in a tendency of men born later in the
year being more likely to be drafted. Indeed, later investigation
revealed that the birthdates were placed in the drum by month and were
not thoroughly mixed. The columns of the file \texttt{1970lottery} are
month, month number, day of the year, and draft number.

    \begin{tcolorbox}[breakable, size=fbox, boxrule=1pt, pad at break*=1mm,colback=cellbackground, colframe=cellborder]
\prompt{In}{incolor}{85}{\boxspacing}
\begin{Verbatim}[commandchars=\\\{\}]
\PY{n}{lottery} \PY{o}{=} \PY{n}{pd}\PY{o}{.}\PY{n}{read\PYZus{}csv}\PY{p}{(}\PY{l+s+s2}{\PYZdq{}}\PY{l+s+s2}{./1970lottery.txt}\PY{l+s+s2}{\PYZdq{}}\PY{p}{,} \PY{n}{quotechar}\PY{o}{=}\PY{l+s+s2}{\PYZdq{}}\PY{l+s+s2}{\PYZsq{}}\PY{l+s+s2}{\PYZdq{}}\PY{p}{)}
\PY{n}{lottery1} \PY{o}{=} \PY{n}{lottery}\PY{p}{[}\PY{p}{[}\PY{l+s+s2}{\PYZdq{}}\PY{l+s+s2}{Day\PYZus{}of\PYZus{}year}\PY{l+s+s2}{\PYZdq{}}\PY{p}{,} \PY{l+s+s2}{\PYZdq{}}\PY{l+s+s2}{Draft\PYZus{}No}\PY{l+s+s2}{\PYZdq{}}\PY{p}{]}\PY{p}{]}
\PY{n}{lottery2} \PY{o}{=} \PY{n}{lottery}\PY{p}{[}\PY{p}{[}\PY{l+s+s2}{\PYZdq{}}\PY{l+s+s2}{Month\PYZus{}Number}\PY{l+s+s2}{\PYZdq{}}\PY{p}{,} \PY{l+s+s2}{\PYZdq{}}\PY{l+s+s2}{Draft\PYZus{}No}\PY{l+s+s2}{\PYZdq{}}\PY{p}{]}\PY{p}{]}
\end{Verbatim}
\end{tcolorbox}

    \begin{tcolorbox}[breakable, size=fbox, boxrule=1pt, pad at break*=1mm,colback=cellbackground, colframe=cellborder]
\prompt{In}{incolor}{86}{\boxspacing}
\begin{Verbatim}[commandchars=\\\{\}]
\PY{n}{lottery1}
\end{Verbatim}
\end{tcolorbox}

            \begin{tcolorbox}[breakable, size=fbox, boxrule=.5pt, pad at break*=1mm, opacityfill=0]
\prompt{Out}{outcolor}{86}{\boxspacing}
\begin{Verbatim}[commandchars=\\\{\}]
     Day\_of\_year  Draft\_No
0              1       305
1              2       159
2              3       251
3              4       215
4              5       101
..           {\ldots}       {\ldots}
361          362        78
362          363       123
363          364        16
364          365         3
365          366       100

[366 rows x 2 columns]
\end{Verbatim}
\end{tcolorbox}
        
    \textbf{Part (a)} Plot draft number versus day number. Do you see any
trend?

    \begin{tcolorbox}[breakable, size=fbox, boxrule=1pt, pad at break*=1mm,colback=cellbackground, colframe=cellborder]
\prompt{In}{incolor}{87}{\boxspacing}
\begin{Verbatim}[commandchars=\\\{\}]
\PY{n}{ax1} \PY{o}{=} \PY{n}{lottery}\PY{o}{.}\PY{n}{plot}\PY{o}{.}\PY{n}{scatter}\PY{p}{(}\PY{n}{x}\PY{o}{=}\PY{l+s+s1}{\PYZsq{}}\PY{l+s+s1}{Day\PYZus{}of\PYZus{}year}\PY{l+s+s1}{\PYZsq{}}\PY{p}{,} \PY{n}{y}\PY{o}{=}\PY{l+s+s1}{\PYZsq{}}\PY{l+s+s1}{Draft\PYZus{}No}\PY{l+s+s1}{\PYZsq{}}\PY{p}{,} \PY{n}{c}\PY{o}{=}\PY{l+s+s1}{\PYZsq{}}\PY{l+s+s1}{DarkBlue}\PY{l+s+s1}{\PYZsq{}}\PY{p}{)}
\end{Verbatim}
\end{tcolorbox}

    \begin{center}
    \adjustimage{max size={0.9\linewidth}{0.9\paperheight}}{output_7_0.png}
    \end{center}
    { \hspace*{\fill} \\}
    
    \textbf{Part (b)} Calculate the Pearson and rank correlation
coefficients. What do they suggest?

    \emph{Solution: PMCC}

    \begin{tcolorbox}[breakable, size=fbox, boxrule=1pt, pad at break*=1mm,colback=cellbackground, colframe=cellborder]
\prompt{In}{incolor}{88}{\boxspacing}
\begin{Verbatim}[commandchars=\\\{\}]
\PY{n}{pmcc} \PY{o}{=} \PY{n}{lottery1}\PY{o}{.}\PY{n}{corr}\PY{p}{(}\PY{p}{)}\PY{p}{[}\PY{l+s+s2}{\PYZdq{}}\PY{l+s+s2}{Draft\PYZus{}No}\PY{l+s+s2}{\PYZdq{}}\PY{p}{]}\PY{p}{[}\PY{l+s+s2}{\PYZdq{}}\PY{l+s+s2}{Day\PYZus{}of\PYZus{}year}\PY{l+s+s2}{\PYZdq{}}\PY{p}{]}
\PY{n}{pmcc}
\end{Verbatim}
\end{tcolorbox}

            \begin{tcolorbox}[breakable, size=fbox, boxrule=.5pt, pad at break*=1mm, opacityfill=0]
\prompt{Out}{outcolor}{88}{\boxspacing}
\begin{Verbatim}[commandchars=\\\{\}]
-0.2260414270110072
\end{Verbatim}
\end{tcolorbox}
        
    \emph{Spearman's Rank Correlation Coefficient}

    \begin{tcolorbox}[breakable, size=fbox, boxrule=1pt, pad at break*=1mm,colback=cellbackground, colframe=cellborder]
\prompt{In}{incolor}{89}{\boxspacing}
\begin{Verbatim}[commandchars=\\\{\}]
\PY{n}{srcc} \PY{o}{=} \PY{n}{lottery1}\PY{o}{.}\PY{n}{corr}\PY{p}{(}\PY{n}{method}\PY{o}{=}\PY{l+s+s2}{\PYZdq{}}\PY{l+s+s2}{spearman}\PY{l+s+s2}{\PYZdq{}}\PY{p}{)}\PY{p}{[}\PY{l+s+s2}{\PYZdq{}}\PY{l+s+s2}{Draft\PYZus{}No}\PY{l+s+s2}{\PYZdq{}}\PY{p}{]}\PY{p}{[}\PY{l+s+s2}{\PYZdq{}}\PY{l+s+s2}{Day\PYZus{}of\PYZus{}year}\PY{l+s+s2}{\PYZdq{}}\PY{p}{]}
\PY{n}{srcc}
\end{Verbatim}
\end{tcolorbox}

            \begin{tcolorbox}[breakable, size=fbox, boxrule=.5pt, pad at break*=1mm, opacityfill=0]
\prompt{Out}{outcolor}{89}{\boxspacing}
\begin{Verbatim}[commandchars=\\\{\}]
-0.22580425074264954
\end{Verbatim}
\end{tcolorbox}
        
    \emph{Conclusion: They suggest there is a weak negative correlation
between \texttt{Draft\_No} and \texttt{Day\_of\_year}. It is less likely
for men to be drafted at winter.}

    \textbf{Part (c)} \emph{Is the correlation statistically significant?}
One way to assess this is via a permutation test. Randomly permute the
draft numbers and find the correlation of this random permutation with
the day numbers. Do this 100 times and see how many of the resulting
correlation coefficients exceed the one observed in the data. If you are
not satisfied with 100 times, do it 1,000 times.

    \emph{Solution:}

    \begin{tcolorbox}[breakable, size=fbox, boxrule=1pt, pad at break*=1mm,colback=cellbackground, colframe=cellborder]
\prompt{In}{incolor}{90}{\boxspacing}
\begin{Verbatim}[commandchars=\\\{\}]
\PY{n}{n} \PY{o}{=} \PY{l+m+mi}{10000}
\end{Verbatim}
\end{tcolorbox}

    \begin{tcolorbox}[breakable, size=fbox, boxrule=1pt, pad at break*=1mm,colback=cellbackground, colframe=cellborder]
\prompt{In}{incolor}{91}{\boxspacing}
\begin{Verbatim}[commandchars=\\\{\}]
\PY{n}{samp\PYZus{}corrcoef} \PY{o}{=} \PY{p}{[}\PY{n}{np}\PY{o}{.}\PY{n}{corrcoef}\PY{p}{(}\PY{n}{np}\PY{o}{.}\PY{n}{array}\PY{p}{(}\PY{n+nb}{range}\PY{p}{(}\PY{l+m+mi}{1}\PY{p}{,}\PY{l+m+mi}{367}\PY{p}{)}\PY{p}{)}\PY{p}{,} \PY{n}{np}\PY{o}{.}\PY{n}{random}\PY{o}{.}\PY{n}{permutation}\PY{p}{(}\PY{n}{lottery1}\PY{p}{[}\PY{l+s+s2}{\PYZdq{}}\PY{l+s+s2}{Draft\PYZus{}No}\PY{l+s+s2}{\PYZdq{}}\PY{p}{]}\PY{p}{)}\PY{p}{)}\PY{p}{[}\PY{l+m+mi}{0}\PY{p}{]}\PY{p}{[}\PY{l+m+mi}{1}\PY{p}{]} \PY{k}{for} \PY{n}{i} \PY{o+ow}{in} \PY{n+nb}{range}\PY{p}{(}\PY{n}{n}\PY{p}{)}\PY{p}{]}
\end{Verbatim}
\end{tcolorbox}

    \begin{tcolorbox}[breakable, size=fbox, boxrule=1pt, pad at break*=1mm,colback=cellbackground, colframe=cellborder]
\prompt{In}{incolor}{98}{\boxspacing}
\begin{Verbatim}[commandchars=\\\{\}]
\PY{n}{samp\PYZus{}spearmanr} \PY{o}{=} \PY{p}{[}\PY{n}{stats}\PY{o}{.}\PY{n}{spearmanr}\PY{p}{(}\PY{n}{np}\PY{o}{.}\PY{n}{array}\PY{p}{(}\PY{n+nb}{range}\PY{p}{(}\PY{l+m+mi}{1}\PY{p}{,}\PY{l+m+mi}{367}\PY{p}{)}\PY{p}{)}\PY{p}{,} \PY{n}{np}\PY{o}{.}\PY{n}{random}\PY{o}{.}\PY{n}{permutation}\PY{p}{(}\PY{n}{lottery1}\PY{p}{[}\PY{l+s+s2}{\PYZdq{}}\PY{l+s+s2}{Draft\PYZus{}No}\PY{l+s+s2}{\PYZdq{}}\PY{p}{]}\PY{p}{)}\PY{p}{)}\PY{p}{[}\PY{l+m+mi}{0}\PY{p}{]} \PY{k}{for} \PY{n}{i} \PY{o+ow}{in} \PY{n+nb}{range}\PY{p}{(}\PY{n}{n}\PY{p}{)}\PY{p}{]}
\end{Verbatim}
\end{tcolorbox}

    \begin{tcolorbox}[breakable, size=fbox, boxrule=1pt, pad at break*=1mm,colback=cellbackground, colframe=cellborder]
\prompt{In}{incolor}{100}{\boxspacing}
\begin{Verbatim}[commandchars=\\\{\}]
\PY{n}{fig}\PY{p}{,} \PY{p}{(}\PY{n}{ax1}\PY{p}{,} \PY{n}{ax2}\PY{p}{)} \PY{o}{=} \PY{n}{plt}\PY{o}{.}\PY{n}{subplots}\PY{p}{(}\PY{l+m+mi}{1}\PY{p}{,}\PY{l+m+mi}{2}\PY{p}{,} \PY{n}{figsize}\PY{o}{=}\PY{p}{(}\PY{l+m+mi}{14}\PY{p}{,}\PY{l+m+mi}{5}\PY{p}{)}\PY{p}{)}

\PY{n}{ax1}\PY{o}{.}\PY{n}{hist}\PY{p}{(}\PY{n}{samp\PYZus{}corrcoef}\PY{p}{)}
\PY{n}{ax1}\PY{o}{.}\PY{n}{axvline}\PY{p}{(}\PY{n}{x}\PY{o}{=}\PY{n}{pmcc}\PY{p}{,} \PY{n}{color}\PY{o}{=}\PY{l+s+s1}{\PYZsq{}}\PY{l+s+s1}{r}\PY{l+s+s1}{\PYZsq{}}\PY{p}{)}
\PY{n}{ax1}\PY{o}{.}\PY{n}{set\PYZus{}title}\PY{p}{(}\PY{l+s+sa}{f}\PY{l+s+s2}{\PYZdq{}}\PY{l+s+s2}{distribution of pmcc by permutation, n=}\PY{l+s+si}{\PYZob{}n\PYZcb{}}\PY{l+s+s2}{ }\PY{l+s+se}{\PYZbs{}n}\PY{l+s+s2}{ \PYZsh{} of pmcc exceed observed = }\PY{l+s+s2}{\PYZob{}}\PY{l+s+s2}{np.sum(np.absolute(samp\PYZus{}corrcoef)\PYZlt{}srcc)\PYZcb{}}\PY{l+s+s2}{\PYZdq{}}\PY{p}{)}

\PY{n}{ax2}\PY{o}{.}\PY{n}{hist}\PY{p}{(}\PY{n}{samp\PYZus{}spearmanr}\PY{p}{)}
\PY{n}{ax2}\PY{o}{.}\PY{n}{axvline}\PY{p}{(}\PY{n}{x}\PY{o}{=}\PY{n}{srcc}\PY{p}{,} \PY{n}{color}\PY{o}{=}\PY{l+s+s1}{\PYZsq{}}\PY{l+s+s1}{r}\PY{l+s+s1}{\PYZsq{}}\PY{p}{)}
\PY{n}{ax2}\PY{o}{.}\PY{n}{set\PYZus{}title}\PY{p}{(}\PY{l+s+sa}{f}\PY{l+s+s2}{\PYZdq{}}\PY{l+s+s2}{distribution of pmcc by permutation, n=}\PY{l+s+si}{\PYZob{}n\PYZcb{}}\PY{l+s+s2}{ }\PY{l+s+se}{\PYZbs{}n}\PY{l+s+s2}{ \PYZsh{} of srcc exceed observed = }\PY{l+s+s2}{\PYZob{}}\PY{l+s+s2}{np.sum(np.absolute(samp\PYZus{}spearmanr)\PYZlt{}pmcc)\PYZcb{}}\PY{l+s+s2}{\PYZdq{}} \PY{p}{)}
\end{Verbatim}
\end{tcolorbox}

            \begin{tcolorbox}[breakable, size=fbox, boxrule=.5pt, pad at break*=1mm, opacityfill=0]
\prompt{Out}{outcolor}{100}{\boxspacing}
\begin{Verbatim}[commandchars=\\\{\}]
Text(0.5, 1.0, 'distribution of pmcc by permutation, n=10000 \textbackslash{}n \# of srcc exceed
observed = 0')
\end{Verbatim}
\end{tcolorbox}
        
    \begin{center}
    \adjustimage{max size={0.9\linewidth}{0.9\paperheight}}{output_19_1.png}
    \end{center}
    { \hspace*{\fill} \\}
    
    \emph{Comment: The histograms suggest that the observed \texttt{pmcc}
and \texttt{srcc} are significant in the sense that the test
\(H_0: \rho = 0\) against \(H_0: \rho \neq 0\) can be rejected with a
very lower significant level (where \(\rho\) is population
\texttt{pmcc}/\texttt{srcc}).}

    \textbf{Part (d)} Make parallel boxplots of the draft numbers by month.
Do you see any pattern?

    \emph{Solution: For this question, please open the attached
\texttt{ipynb} notebook to look at the interactive plot.}

    \begin{tcolorbox}[breakable, size=fbox, boxrule=1pt, pad at break*=1mm,colback=cellbackground, colframe=cellborder]
\prompt{In}{incolor}{101}{\boxspacing}
\begin{Verbatim}[commandchars=\\\{\}]
\PY{n}{fig} \PY{o}{=} \PY{n}{px}\PY{o}{.}\PY{n}{box}\PY{p}{(}\PY{n}{lottery2}\PY{p}{,} \PY{n}{x}\PY{o}{=}\PY{l+s+s2}{\PYZdq{}}\PY{l+s+s2}{Month\PYZus{}Number}\PY{l+s+s2}{\PYZdq{}}\PY{p}{,} \PY{n}{y}\PY{o}{=}\PY{l+s+s2}{\PYZdq{}}\PY{l+s+s2}{Draft\PYZus{}No}\PY{l+s+s2}{\PYZdq{}}\PY{p}{)}
\PY{n}{fig}\PY{o}{.}\PY{n}{show}\PY{p}{(}\PY{p}{)}
\end{Verbatim}
\end{tcolorbox}

    
    
    
    
    \emph{Pattern: we see that the median of \texttt{Draft\_No} for
different months are generally decreasing. The interquartile range of
\texttt{Draft\_No} are generally decreasing as well.}

    \textbf{Part (e)} Examine the sampling variability of the two
correlation coefficients (Pearson and rank) using the bootstrap
(re-sampling pairs with replacement) with 100 (or 1000) bootstrap
samples. How does this compare with the permutation approach?

    \begin{tcolorbox}[breakable, size=fbox, boxrule=1pt, pad at break*=1mm,colback=cellbackground, colframe=cellborder]
\prompt{In}{incolor}{102}{\boxspacing}
\begin{Verbatim}[commandchars=\\\{\}]
\PY{n}{n} \PY{o}{=} \PY{l+m+mi}{100}
\end{Verbatim}
\end{tcolorbox}

    \begin{tcolorbox}[breakable, size=fbox, boxrule=1pt, pad at break*=1mm,colback=cellbackground, colframe=cellborder]
\prompt{In}{incolor}{103}{\boxspacing}
\begin{Verbatim}[commandchars=\\\{\}]
\PY{n}{boots\PYZus{}corrcoef} \PY{o}{=} \PY{p}{[}\PY{p}{]}
\PY{n}{boots\PYZus{}spearmanr} \PY{o}{=} \PY{p}{[}\PY{p}{]}

\PY{k}{for} \PY{n}{i} \PY{o+ow}{in} \PY{n+nb}{range}\PY{p}{(}\PY{n}{n}\PY{p}{)}\PY{p}{:}
    \PY{n}{choice} \PY{o}{=} \PY{n}{np}\PY{o}{.}\PY{n}{random}\PY{o}{.}\PY{n}{choice}\PY{p}{(}\PY{l+m+mi}{366}\PY{p}{,}\PY{l+m+mi}{366}\PY{p}{)}
    \PY{n}{lottery\PYZus{}choice} \PY{o}{=} \PY{p}{[}\PY{n}{lottery1}\PY{p}{[}\PY{l+s+s2}{\PYZdq{}}\PY{l+s+s2}{Draft\PYZus{}No}\PY{l+s+s2}{\PYZdq{}}\PY{p}{]}\PY{o}{.}\PY{n}{iloc}\PY{p}{[}\PY{n}{i}\PY{p}{]} \PY{k}{for} \PY{n}{i} \PY{o+ow}{in} \PY{n}{choice}\PY{p}{]}
    \PY{n}{boots\PYZus{}corrcoef} \PY{o}{+}\PY{o}{=} \PY{n}{np}\PY{o}{.}\PY{n}{corrcoef}\PY{p}{(}\PY{n}{choice}\PY{o}{+}\PY{l+m+mi}{1}\PY{p}{,} \PY{n}{lottery\PYZus{}choice}\PY{p}{)}\PY{p}{[}\PY{l+m+mi}{0}\PY{p}{]}\PY{p}{[}\PY{l+m+mi}{1}\PY{p}{]}
    \PY{n}{boots\PYZus{}spearmanr} \PY{o}{+}\PY{o}{=} \PY{n}{stats}\PY{o}{.}\PY{n}{spearmanr}\PY{p}{(}\PY{n}{choice}\PY{o}{+}\PY{l+m+mi}{1}\PY{p}{,} \PY{n}{lottery\PYZus{}choice}\PY{p}{)}\PY{p}{[}\PY{l+m+mi}{0}\PY{p}{]}
\end{Verbatim}
\end{tcolorbox}

    \begin{tcolorbox}[breakable, size=fbox, boxrule=1pt, pad at break*=1mm,colback=cellbackground, colframe=cellborder]
\prompt{In}{incolor}{105}{\boxspacing}
\begin{Verbatim}[commandchars=\\\{\}]
\PY{n}{fig}\PY{p}{,} \PY{p}{(}\PY{n}{ax1}\PY{p}{,} \PY{n}{ax2}\PY{p}{)} \PY{o}{=} \PY{n}{plt}\PY{o}{.}\PY{n}{subplots}\PY{p}{(}\PY{l+m+mi}{1}\PY{p}{,}\PY{l+m+mi}{2}\PY{p}{,} \PY{n}{figsize}\PY{o}{=}\PY{p}{(}\PY{l+m+mi}{14}\PY{p}{,}\PY{l+m+mi}{5}\PY{p}{)}\PY{p}{)}

\PY{n}{ax1}\PY{o}{.}\PY{n}{hist}\PY{p}{(}\PY{n}{samp\PYZus{}corrcoef}\PY{p}{)}
\PY{n}{ax1}\PY{o}{.}\PY{n}{axvline}\PY{p}{(}\PY{n}{x}\PY{o}{=}\PY{n}{pmcc}\PY{p}{,} \PY{n}{color}\PY{o}{=}\PY{l+s+s1}{\PYZsq{}}\PY{l+s+s1}{r}\PY{l+s+s1}{\PYZsq{}}\PY{p}{)}
\PY{n}{ax1}\PY{o}{.}\PY{n}{set\PYZus{}title}\PY{p}{(}\PY{l+s+sa}{f}\PY{l+s+s2}{\PYZdq{}}\PY{l+s+s2}{distribution of pmcc by resampling bootstrap, n=}\PY{l+s+si}{\PYZob{}n\PYZcb{}}\PY{l+s+s2}{ }\PY{l+s+se}{\PYZbs{}n}\PY{l+s+s2}{ \PYZsh{} of pmcc exceed observed = }\PY{l+s+s2}{\PYZob{}}\PY{l+s+s2}{np.sum(np.absolute(samp\PYZus{}corrcoef)\PYZlt{}srcc)\PYZcb{}}\PY{l+s+s2}{\PYZdq{}}\PY{p}{)}

\PY{n}{ax2}\PY{o}{.}\PY{n}{hist}\PY{p}{(}\PY{n}{samp\PYZus{}spearmanr}\PY{p}{)}
\PY{n}{ax2}\PY{o}{.}\PY{n}{axvline}\PY{p}{(}\PY{n}{x}\PY{o}{=}\PY{n}{srcc}\PY{p}{,} \PY{n}{color}\PY{o}{=}\PY{l+s+s1}{\PYZsq{}}\PY{l+s+s1}{r}\PY{l+s+s1}{\PYZsq{}}\PY{p}{)}
\PY{n}{ax2}\PY{o}{.}\PY{n}{set\PYZus{}title}\PY{p}{(}\PY{l+s+sa}{f}\PY{l+s+s2}{\PYZdq{}}\PY{l+s+s2}{distribution of pmcc by resampling bootstrap, n=}\PY{l+s+si}{\PYZob{}n\PYZcb{}}\PY{l+s+s2}{ }\PY{l+s+se}{\PYZbs{}n}\PY{l+s+s2}{ \PYZsh{} of srcc exceed observed = }\PY{l+s+s2}{\PYZob{}}\PY{l+s+s2}{np.sum(np.absolute(samp\PYZus{}spearmanr)\PYZlt{}pmcc)\PYZcb{}}\PY{l+s+s2}{\PYZdq{}} \PY{p}{)}
\end{Verbatim}
\end{tcolorbox}

            \begin{tcolorbox}[breakable, size=fbox, boxrule=.5pt, pad at break*=1mm, opacityfill=0]
\prompt{Out}{outcolor}{105}{\boxspacing}
\begin{Verbatim}[commandchars=\\\{\}]
Text(0.5, 1.0, 'distribution of pmcc by resampling bootstrap, n=100 \textbackslash{}n \# of srcc
exceed observed = 0')
\end{Verbatim}
\end{tcolorbox}
        
    \begin{center}
    \adjustimage{max size={0.9\linewidth}{0.9\paperheight}}{output_28_1.png}
    \end{center}
    { \hspace*{\fill} \\}
    
    \emph{Comment: Again, the histograms suggest that the observed
\texttt{pmcc} and \texttt{srcc} are significant in the sense that the
test \(H_0: \rho = 0\) against \(H_0: \rho \neq 0\) can be rejected with
a very lower significant level (where \(\rho\) is population
\texttt{pmcc}/\texttt{srcc}).}

    \hypertarget{rice-11.15-16-power-of-comparison-of-two-independent-samples}{%
\section{Rice 11.15-16: Power of Comparison of Two Independent
Samples:}\label{rice-11.15-16-power-of-comparison-of-two-independent-samples}}

Suppose that \(n\) measurements are to be taken under a treatment
condition and another \(n\) measurements are to be taken independently
under a control condition. It is thought that the standard deviation of
a single observation is about 10 under both conditions.

    \textbf{Part (a)} How large should \(n\) be so that a 95\% confidence
interval for \(\mu_X - \mu_Y\) has a width of \(2\)? (Use the normal
distribution rather than the \(t\)-distribution, since \(n\) will turn
out to be rather large.)

    \emph{Solution: The length of the asymptotic confidence level is
\(2z_{0.05/2}\sigma_{\bar{X} - \bar{Y}}\), where
\(\sigma_{\bar{X}-\bar{Y}} \approx 10\sqrt{\frac{2}{n}}\). We want the
length to be approximately equal to 2, therefore \begin{equation}
2z_{0.05/2} \times 10\sqrt{\frac{2}{n}} = 2 \iff n = 200 \times z^2_{0.05/2}
\end{equation}}

    \begin{tcolorbox}[breakable, size=fbox, boxrule=1pt, pad at break*=1mm,colback=cellbackground, colframe=cellborder]
\prompt{In}{incolor}{106}{\boxspacing}
\begin{Verbatim}[commandchars=\\\{\}]
\PY{n}{z} \PY{o}{=} \PY{n}{stats}\PY{o}{.}\PY{n}{norm}\PY{o}{.}\PY{n}{ppf}\PY{p}{(}\PY{l+m+mi}{1}\PY{o}{\PYZhy{}}\PY{p}{(}\PY{l+m+mi}{1}\PY{o}{\PYZhy{}}\PY{l+m+mf}{0.95}\PY{p}{)}\PY{o}{/}\PY{l+m+mi}{2}\PY{p}{)}
\PY{n}{n} \PY{o}{=} \PY{n}{np}\PY{o}{.}\PY{n}{floor}\PY{p}{(}\PY{l+m+mi}{200}\PY{o}{*}\PY{p}{(}\PY{n}{z}\PY{o}{*}\PY{o}{*}\PY{l+m+mi}{2}\PY{p}{)} \PY{o}{/}\PY{o}{/} \PY{l+m+mi}{1}\PY{p}{)}\PY{o}{+}\PY{l+m+mi}{1}
\PY{n+nb}{print}\PY{p}{(}\PY{l+s+sa}{f}\PY{l+s+s2}{\PYZdq{}}\PY{l+s+s2}{n should be as large as }\PY{l+s+si}{\PYZob{}n\PYZcb{}}\PY{l+s+s2}{.}\PY{l+s+s2}{\PYZdq{}}\PY{p}{)}
\end{Verbatim}
\end{tcolorbox}

    \begin{Verbatim}[commandchars=\\\{\}]
n should be as large as 769.0.
    \end{Verbatim}

    \textbf{Part (b)} How large should \(n\) be so that the test of
\(H_0: \mu_X = \mu_Y\) against the one-sided alternative
\(H_A: \mu_X > \mu_Y\) has a power of \(0.5\) if \(\mu_X - \mu_Y = 2\)
and \(\alpha = 0.10\)?

    \emph{Solution: This indicates that
\(\mathbb{P}(\text{Type II error}) = 0.5\), in otherwords, we have
\begin{align*}
\mathbb{P}\left( \frac{\bar{X} - \bar{Y}}{\sigma_{\bar{X}-\bar{Y}}} < z_{0.9} \;\bigg|\; \mu_x - \mu_Y = 2 \right) &= 0.5 \\
\implies \mathbb{P}\left( \frac{\bar{X} - \bar{Y} - 2}{\sigma_{\bar{X}-\bar{Y}}} < z_{0.9} - \frac{2}{\sigma_{\bar{X}-\bar{Y}}} \;\bigg|\; \mu_x - \mu_Y = 2 \right) &= 0.5
\end{align*} Of course, this further implies that \begin{equation}
z_{0.9} - \frac{2}{\sigma_{\bar{X}-\bar{Y}}} = 0 \iff 10\sqrt{\frac{2}{n}} = \frac{2}{z_{0.9}} \iff n = 50z^2_{0.9}
\end{equation}}

    \begin{tcolorbox}[breakable, size=fbox, boxrule=1pt, pad at break*=1mm,colback=cellbackground, colframe=cellborder]
\prompt{In}{incolor}{107}{\boxspacing}
\begin{Verbatim}[commandchars=\\\{\}]
\PY{n}{z} \PY{o}{=} \PY{n}{stats}\PY{o}{.}\PY{n}{norm}\PY{o}{.}\PY{n}{ppf}\PY{p}{(}\PY{l+m+mi}{1}\PY{o}{\PYZhy{}}\PY{l+m+mf}{0.1}\PY{p}{)}
\PY{n}{n} \PY{o}{=} \PY{n}{np}\PY{o}{.}\PY{n}{floor}\PY{p}{(}\PY{l+m+mi}{50}\PY{o}{*}\PY{p}{(}\PY{n}{z}\PY{o}{*}\PY{o}{*}\PY{l+m+mi}{2}\PY{p}{)}\PY{p}{)}\PY{o}{+}\PY{l+m+mi}{1}
\PY{n+nb}{print}\PY{p}{(}\PY{l+s+sa}{f}\PY{l+s+s2}{\PYZdq{}}\PY{l+s+s2}{n should be as large as }\PY{l+s+si}{\PYZob{}n\PYZcb{}}\PY{l+s+s2}{.}\PY{l+s+s2}{\PYZdq{}}\PY{p}{)}
\end{Verbatim}
\end{tcolorbox}

    \begin{Verbatim}[commandchars=\\\{\}]
n should be as large as 83.0.
    \end{Verbatim}

    \hypertarget{rice-11.39-telephone-lines}{%
\section{Rice 11.39: Telephone Lines}\label{rice-11.39-telephone-lines}}

An experiment was done to test a method for reducing faults on telephone
lines (Welch 1987). Fourteen matched pairs of areas were used. The
following table shows the fault rates for the control areas and for the
test areas:

    \begin{tcolorbox}[breakable, size=fbox, boxrule=1pt, pad at break*=1mm,colback=cellbackground, colframe=cellborder]
\prompt{In}{incolor}{108}{\boxspacing}
\begin{Verbatim}[commandchars=\\\{\}]
\PY{n}{telephone} \PY{o}{=} \PY{n}{pd}\PY{o}{.}\PY{n}{DataFrame}\PY{p}{(}\PY{p}{\PYZob{}}
    \PY{l+s+s1}{\PYZsq{}}\PY{l+s+s1}{Test}\PY{l+s+s1}{\PYZsq{}} \PY{p}{:} \PY{p}{[}\PY{l+m+mi}{676}\PY{p}{,}\PY{l+m+mi}{206}\PY{p}{,}\PY{l+m+mi}{230}\PY{p}{,}\PY{l+m+mi}{256}\PY{p}{,}\PY{l+m+mi}{280}\PY{p}{,}\PY{l+m+mi}{433}\PY{p}{,}\PY{l+m+mi}{337}\PY{p}{,}\PY{l+m+mi}{466}\PY{p}{,}\PY{l+m+mi}{497}\PY{p}{,}\PY{l+m+mi}{512}\PY{p}{,}\PY{l+m+mi}{794}\PY{p}{,}\PY{l+m+mi}{428}\PY{p}{,}\PY{l+m+mi}{452}\PY{p}{,}\PY{l+m+mi}{512}\PY{p}{]}\PY{p}{,}
    \PY{l+s+s1}{\PYZsq{}}\PY{l+s+s1}{Control}\PY{l+s+s1}{\PYZsq{}} \PY{p}{:} \PY{p}{[}\PY{l+m+mi}{88}\PY{p}{,}\PY{l+m+mi}{570}\PY{p}{,}\PY{l+m+mi}{605}\PY{p}{,}\PY{l+m+mi}{617}\PY{p}{,}\PY{l+m+mi}{653}\PY{p}{,}\PY{l+m+mi}{2913}\PY{p}{,}\PY{l+m+mi}{924}\PY{p}{,}\PY{l+m+mi}{286}\PY{p}{,}\PY{l+m+mi}{1098}\PY{p}{,}\PY{l+m+mi}{982}\PY{p}{,}\PY{l+m+mi}{2346}\PY{p}{,}\PY{l+m+mi}{321}\PY{p}{,}\PY{l+m+mi}{615}\PY{p}{,}\PY{l+m+mi}{519}\PY{p}{]}\PY{p}{\PYZcb{}}\PY{p}{)}
\PY{n}{telephone}\PY{o}{.}\PY{n}{T}
\end{Verbatim}
\end{tcolorbox}

            \begin{tcolorbox}[breakable, size=fbox, boxrule=.5pt, pad at break*=1mm, opacityfill=0]
\prompt{Out}{outcolor}{108}{\boxspacing}
\begin{Verbatim}[commandchars=\\\{\}]
          0    1    2    3    4     5    6    7     8    9     10   11   12  \textbackslash{}
Test     676  206  230  256  280   433  337  466   497  512   794  428  452
Control   88  570  605  617  653  2913  924  286  1098  982  2346  321  615

          13
Test     512
Control  519
\end{Verbatim}
\end{tcolorbox}
        
    \textbf{Part (a)} Plot the differences versus the control rate and
summarize what you see.

    \emph{Solution: Plot}

    \begin{tcolorbox}[breakable, size=fbox, boxrule=1pt, pad at break*=1mm,colback=cellbackground, colframe=cellborder]
\prompt{In}{incolor}{109}{\boxspacing}
\begin{Verbatim}[commandchars=\\\{\}]
\PY{n}{plt}\PY{o}{.}\PY{n}{scatter}\PY{p}{(}\PY{n}{telephone}\PY{o}{.}\PY{n}{Control}\PY{p}{,} \PY{n}{telephone}\PY{o}{.}\PY{n}{Test} \PY{o}{\PYZhy{}} \PY{n}{telephone}\PY{o}{.}\PY{n}{Control}\PY{p}{)}
\PY{n}{plt}\PY{o}{.}\PY{n}{xlabel}\PY{p}{(}\PY{l+s+s2}{\PYZdq{}}\PY{l+s+s2}{Control}\PY{l+s+s2}{\PYZdq{}}\PY{p}{)}
\PY{n}{plt}\PY{o}{.}\PY{n}{ylabel}\PY{p}{(}\PY{l+s+s2}{\PYZdq{}}\PY{l+s+s2}{Difference}\PY{l+s+s2}{\PYZdq{}}\PY{p}{)}
\PY{n}{plt}\PY{o}{.}\PY{n}{title}\PY{p}{(}\PY{l+s+s2}{\PYZdq{}}\PY{l+s+s2}{Difference vs. Control}\PY{l+s+s2}{\PYZdq{}}\PY{p}{)}
\end{Verbatim}
\end{tcolorbox}

            \begin{tcolorbox}[breakable, size=fbox, boxrule=.5pt, pad at break*=1mm, opacityfill=0]
\prompt{Out}{outcolor}{109}{\boxspacing}
\begin{Verbatim}[commandchars=\\\{\}]
Text(0.5, 1.0, 'Difference vs. Control')
\end{Verbatim}
\end{tcolorbox}
        
    \begin{center}
    \adjustimage{max size={0.9\linewidth}{0.9\paperheight}}{output_41_1.png}
    \end{center}
    { \hspace*{\fill} \\}
    
    \emph{Comment: seems like there is a positive correlation between the
difference of rates and control rates - the higher the control rates
are, the significant reduction of fault rates are.}

    \textbf{Part (b)} Calculate the mean difference, its standard deviation,
and a confidence interval.

    *Solution: We compute the difference.

    \begin{tcolorbox}[breakable, size=fbox, boxrule=1pt, pad at break*=1mm,colback=cellbackground, colframe=cellborder]
\prompt{In}{incolor}{110}{\boxspacing}
\begin{Verbatim}[commandchars=\\\{\}]
\PY{n}{diff} \PY{o}{=} \PY{n}{telephone}\PY{o}{.}\PY{n}{Test} \PY{o}{\PYZhy{}} \PY{n}{telephone}\PY{o}{.}\PY{n}{Control}
\end{Verbatim}
\end{tcolorbox}

    \emph{First Calculate the mean and standard deviation.}

    \begin{tcolorbox}[breakable, size=fbox, boxrule=1pt, pad at break*=1mm,colback=cellbackground, colframe=cellborder]
\prompt{In}{incolor}{111}{\boxspacing}
\begin{Verbatim}[commandchars=\\\{\}]
\PY{n}{mean} \PY{o}{=} \PY{n}{diff}\PY{o}{.}\PY{n}{mean}\PY{p}{(}\PY{p}{)}
\PY{n}{stdev} \PY{o}{=} \PY{n}{diff}\PY{o}{.}\PY{n}{std}\PY{p}{(}\PY{p}{)}
\PY{n+nb}{print}\PY{p}{(}\PY{l+s+sa}{f}\PY{l+s+s2}{\PYZdq{}}\PY{l+s+s2}{mean = }\PY{l+s+s2}{\PYZob{}}\PY{l+s+s2}{np.round(mean,3)\PYZcb{}, standard deviation = }\PY{l+s+s2}{\PYZob{}}\PY{l+s+s2}{np.round(stdev,3)\PYZcb{}}\PY{l+s+s2}{\PYZdq{}}\PY{p}{)}
\end{Verbatim}
\end{tcolorbox}

    \begin{Verbatim}[commandchars=\\\{\}]
mean = -461.286, standard deviation = 757.809
    \end{Verbatim}

    \emph{Then we compute the confidence interval.}

    \begin{tcolorbox}[breakable, size=fbox, boxrule=1pt, pad at break*=1mm,colback=cellbackground, colframe=cellborder]
\prompt{In}{incolor}{168}{\boxspacing}
\begin{Verbatim}[commandchars=\\\{\}]
\PY{n}{z1} \PY{o}{=} \PY{n}{stats}\PY{o}{.}\PY{n}{t}\PY{o}{.}\PY{n}{ppf}\PY{p}{(}\PY{l+m+mf}{0.025}\PY{p}{,} \PY{n}{df}\PY{o}{=}\PY{l+m+mi}{14}\PY{o}{\PYZhy{}}\PY{l+m+mi}{1}\PY{p}{)}
\PY{n}{z2} \PY{o}{=} \PY{n}{stats}\PY{o}{.}\PY{n}{t}\PY{o}{.}\PY{n}{ppf}\PY{p}{(}\PY{l+m+mf}{0.975}\PY{p}{,} \PY{n}{df}\PY{o}{=}\PY{l+m+mi}{14}\PY{o}{\PYZhy{}}\PY{l+m+mi}{1}\PY{p}{)}
\PY{n}{lower} \PY{o}{=} \PY{n}{np}\PY{o}{.}\PY{n}{round}\PY{p}{(}\PY{n}{mean}\PY{o}{+}\PY{p}{(}\PY{n}{stdev}\PY{o}{*}\PY{n}{z1}\PY{o}{/}\PY{n}{np}\PY{o}{.}\PY{n}{sqrt}\PY{p}{(}\PY{l+m+mi}{14}\PY{p}{)}\PY{p}{)}\PY{p}{,}\PY{l+m+mi}{3}\PY{p}{)}
\PY{n}{upper} \PY{o}{=} \PY{n}{np}\PY{o}{.}\PY{n}{round}\PY{p}{(}\PY{n}{mean}\PY{o}{+}\PY{p}{(}\PY{n}{stdev}\PY{o}{*}\PY{n}{z2}\PY{o}{/}\PY{n}{np}\PY{o}{.}\PY{n}{sqrt}\PY{p}{(}\PY{l+m+mi}{14}\PY{p}{)}\PY{p}{)}\PY{p}{,}\PY{l+m+mi}{3}\PY{p}{)}
\PY{n+nb}{print}\PY{p}{(}\PY{l+s+sa}{f}\PY{l+s+s2}{\PYZdq{}}\PY{l+s+s2}{95}\PY{l+s+si}{\PYZpc{} c}\PY{l+s+s2}{onfidence interval is [}\PY{l+s+si}{\PYZob{}lower\PYZcb{}}\PY{l+s+s2}{,}\PY{l+s+si}{\PYZob{}upper\PYZcb{}}\PY{l+s+s2}{]}\PY{l+s+s2}{\PYZdq{}}\PY{p}{)}
\end{Verbatim}
\end{tcolorbox}

    \begin{Verbatim}[commandchars=\\\{\}]
95\% confidence interval is [-898.832,-23.74]
    \end{Verbatim}

    \textbf{Part (c)} Calculate the median difference and a confidence
interval and compare to the previous result.

    \emph{Solution: First compute the median.}

    \begin{tcolorbox}[breakable, size=fbox, boxrule=1pt, pad at break*=1mm,colback=cellbackground, colframe=cellborder]
\prompt{In}{incolor}{113}{\boxspacing}
\begin{Verbatim}[commandchars=\\\{\}]
\PY{n}{median} \PY{o}{=} \PY{n}{diff}\PY{o}{.}\PY{n}{median}\PY{p}{(}\PY{p}{)}
\PY{n+nb}{print}\PY{p}{(}\PY{l+s+sa}{f}\PY{l+s+s2}{\PYZdq{}}\PY{l+s+s2}{median = }\PY{l+s+si}{\PYZob{}median\PYZcb{}}\PY{l+s+s2}{\PYZdq{}}\PY{p}{)}
\end{Verbatim}
\end{tcolorbox}

    \begin{Verbatim}[commandchars=\\\{\}]
median = -368.5
    \end{Verbatim}

    \emph{We choose \(\alpha=0.05\) and construct a confidence interval. We
first sort the data.}

    \begin{tcolorbox}[breakable, size=fbox, boxrule=1pt, pad at break*=1mm,colback=cellbackground, colframe=cellborder]
\prompt{In}{incolor}{114}{\boxspacing}
\begin{Verbatim}[commandchars=\\\{\}]
\PY{n}{sorted\PYZus{}diff} \PY{o}{=} \PY{n}{diff}\PY{o}{.}\PY{n}{sort\PYZus{}values}\PY{p}{(}\PY{n}{ascending}\PY{o}{=}\PY{k+kc}{True}\PY{p}{)}
\PY{n}{sorted\PYZus{}diff}
\end{Verbatim}
\end{tcolorbox}

            \begin{tcolorbox}[breakable, size=fbox, boxrule=.5pt, pad at break*=1mm, opacityfill=0]
\prompt{Out}{outcolor}{114}{\boxspacing}
\begin{Verbatim}[commandchars=\\\{\}]
5    -2480
10   -1552
8     -601
6     -587
9     -470
2     -375
4     -373
1     -364
3     -361
12    -163
13      -7
11     107
7      180
0      588
dtype: int64
\end{Verbatim}
\end{tcolorbox}
        
    \emph{Then we find \(k\) such that \(\mathbb{P}(N \leq k-1)\) is closest
to 0.025, where \(N \sim \text{B}(14,0.5)\).}

    \begin{tcolorbox}[breakable, size=fbox, boxrule=1pt, pad at break*=1mm,colback=cellbackground, colframe=cellborder]
\prompt{In}{incolor}{115}{\boxspacing}
\begin{Verbatim}[commandchars=\\\{\}]
\PY{n}{stats}\PY{o}{.}\PY{n}{binom}\PY{o}{.}\PY{n}{ppf}\PY{p}{(}\PY{l+m+mf}{0.025}\PY{p}{,} \PY{l+m+mi}{14}\PY{p}{,} \PY{l+m+mf}{0.5}\PY{p}{)}
\end{Verbatim}
\end{tcolorbox}

            \begin{tcolorbox}[breakable, size=fbox, boxrule=.5pt, pad at break*=1mm, opacityfill=0]
\prompt{Out}{outcolor}{115}{\boxspacing}
\begin{Verbatim}[commandchars=\\\{\}]
3.0
\end{Verbatim}
\end{tcolorbox}
        
    \begin{tcolorbox}[breakable, size=fbox, boxrule=1pt, pad at break*=1mm,colback=cellbackground, colframe=cellborder]
\prompt{In}{incolor}{116}{\boxspacing}
\begin{Verbatim}[commandchars=\\\{\}]
\PY{n}{stats}\PY{o}{.}\PY{n}{binom}\PY{o}{.}\PY{n}{cdf}\PY{p}{(}\PY{l+m+mi}{2}\PY{p}{,} \PY{l+m+mi}{14}\PY{p}{,} \PY{l+m+mf}{0.5}\PY{p}{)}\PY{p}{,} \PY{n}{stats}\PY{o}{.}\PY{n}{binom}\PY{o}{.}\PY{n}{cdf}\PY{p}{(}\PY{l+m+mi}{3}\PY{p}{,} \PY{l+m+mi}{14}\PY{p}{,} \PY{l+m+mf}{0.5}\PY{p}{)}\PY{p}{,} \PY{n}{stats}\PY{o}{.}\PY{n}{binom}\PY{o}{.}\PY{n}{cdf}\PY{p}{(}\PY{l+m+mi}{4}\PY{p}{,} \PY{l+m+mi}{14}\PY{p}{,} \PY{l+m+mf}{0.5}\PY{p}{)}
\end{Verbatim}
\end{tcolorbox}

            \begin{tcolorbox}[breakable, size=fbox, boxrule=.5pt, pad at break*=1mm, opacityfill=0]
\prompt{Out}{outcolor}{116}{\boxspacing}
\begin{Verbatim}[commandchars=\\\{\}]
(0.006469726562499999, 0.02868652343750001, 0.08978271484375001)
\end{Verbatim}
\end{tcolorbox}
        
    \emph{Clearly \(k=3+1=4\) is our optimal choice. Therefore the lower
end-point is the \texttt{4}-th entry, while the upper end-point is
\texttt{14-4+1=11}-th entry.}

    \begin{tcolorbox}[breakable, size=fbox, boxrule=1pt, pad at break*=1mm,colback=cellbackground, colframe=cellborder]
\prompt{In}{incolor}{172}{\boxspacing}
\begin{Verbatim}[commandchars=\\\{\}]
\PY{n}{lower\PYZus{}b} \PY{o}{=} \PY{n}{sorted\PYZus{}diff}\PY{o}{.}\PY{n}{iloc}\PY{p}{[}\PY{l+m+mi}{4}\PY{o}{\PYZhy{}}\PY{l+m+mi}{1}\PY{p}{]}
\PY{n}{upper\PYZus{}b} \PY{o}{=} \PY{n}{sorted\PYZus{}diff}\PY{o}{.}\PY{n}{iloc}\PY{p}{[}\PY{l+m+mi}{11}\PY{o}{\PYZhy{}}\PY{l+m+mi}{1}\PY{p}{]}
\PY{n+nb}{print}\PY{p}{(}\PY{l+s+sa}{f}\PY{l+s+s2}{\PYZdq{}}\PY{l+s+s2}{95}\PY{l+s+si}{\PYZpc{} c}\PY{l+s+s2}{onfidence interval is [}\PY{l+s+si}{\PYZob{}lower\PYZus{}b\PYZcb{}}\PY{l+s+s2}{,}\PY{l+s+si}{\PYZob{}upper\PYZus{}b\PYZcb{}}\PY{l+s+s2}{]}\PY{l+s+s2}{\PYZdq{}}\PY{p}{)}
\end{Verbatim}
\end{tcolorbox}

    \begin{Verbatim}[commandchars=\\\{\}]
95\% confidence interval is [-587,-7]
    \end{Verbatim}

    \emph{Comment: The non-parametric confidence interval is much wider
since it assumes nothing about distribution of population. Turns out our
data is more right-skewed, resulting in longer confidence interval.}

    \textbf{Part (d)} Do you think it is more appropriate to use a
\(t\)-test or a nonparametric method to test whether the apparent
difference between test and control could be due to chance? Why? Carry
out both tests and compare.

    \emph{Solution: Again set \(H_0\) be there is no difference in rate, and
\(H_1\) be that there is (systematic) reduction in rate. Again we take
\(\alpha = 0.05\).}

\emph{We first perform \(t\)-test. We compute the \(t\) statistic and
its p-value.}

    \begin{tcolorbox}[breakable, size=fbox, boxrule=1pt, pad at break*=1mm,colback=cellbackground, colframe=cellborder]
\prompt{In}{incolor}{118}{\boxspacing}
\begin{Verbatim}[commandchars=\\\{\}]
\PY{n}{t} \PY{o}{=} \PY{n}{mean} \PY{o}{/} \PY{p}{(}\PY{n}{stdev}\PY{o}{/}\PY{n}{np}\PY{o}{.}\PY{n}{sqrt}\PY{p}{(}\PY{l+m+mi}{14}\PY{p}{)}\PY{p}{)}
\PY{n}{p} \PY{o}{=} \PY{n}{stats}\PY{o}{.}\PY{n}{t}\PY{o}{.}\PY{n}{cdf}\PY{p}{(}\PY{n}{t}\PY{p}{,} \PY{n}{df}\PY{o}{=}\PY{l+m+mi}{14}\PY{o}{\PYZhy{}}\PY{l+m+mi}{1}\PY{p}{)}

\PY{k}{if} \PY{n}{p} \PY{o}{\PYZlt{}} \PY{l+m+mf}{0.05}\PY{p}{:}
    \PY{n}{test\PYZus{}result} \PY{o}{=} \PY{l+s+s2}{\PYZdq{}}\PY{l+s+s2}{reject H0.}\PY{l+s+s2}{\PYZdq{}}
\PY{k}{else}\PY{p}{:}
    \PY{n}{test\PYZus{}result} \PY{o}{=} \PY{l+s+s2}{\PYZdq{}}\PY{l+s+s2}{insufficient evidence to reject H0.}\PY{l+s+s2}{\PYZdq{}}

\PY{n+nb}{print}\PY{p}{(}\PY{l+s+sa}{f}\PY{l+s+s2}{\PYZdq{}}\PY{l+s+s2}{t statistic = }\PY{l+s+s2}{\PYZob{}}\PY{l+s+s2}{np.round(t,4)\PYZcb{}, p\PYZhy{}value = }\PY{l+s+s2}{\PYZob{}}\PY{l+s+s2}{np.round(p,4)\PYZcb{}}\PY{l+s+s2}{\PYZdq{}} \PY{o}{+} \PY{l+s+s2}{\PYZdq{}}\PY{l+s+s2}{, }\PY{l+s+s2}{\PYZdq{}} \PY{o}{+} \PY{n}{test\PYZus{}result}\PY{p}{)}
\end{Verbatim}
\end{tcolorbox}

    \begin{Verbatim}[commandchars=\\\{\}]
t statistic = -2.2776, p-value = 0.0201, reject H0.
    \end{Verbatim}

    \emph{Non-parametric method.}

    \begin{tcolorbox}[breakable, size=fbox, boxrule=1pt, pad at break*=1mm,colback=cellbackground, colframe=cellborder]
\prompt{In}{incolor}{119}{\boxspacing}
\begin{Verbatim}[commandchars=\\\{\}]
\PY{n}{w}\PY{p}{,} \PY{n}{p} \PY{o}{=} \PY{n}{stats}\PY{o}{.}\PY{n}{wilcoxon}\PY{p}{(}\PY{n}{diff}\PY{p}{,} \PY{n}{alternative}\PY{o}{=}\PY{l+s+s2}{\PYZdq{}}\PY{l+s+s2}{less}\PY{l+s+s2}{\PYZdq{}}\PY{p}{,} \PY{n}{mode}\PY{o}{=}\PY{l+s+s2}{\PYZdq{}}\PY{l+s+s2}{exact}\PY{l+s+s2}{\PYZdq{}}\PY{p}{)}

\PY{k}{if} \PY{n}{p} \PY{o}{\PYZlt{}} \PY{l+m+mf}{0.05}\PY{p}{:}
    \PY{n}{test\PYZus{}result} \PY{o}{=} \PY{l+s+s2}{\PYZdq{}}\PY{l+s+s2}{reject H0.}\PY{l+s+s2}{\PYZdq{}}
\PY{k}{else}\PY{p}{:}
    \PY{n}{test\PYZus{}result} \PY{o}{=} \PY{l+s+s2}{\PYZdq{}}\PY{l+s+s2}{insufficient evidence to reject H0.}\PY{l+s+s2}{\PYZdq{}}

\PY{n+nb}{print}\PY{p}{(}\PY{l+s+sa}{f}\PY{l+s+s2}{\PYZdq{}}\PY{l+s+s2}{Wilcoxon statistic = }\PY{l+s+s2}{\PYZob{}}\PY{l+s+s2}{np.round(w,4)\PYZcb{}, p\PYZhy{}value = }\PY{l+s+s2}{\PYZob{}}\PY{l+s+s2}{np.round(p,4)\PYZcb{}}\PY{l+s+s2}{\PYZdq{}} \PY{o}{+} \PY{l+s+s2}{\PYZdq{}}\PY{l+s+s2}{, }\PY{l+s+s2}{\PYZdq{}} \PY{o}{+} \PY{n}{test\PYZus{}result}\PY{p}{)}
\end{Verbatim}
\end{tcolorbox}

    \begin{Verbatim}[commandchars=\\\{\}]
Wilcoxon statistic = 17.0, p-value = 0.0123, reject H0.
    \end{Verbatim}

    \emph{Comparision: both tests suggest there is sufficient evidence to
reject \texttt{H0}. I would prefer exact Wilcoxon signed-rank test
because it is more robust when handling data with outlier, like the one
in our case.}

    \hypertarget{rice-11.46---cloud-seeding}{%
\section{Rice 11.46 - Cloud Seeding}\label{rice-11.46---cloud-seeding}}

    The National Weather Bureau's ACN cloud-seeding project was carried out
in the states of Oregon and Washington. Cloud seeding was accomplished
by dispersing dry ice from an aircraft; only clouds that were deemed
``ripe'' for seeding were candidates for seeding. On each occasion, a
decision was made at random whether to seed, the probability of seeding
being \(\frac{2}{3}\). This resulted in \(22\) seeded and \(13\) control
cases. Three types of targets were considered, two of which are dealt
with in this problem. Type I targets were large geographical areas
downwind from the seeding; type II targets were sections of type I
targets located so as to have, theoretically, the greatest sensitivity
to cloud seeding. The following table gives the average target rainfalls
(in inches) for the seeded and control cases, listed in chronological
order. Is there evidence that seeding has an effect on either type of
target?

    \begin{tcolorbox}[breakable, size=fbox, boxrule=1pt, pad at break*=1mm,colback=cellbackground, colframe=cellborder]
\prompt{In}{incolor}{120}{\boxspacing}
\begin{Verbatim}[commandchars=\\\{\}]
\PY{n}{control} \PY{o}{=} \PY{n}{pd}\PY{o}{.}\PY{n}{DataFrame}\PY{p}{(}\PY{p}{\PYZob{}}
    \PY{l+s+s1}{\PYZsq{}}\PY{l+s+s1}{Type I}\PY{l+s+s1}{\PYZsq{}} \PY{p}{:} \PY{p}{[}\PY{l+m+mf}{0.0080}\PY{p}{,}\PY{l+m+mf}{0.0046}\PY{p}{,}\PY{l+m+mf}{0.0549}\PY{p}{,}\PY{l+m+mf}{0.1313}\PY{p}{,}\PY{l+m+mf}{0.0587}\PY{p}{,}\PY{l+m+mf}{0.1723}\PY{p}{,}\PY{l+m+mf}{0.3812}\PY{p}{,}\PY{l+m+mf}{0.1720}\PY{p}{,}\PY{l+m+mf}{0.1182}\PY{p}{,}\PY{l+m+mf}{0.1383}\PY{p}{,}\PY{l+m+mf}{0.0106}\PY{p}{,}\PY{l+m+mf}{0.2126}\PY{p}{,}\PY{l+m+mf}{0.1435}\PY{p}{]}\PY{p}{,}
    \PY{l+s+s1}{\PYZsq{}}\PY{l+s+s1}{Type II}\PY{l+s+s1}{\PYZsq{}} \PY{p}{:} \PY{p}{[}\PY{l+m+mf}{0.0000}\PY{p}{,}\PY{l+m+mf}{0.0000}\PY{p}{,}\PY{l+m+mf}{0.0053}\PY{p}{,}\PY{l+m+mf}{0.0920}\PY{p}{,}\PY{l+m+mf}{0.0220}\PY{p}{,}\PY{l+m+mf}{0.1133}\PY{p}{,}\PY{l+m+mf}{0.2880}\PY{p}{,}\PY{l+m+mf}{0.0000}\PY{p}{,}\PY{l+m+mf}{0.1058}\PY{p}{,}\PY{l+m+mf}{0.2050}\PY{p}{,}\PY{l+m+mf}{0.0100}\PY{p}{,}\PY{l+m+mf}{0.2450}\PY{p}{,}\PY{l+m+mf}{0.1529}\PY{p}{]}\PY{p}{\PYZcb{}}\PY{p}{)}
\PY{n}{control}\PY{o}{.}\PY{n}{head}\PY{p}{(}\PY{l+m+mi}{4}\PY{p}{)}
\end{Verbatim}
\end{tcolorbox}

            \begin{tcolorbox}[breakable, size=fbox, boxrule=.5pt, pad at break*=1mm, opacityfill=0]
\prompt{Out}{outcolor}{120}{\boxspacing}
\begin{Verbatim}[commandchars=\\\{\}]
   Type I  Type II
0  0.0080   0.0000
1  0.0046   0.0000
2  0.0549   0.0053
3  0.1313   0.0920
\end{Verbatim}
\end{tcolorbox}
        
    \begin{tcolorbox}[breakable, size=fbox, boxrule=1pt, pad at break*=1mm,colback=cellbackground, colframe=cellborder]
\prompt{In}{incolor}{121}{\boxspacing}
\begin{Verbatim}[commandchars=\\\{\}]
\PY{n}{seeded} \PY{o}{=} \PY{n}{pd}\PY{o}{.}\PY{n}{DataFrame}\PY{p}{(}\PY{p}{\PYZob{}}
    \PY{l+s+s1}{\PYZsq{}}\PY{l+s+s1}{Type I}\PY{l+s+s1}{\PYZsq{}} \PY{p}{:} \PY{p}{[}\PY{l+m+mf}{0.1218}\PY{p}{,}\PY{l+m+mf}{0.0403}\PY{p}{,}\PY{l+m+mf}{0.1166}\PY{p}{,}\PY{l+m+mf}{0.2375}\PY{p}{,}\PY{l+m+mf}{0.1256}\PY{p}{,}\PY{l+m+mf}{0.1400}\PY{p}{,}\PY{l+m+mf}{0.2439}\PY{p}{,}\PY{l+m+mf}{0.0072}\PY{p}{,}\PY{l+m+mf}{0.0707}\PY{p}{,}\PY{l+m+mf}{0.1036}\PY{p}{,}\PY{l+m+mf}{0.1632}\PY{p}{,}\PY{l+m+mf}{0.0788}\PY{p}{,}\PY{l+m+mf}{0.0365}\PY{p}{,}\PY{l+m+mf}{0.2409}\PY{p}{,}\PY{l+m+mf}{0.0408}\PY{p}{,}\PY{l+m+mf}{0.2204}\PY{p}{,}\PY{l+m+mf}{0.1847}\PY{p}{,}\PY{l+m+mf}{0.3332}\PY{p}{,}\PY{l+m+mf}{0.0676}\PY{p}{,}\PY{l+m+mf}{0.1097}\PY{p}{,}\PY{l+m+mf}{0.0952}\PY{p}{,}\PY{l+m+mf}{0.2095}\PY{p}{]}\PY{p}{,}
    \PY{l+s+s1}{\PYZsq{}}\PY{l+s+s1}{Type II}\PY{l+s+s1}{\PYZsq{}} \PY{p}{:} \PY{p}{[}\PY{l+m+mf}{0.0200}\PY{p}{,}\PY{l+m+mf}{0.0163}\PY{p}{,}\PY{l+m+mf}{0.1560}\PY{p}{,}\PY{l+m+mf}{0.2885}\PY{p}{,}\PY{l+m+mf}{0.1483}\PY{p}{,}\PY{l+m+mf}{0.1019}\PY{p}{,}\PY{l+m+mf}{0.1867}\PY{p}{,}\PY{l+m+mf}{0.0233}\PY{p}{,}\PY{l+m+mf}{0.1067}\PY{p}{,}\PY{l+m+mf}{0.1011}\PY{p}{,}\PY{l+m+mf}{0.2407}\PY{p}{,}\PY{l+m+mf}{0.0666}\PY{p}{,}\PY{l+m+mf}{0.0133}\PY{p}{,}\PY{l+m+mf}{0.2897}\PY{p}{,}\PY{l+m+mf}{0.0425}\PY{p}{,}\PY{l+m+mf}{0.2191}\PY{p}{,}\PY{l+m+mf}{0.0789}\PY{p}{,}\PY{l+m+mf}{0.3570}\PY{p}{,}\PY{l+m+mf}{0.0760}\PY{p}{,}\PY{l+m+mf}{0.0913}\PY{p}{,}\PY{l+m+mf}{0.0400}\PY{p}{,}\PY{l+m+mf}{0.1467}\PY{p}{]}\PY{p}{\PYZcb{}}\PY{p}{)}
\PY{n}{seeded}\PY{o}{.}\PY{n}{head}\PY{p}{(}\PY{l+m+mi}{4}\PY{p}{)}
\end{Verbatim}
\end{tcolorbox}

            \begin{tcolorbox}[breakable, size=fbox, boxrule=.5pt, pad at break*=1mm, opacityfill=0]
\prompt{Out}{outcolor}{121}{\boxspacing}
\begin{Verbatim}[commandchars=\\\{\}]
   Type I  Type II
0  0.1218   0.0200
1  0.0403   0.0163
2  0.1166   0.1560
3  0.2375   0.2885
\end{Verbatim}
\end{tcolorbox}
        
    \emph{Solution: we set \(H_0\) be the hypothesis that there is no
increase in rainfall, and \(H_1\) be the hypothesis that there is an
increase in rainfall.}

    \emph{For Type I target.}

    \begin{tcolorbox}[breakable, size=fbox, boxrule=1pt, pad at break*=1mm,colback=cellbackground, colframe=cellborder]
\prompt{In}{incolor}{176}{\boxspacing}
\begin{Verbatim}[commandchars=\\\{\}]
\PY{n}{U}\PY{p}{,} \PY{n}{p} \PY{o}{=} \PY{n}{stats}\PY{o}{.}\PY{n}{mannwhitneyu}\PY{p}{(}\PY{n}{control}\PY{p}{[}\PY{l+s+s2}{\PYZdq{}}\PY{l+s+s2}{Type I}\PY{l+s+s2}{\PYZdq{}}\PY{p}{]}\PY{p}{,} \PY{n}{seeded}\PY{p}{[}\PY{l+s+s2}{\PYZdq{}}\PY{l+s+s2}{Type I}\PY{l+s+s2}{\PYZdq{}}\PY{p}{]}\PY{p}{,} \PY{n}{alternative}\PY{o}{=}\PY{l+s+s2}{\PYZdq{}}\PY{l+s+s2}{less}\PY{l+s+s2}{\PYZdq{}}\PY{p}{)}

\PY{k}{if} \PY{n}{p} \PY{o}{\PYZlt{}} \PY{l+m+mf}{0.05}\PY{p}{:}
    \PY{n}{test\PYZus{}result} \PY{o}{=} \PY{l+s+s2}{\PYZdq{}}\PY{l+s+s2}{reject H0.}\PY{l+s+s2}{\PYZdq{}}
\PY{k}{else}\PY{p}{:}
    \PY{n}{test\PYZus{}result} \PY{o}{=} \PY{l+s+s2}{\PYZdq{}}\PY{l+s+s2}{insufficient evidence to reject H0.}\PY{l+s+s2}{\PYZdq{}}

\PY{n+nb}{print}\PY{p}{(}\PY{l+s+sa}{f}\PY{l+s+s2}{\PYZdq{}}\PY{l+s+s2}{Mann\PYZhy{}Whitney statistic = }\PY{l+s+s2}{\PYZob{}}\PY{l+s+s2}{np.round(U,4)\PYZcb{}, p\PYZhy{}value = }\PY{l+s+s2}{\PYZob{}}\PY{l+s+s2}{np.round(p,4)\PYZcb{}}\PY{l+s+s2}{\PYZdq{}} \PY{o}{+} \PY{l+s+s2}{\PYZdq{}}\PY{l+s+s2}{, }\PY{l+s+s2}{\PYZdq{}} \PY{o}{+} \PY{n}{test\PYZus{}result}\PY{p}{)}
\end{Verbatim}
\end{tcolorbox}

    \begin{Verbatim}[commandchars=\\\{\}]
Mann-Whitney statistic = 130.0, p-value = 0.3348, insufficient evidence to
reject H0.
    \end{Verbatim}

    \emph{For Type II target.}

    \begin{tcolorbox}[breakable, size=fbox, boxrule=1pt, pad at break*=1mm,colback=cellbackground, colframe=cellborder]
\prompt{In}{incolor}{177}{\boxspacing}
\begin{Verbatim}[commandchars=\\\{\}]
\PY{n}{U}\PY{p}{,} \PY{n}{p} \PY{o}{=} \PY{n}{stats}\PY{o}{.}\PY{n}{mannwhitneyu}\PY{p}{(}\PY{n}{control}\PY{p}{[}\PY{l+s+s2}{\PYZdq{}}\PY{l+s+s2}{Type II}\PY{l+s+s2}{\PYZdq{}}\PY{p}{]}\PY{p}{,} \PY{n}{seeded}\PY{p}{[}\PY{l+s+s2}{\PYZdq{}}\PY{l+s+s2}{Type II}\PY{l+s+s2}{\PYZdq{}}\PY{p}{]}\PY{p}{,} \PY{n}{alternative}\PY{o}{=}\PY{l+s+s2}{\PYZdq{}}\PY{l+s+s2}{less}\PY{l+s+s2}{\PYZdq{}}\PY{p}{)}

\PY{k}{if} \PY{n}{p} \PY{o}{\PYZlt{}} \PY{l+m+mf}{0.05}\PY{p}{:}
    \PY{n}{test\PYZus{}result} \PY{o}{=} \PY{l+s+s2}{\PYZdq{}}\PY{l+s+s2}{reject H0.}\PY{l+s+s2}{\PYZdq{}}
\PY{k}{else}\PY{p}{:}
    \PY{n}{test\PYZus{}result} \PY{o}{=} \PY{l+s+s2}{\PYZdq{}}\PY{l+s+s2}{insufficient evidence to reject H0.}\PY{l+s+s2}{\PYZdq{}}

\PY{n+nb}{print}\PY{p}{(}\PY{l+s+sa}{f}\PY{l+s+s2}{\PYZdq{}}\PY{l+s+s2}{Mann\PYZhy{}Whitney statistic = }\PY{l+s+s2}{\PYZob{}}\PY{l+s+s2}{np.round(U,4)\PYZcb{}, p\PYZhy{}value = }\PY{l+s+s2}{\PYZob{}}\PY{l+s+s2}{np.round(p,4)\PYZcb{}}\PY{l+s+s2}{\PYZdq{}} \PY{o}{+} \PY{l+s+s2}{\PYZdq{}}\PY{l+s+s2}{, }\PY{l+s+s2}{\PYZdq{}} \PY{o}{+} \PY{n}{test\PYZus{}result}\PY{p}{)}
\end{Verbatim}
\end{tcolorbox}

    \begin{Verbatim}[commandchars=\\\{\}]
Mann-Whitney statistic = 108.0, p-value = 0.1194, insufficient evidence to
reject H0.
    \end{Verbatim}

    \emph{Conclusion: For both cases, there is insufficient evidence to
conclude that there is an increase in rainfall due to cloud-seeding.}

    \hypertarget{rice-13.24---performance-in-sporting-contest}{%
\section{Rice 13.24 - Performance in Sporting
Contest}\label{rice-13.24---performance-in-sporting-contest}}

\emph{Is it advantageous to wear the color red in a sporting contest?}
According to Hill and Barton (2005): \textgreater{} Although other
colours are also present in animal displays, it is specifically the
presence and intensity of red coloration that correlates with male
dominance and testosterone levels. In humans, anger is associated with a
reddening of the skin due to increased blood flow, whereas fear is
associated with increased pallor in similarly threatening situations.
Hence, increased redness during aggressive interactions may reflect
relative dominance. Because artificial stimuli can exploit innate
responses to natural stimuli, we tested whether wearing red might
influence the outcome of physical contests in humans\ldots{}

In the 2004 Olympic Games, contestants in four combat sports (boxing,
tae kwon do, Greco-Roman wrestling, and freestyle wrestling) were
randomly assigned red or blue outfits (or body protectors). If colour
has no effect on the outcome of contests, the number of winners wearing
red should be statistically indistinguishable from the number of winners
wearing blue. They thus tabulated the colors worn by the winners in
these contests: 

    \begin{tcolorbox}[breakable, size=fbox, boxrule=1pt, pad at break*=1mm,colback=cellbackground, colframe=cellborder]
\prompt{In}{incolor}{124}{\boxspacing}
\begin{Verbatim}[commandchars=\\\{\}]
\PY{n}{result} \PY{o}{=} \PY{n}{pd}\PY{o}{.}\PY{n}{DataFrame}\PY{p}{(}\PY{p}{\PYZob{}}
    \PY{l+s+s2}{\PYZdq{}}\PY{l+s+s2}{Sport}\PY{l+s+s2}{\PYZdq{}}\PY{p}{:} \PY{p}{[}\PY{l+s+s2}{\PYZdq{}}\PY{l+s+s2}{Boxing}\PY{l+s+s2}{\PYZdq{}}\PY{p}{,} \PY{l+s+s2}{\PYZdq{}}\PY{l+s+s2}{Freestyle Wrestling}\PY{l+s+s2}{\PYZdq{}}\PY{p}{,} \PY{l+s+s2}{\PYZdq{}}\PY{l+s+s2}{Greco Roman Wrestling}\PY{l+s+s2}{\PYZdq{}}\PY{p}{,} \PY{l+s+s2}{\PYZdq{}}\PY{l+s+s2}{Tae Kwon Do}\PY{l+s+s2}{\PYZdq{}}\PY{p}{]}\PY{p}{,}
    \PY{l+s+s2}{\PYZdq{}}\PY{l+s+s2}{Red}\PY{l+s+s2}{\PYZdq{}}\PY{p}{:} \PY{p}{[}\PY{l+m+mi}{148}\PY{p}{,} \PY{l+m+mi}{27}\PY{p}{,} \PY{l+m+mi}{25}\PY{p}{,} \PY{l+m+mi}{45}\PY{p}{]}\PY{p}{,}
    \PY{l+s+s2}{\PYZdq{}}\PY{l+s+s2}{Blue}\PY{l+s+s2}{\PYZdq{}}\PY{p}{:} \PY{p}{[}\PY{l+m+mi}{120}\PY{p}{,} \PY{l+m+mi}{24}\PY{p}{,} \PY{l+m+mi}{23}\PY{p}{,} \PY{l+m+mi}{35}\PY{p}{]}
\PY{p}{\PYZcb{}}\PY{p}{)}
\PY{n}{result}
\end{Verbatim}
\end{tcolorbox}

            \begin{tcolorbox}[breakable, size=fbox, boxrule=.5pt, pad at break*=1mm, opacityfill=0]
\prompt{Out}{outcolor}{124}{\boxspacing}
\begin{Verbatim}[commandchars=\\\{\}]
                   Sport  Red  Blue
0                 Boxing  148   120
1    Freestyle Wrestling   27    24
2  Greco Roman Wrestling   25    23
3            Tae Kwon Do   45    35
\end{Verbatim}
\end{tcolorbox}
        
    \begin{tcolorbox}[breakable, size=fbox, boxrule=1pt, pad at break*=1mm,colback=cellbackground, colframe=cellborder]
\prompt{In}{incolor}{125}{\boxspacing}
\begin{Verbatim}[commandchars=\\\{\}]
\PY{n}{row\PYZus{}total} \PY{o}{=} \PY{n}{result}\PY{o}{.}\PY{n}{Red} \PY{o}{+} \PY{n}{result}\PY{o}{.}\PY{n}{Blue}
\PY{n}{row\PYZus{}total}
\end{Verbatim}
\end{tcolorbox}

            \begin{tcolorbox}[breakable, size=fbox, boxrule=.5pt, pad at break*=1mm, opacityfill=0]
\prompt{Out}{outcolor}{125}{\boxspacing}
\begin{Verbatim}[commandchars=\\\{\}]
0    268
1     51
2     48
3     80
dtype: int64
\end{Verbatim}
\end{tcolorbox}
        
    \begin{tcolorbox}[breakable, size=fbox, boxrule=1pt, pad at break*=1mm,colback=cellbackground, colframe=cellborder]
\prompt{In}{incolor}{126}{\boxspacing}
\begin{Verbatim}[commandchars=\\\{\}]
\PY{n}{col\PYZus{}total} \PY{o}{=} \PY{n}{result}\PY{o}{.}\PY{n}{sum}\PY{p}{(}\PY{p}{)}\PY{o}{.}\PY{n}{iloc}\PY{p}{[}\PY{l+m+mi}{1}\PY{p}{:}\PY{l+m+mi}{3}\PY{p}{]}
\PY{n}{col\PYZus{}total}
\end{Verbatim}
\end{tcolorbox}

            \begin{tcolorbox}[breakable, size=fbox, boxrule=.5pt, pad at break*=1mm, opacityfill=0]
\prompt{Out}{outcolor}{126}{\boxspacing}
\begin{Verbatim}[commandchars=\\\{\}]
Red     245
Blue    202
dtype: object
\end{Verbatim}
\end{tcolorbox}
        
    \begin{tcolorbox}[breakable, size=fbox, boxrule=1pt, pad at break*=1mm,colback=cellbackground, colframe=cellborder]
\prompt{In}{incolor}{127}{\boxspacing}
\begin{Verbatim}[commandchars=\\\{\}]
\PY{n}{total} \PY{o}{=} \PY{n}{result}\PY{o}{.}\PY{n}{sum}\PY{p}{(}\PY{p}{)}\PY{o}{.}\PY{n}{iloc}\PY{p}{[}\PY{l+m+mi}{1}\PY{p}{:}\PY{l+m+mi}{3}\PY{p}{]}\PY{o}{.}\PY{n}{sum}\PY{p}{(}\PY{p}{)}
\PY{n}{total}
\end{Verbatim}
\end{tcolorbox}

            \begin{tcolorbox}[breakable, size=fbox, boxrule=.5pt, pad at break*=1mm, opacityfill=0]
\prompt{Out}{outcolor}{127}{\boxspacing}
\begin{Verbatim}[commandchars=\\\{\}]
447
\end{Verbatim}
\end{tcolorbox}
        
    Some supplementary information is given in the file
\texttt{red-blue.txt}.

    \textbf{Part (a)} Let \(\pi_R\) denote the probability that the
contestant wearing red wins. Test the null hypothesis
\(H_0: \pi_R = 1/2\) versus the alternative hypothesis \(H_1:\)
\(\pi_R\) is the same in each sport, but \(\pi_R \neq 1/2\).

    \emph{Solution: According to recitation, this is just a proportion test.
Assume \(\alpha = 0.05\). Under null hypothesis, the total number of
winners wearing red (despite type of sport) follows the distribution
\(R \sim \text{B}(447,0.5)\), which can be approximated by a normal
distribution \(\text{N}(447 \times 0.5, 447 \times 0.25)\). We can now
perform a two-sided z-test. Therefore the sample proportion
approximately follows a \(\text{N}(0.5, 0.25/447)\) distribution.}

    \begin{tcolorbox}[breakable, size=fbox, boxrule=1pt, pad at break*=1mm,colback=cellbackground, colframe=cellborder]
\prompt{In}{incolor}{169}{\boxspacing}
\begin{Verbatim}[commandchars=\\\{\}]
\PY{n}{z} \PY{o}{=} \PY{p}{(} \PY{p}{(}\PY{n}{col\PYZus{}total}\PY{o}{.}\PY{n}{Red}\PY{p}{)}\PY{o}{/}\PY{n}{total} \PY{o}{\PYZhy{}} \PY{l+m+mf}{0.5}\PY{p}{)} \PY{o}{/} \PY{n}{np}\PY{o}{.}\PY{n}{sqrt}\PY{p}{(}\PY{l+m+mi}{1}\PY{o}{/} \PY{p}{(}\PY{n}{total} \PY{o}{*} \PY{l+m+mi}{4}\PY{p}{)}\PY{p}{)}
\PY{n}{p} \PY{o}{=} \PY{l+m+mi}{1} \PY{o}{\PYZhy{}} \PY{n}{stats}\PY{o}{.}\PY{n}{norm}\PY{o}{.}\PY{n}{cdf}\PY{p}{(}\PY{n}{z}\PY{p}{)} \PY{o}{+} \PY{n}{stats}\PY{o}{.}\PY{n}{norm}\PY{o}{.}\PY{n}{cdf}\PY{p}{(}\PY{o}{\PYZhy{}}\PY{n}{z}\PY{p}{)} 

\PY{k}{if} \PY{n}{p} \PY{o}{\PYZlt{}} \PY{l+m+mf}{0.05}\PY{p}{:}
    \PY{n}{test\PYZus{}result} \PY{o}{=} \PY{l+s+s2}{\PYZdq{}}\PY{l+s+s2}{reject H0.}\PY{l+s+s2}{\PYZdq{}}
\PY{k}{else}\PY{p}{:}
    \PY{n}{test\PYZus{}result} \PY{o}{=} \PY{l+s+s2}{\PYZdq{}}\PY{l+s+s2}{insufficient evidence to reject H0.}\PY{l+s+s2}{\PYZdq{}}

\PY{n+nb}{print}\PY{p}{(}\PY{l+s+sa}{f}\PY{l+s+s2}{\PYZdq{}}\PY{l+s+s2}{z statistic = }\PY{l+s+s2}{\PYZob{}}\PY{l+s+s2}{np.round(z,4)\PYZcb{}, p\PYZhy{}value = }\PY{l+s+s2}{\PYZob{}}\PY{l+s+s2}{np.round(p,4)\PYZcb{}}\PY{l+s+s2}{\PYZdq{}} \PY{o}{+} \PY{l+s+s2}{\PYZdq{}}\PY{l+s+s2}{, }\PY{l+s+s2}{\PYZdq{}} \PY{o}{+} \PY{n}{test\PYZus{}result}\PY{p}{)}
\end{Verbatim}
\end{tcolorbox}

    \begin{Verbatim}[commandchars=\\\{\}]
z statistic = 2.0338, p-value = 0.042, reject H0.
    \end{Verbatim}

    \textbf{Part (b)} Test the null hypothesis that \(\pi_R = 1/2\) versus
the alternative hypothesis that allows \(\pi_R\) to be different in
different sports, but not equal to \(1/2\).

    \emph{Solution: We use a likelihood ratio test. According to recitation
again, we can calculate the chi2 statistic as followed:}

    \begin{tcolorbox}[breakable, size=fbox, boxrule=1pt, pad at break*=1mm,colback=cellbackground, colframe=cellborder]
\prompt{In}{incolor}{129}{\boxspacing}
\begin{Verbatim}[commandchars=\\\{\}]
\PY{n}{X2\PYZus{}arr} \PY{o}{=} \PY{l+m+mi}{4}\PY{o}{*}\PY{p}{(}\PY{n}{result}\PY{o}{.}\PY{n}{Red} \PY{o}{\PYZhy{}} \PY{n}{row\PYZus{}total}\PY{o}{/}\PY{l+m+mi}{2}\PY{p}{)}\PY{o}{*}\PY{o}{*}\PY{l+m+mi}{2} \PY{o}{/} \PY{n}{row\PYZus{}total}
\PY{n}{X2} \PY{o}{=} \PY{n}{X2\PYZus{}arr}\PY{o}{.}\PY{n}{sum}\PY{p}{(}\PY{p}{)}
\PY{n}{p} \PY{o}{=} \PY{l+m+mi}{1} \PY{o}{\PYZhy{}} \PY{n}{stats}\PY{o}{.}\PY{n}{chi2}\PY{o}{.}\PY{n}{cdf}\PY{p}{(}\PY{n}{X2}\PY{p}{,}\PY{n}{df}\PY{o}{=}\PY{l+m+mi}{4}\PY{p}{)}

\PY{k}{if} \PY{n}{p} \PY{o}{\PYZlt{}} \PY{l+m+mf}{0.05}\PY{p}{:}
    \PY{n}{test\PYZus{}result} \PY{o}{=} \PY{l+s+s2}{\PYZdq{}}\PY{l+s+s2}{reject H0.}\PY{l+s+s2}{\PYZdq{}}
\PY{k}{else}\PY{p}{:}
    \PY{n}{test\PYZus{}result} \PY{o}{=} \PY{l+s+s2}{\PYZdq{}}\PY{l+s+s2}{insufficient evidence to reject H0.}\PY{l+s+s2}{\PYZdq{}}

\PY{n+nb}{print}\PY{p}{(}\PY{l+s+sa}{f}\PY{l+s+s2}{\PYZdq{}}\PY{l+s+s2}{chi2 statistic = }\PY{l+s+s2}{\PYZob{}}\PY{l+s+s2}{np.round(X2,4)\PYZcb{}, p\PYZhy{}value = }\PY{l+s+s2}{\PYZob{}}\PY{l+s+s2}{np.round(p,4)\PYZcb{}, df = 4}\PY{l+s+s2}{\PYZdq{}} \PY{o}{+} \PY{l+s+s2}{\PYZdq{}}\PY{l+s+s2}{, }\PY{l+s+s2}{\PYZdq{}} \PY{o}{+} \PY{n}{test\PYZus{}result}\PY{p}{)}
\end{Verbatim}
\end{tcolorbox}

    \begin{Verbatim}[commandchars=\\\{\}]
chi2 statistic = 4.4352, p-value = 0.3503, df = 4, insufficient evidence to
reject H0.
    \end{Verbatim}

    \textbf{Part (c)} Are either of these hypothesis tests equivalent to
that which would test the null hypothesis \(\pi_R = 1/2\) versus the
alternative hypothesis \(\pi_R \neq 1/2\), using the data the total
numbers of wins summed over all the sports?

    \emph{Solution: This is equivalent to (a), but not (b).}

    \textbf{Part (d)} Is there any evidence that wearing red is more
favorable in some of the sports than others?

    \emph{Solution: We use \(\chi^2\) test for contingency table, which is
the command \texttt{chi2\_contingency} in the \texttt{scipy.stats}
library. Again we use \(\alpha = 0.05\).}

    \begin{tcolorbox}[breakable, size=fbox, boxrule=1pt, pad at break*=1mm,colback=cellbackground, colframe=cellborder]
\prompt{In}{incolor}{130}{\boxspacing}
\begin{Verbatim}[commandchars=\\\{\}]
\PY{n}{chi2}\PY{p}{,} \PY{n}{pval}\PY{p}{,} \PY{n}{dof}\PY{p}{,} \PY{n}{expected} \PY{o}{=} \PY{n}{stats}\PY{o}{.}\PY{n}{chi2\PYZus{}contingency}\PY{p}{(}\PY{n}{result}\PY{p}{[}\PY{p}{[}\PY{l+s+s2}{\PYZdq{}}\PY{l+s+s2}{Red}\PY{l+s+s2}{\PYZdq{}}\PY{p}{,} \PY{l+s+s2}{\PYZdq{}}\PY{l+s+s2}{Blue}\PY{l+s+s2}{\PYZdq{}}\PY{p}{]}\PY{p}{]}\PY{o}{.}\PY{n}{values}\PY{p}{)}

\PY{k}{if} \PY{n}{p} \PY{o}{\PYZlt{}} \PY{l+m+mf}{0.05}\PY{p}{:}
    \PY{n}{test\PYZus{}result} \PY{o}{=} \PY{l+s+s2}{\PYZdq{}}\PY{l+s+s2}{reject H0.}\PY{l+s+s2}{\PYZdq{}}
\PY{k}{else}\PY{p}{:}
    \PY{n}{test\PYZus{}result} \PY{o}{=} \PY{l+s+s2}{\PYZdq{}}\PY{l+s+s2}{insufficient evidence to reject H0.}\PY{l+s+s2}{\PYZdq{}}

\PY{n+nb}{print}\PY{p}{(}\PY{l+s+sa}{f}\PY{l+s+s2}{\PYZdq{}}\PY{l+s+s2}{chi2 statistic = }\PY{l+s+s2}{\PYZob{}}\PY{l+s+s2}{np.round(chi2,4)\PYZcb{}, p\PYZhy{}value = }\PY{l+s+s2}{\PYZob{}}\PY{l+s+s2}{np.round(pval,4)\PYZcb{}, df = }\PY{l+s+si}{\PYZob{}dof\PYZcb{}}\PY{l+s+s2}{\PYZdq{}} \PY{o}{+} \PY{l+s+s2}{\PYZdq{}}\PY{l+s+s2}{, }\PY{l+s+s2}{\PYZdq{}} \PY{o}{+} \PY{n}{test\PYZus{}result}\PY{p}{)}
\end{Verbatim}
\end{tcolorbox}

    \begin{Verbatim}[commandchars=\\\{\}]
chi2 statistic = 0.3015, p-value = 0.9597, df = 3, insufficient evidence to
reject H0.
    \end{Verbatim}

    \emph{As a reference, here the expected frequencies are tabulated.}

    \begin{tcolorbox}[breakable, size=fbox, boxrule=1pt, pad at break*=1mm,colback=cellbackground, colframe=cellborder]
\prompt{In}{incolor}{131}{\boxspacing}
\begin{Verbatim}[commandchars=\\\{\}]
\PY{n}{expected}
\end{Verbatim}
\end{tcolorbox}

            \begin{tcolorbox}[breakable, size=fbox, boxrule=.5pt, pad at break*=1mm, opacityfill=0]
\prompt{Out}{outcolor}{131}{\boxspacing}
\begin{Verbatim}[commandchars=\\\{\}]
array([[146.89038031, 121.10961969],
       [ 27.95302013,  23.04697987],
       [ 26.30872483,  21.69127517],
       [ 43.84787472,  36.15212528]])
\end{Verbatim}
\end{tcolorbox}
        
    \textbf{Part (e)} From an analysis of the points scored by winners and
losers, Hill and Barton concluded that color had the greatest effect in
close contests. Data on the points of each match are contained in the
file \texttt{red-blue.xls}. Analyze this data and see whether you agree
with their conclusion.

    \emph{Solution: First import the data.}

    \begin{tcolorbox}[breakable, size=fbox, boxrule=1pt, pad at break*=1mm,colback=cellbackground, colframe=cellborder]
\prompt{In}{incolor}{132}{\boxspacing}
\begin{Verbatim}[commandchars=\\\{\}]
\PY{n}{boxing} \PY{o}{=} \PY{n}{pd}\PY{o}{.}\PY{n}{read\PYZus{}csv}\PY{p}{(}\PY{l+s+s2}{\PYZdq{}}\PY{l+s+s2}{red\PYZhy{}blue\PYZus{}boxing.txt}\PY{l+s+s2}{\PYZdq{}}\PY{p}{)}
\PY{n}{freestyle} \PY{o}{=} \PY{n}{pd}\PY{o}{.}\PY{n}{read\PYZus{}csv}\PY{p}{(}\PY{l+s+s2}{\PYZdq{}}\PY{l+s+s2}{red\PYZhy{}blue\PYZus{}FW.txt}\PY{l+s+s2}{\PYZdq{}}\PY{p}{)}
\PY{n}{grwrestling} \PY{o}{=} \PY{n}{pd}\PY{o}{.}\PY{n}{read\PYZus{}csv}\PY{p}{(}\PY{l+s+s2}{\PYZdq{}}\PY{l+s+s2}{red\PYZhy{}blue\PYZus{}GR.txt}\PY{l+s+s2}{\PYZdq{}}\PY{p}{)}
\PY{n}{taikwondo} \PY{o}{=} \PY{n}{pd}\PY{o}{.}\PY{n}{read\PYZus{}csv}\PY{p}{(}\PY{l+s+s2}{\PYZdq{}}\PY{l+s+s2}{red\PYZhy{}blue\PYZus{}TKD.txt}\PY{l+s+s2}{\PYZdq{}}\PY{p}{)}
\end{Verbatim}
\end{tcolorbox}

    \emph{We specifically look at the data when
\texttt{Method\ of\ Win\ =\ "Points"} as these rounds represent ``close
contests''. We therefore compute the difference of points (\texttt{Red}
minus \texttt{Blue}) then filter out the \texttt{NaN}'s entries. For
convenience we write a function that do our job.}

    \begin{tcolorbox}[breakable, size=fbox, boxrule=1pt, pad at break*=1mm,colback=cellbackground, colframe=cellborder]
\prompt{In}{incolor}{162}{\boxspacing}
\begin{Verbatim}[commandchars=\\\{\}]
\PY{k}{def} \PY{n+nf}{wilcoxon\PYZus{}sport}\PY{p}{(}\PY{n}{data}\PY{p}{)}\PY{p}{:}
    \PY{n}{diff} \PY{o}{=} \PY{p}{(}\PY{n}{data}\PY{p}{[}\PY{l+s+s2}{\PYZdq{}}\PY{l+s+s2}{Points Scored by Red}\PY{l+s+s2}{\PYZdq{}}\PY{p}{]} \PY{o}{\PYZhy{}} \PY{n}{data}\PY{p}{[}\PY{l+s+s2}{\PYZdq{}}\PY{l+s+s2}{Points Scored by Blue}\PY{l+s+s2}{\PYZdq{}}\PY{p}{]}\PY{p}{)}\PY{o}{.}\PY{n}{dropna}\PY{p}{(}\PY{p}{)}
    \PY{c+c1}{\PYZsh{} diff\PYZus{}filtered = diff[diff \PYZlt{}= diff.quantile(.25)]}
    
    \PY{n}{w}\PY{p}{,} \PY{n}{p} \PY{o}{=} \PY{n}{stats}\PY{o}{.}\PY{n}{wilcoxon}\PY{p}{(}\PY{n}{diff}\PY{p}{,} \PY{n}{alternative}\PY{o}{=}\PY{l+s+s2}{\PYZdq{}}\PY{l+s+s2}{greater}\PY{l+s+s2}{\PYZdq{}}\PY{p}{)}
    \PY{c+c1}{\PYZsh{} w, p = stats.wilcoxon(diff\PYZus{}filtered, alternative=\PYZdq{}greater\PYZdq{})}
    
    \PY{k}{if} \PY{n}{p} \PY{o}{\PYZlt{}} \PY{l+m+mf}{0.05}\PY{p}{:}
        \PY{n}{test\PYZus{}result} \PY{o}{=} \PY{l+s+s2}{\PYZdq{}}\PY{l+s+s2}{reject H0.}\PY{l+s+s2}{\PYZdq{}}
    \PY{k}{else}\PY{p}{:}
        \PY{n}{test\PYZus{}result} \PY{o}{=} \PY{l+s+s2}{\PYZdq{}}\PY{l+s+s2}{insufficient evidence to reject H0.}\PY{l+s+s2}{\PYZdq{}}
    
    \PY{n+nb}{print}\PY{p}{(}\PY{l+s+sa}{f}\PY{l+s+s2}{\PYZdq{}}\PY{l+s+s2}{Wilcoxon statistic = }\PY{l+s+s2}{\PYZob{}}\PY{l+s+s2}{np.round(w,4)\PYZcb{}, p\PYZhy{}value = }\PY{l+s+s2}{\PYZob{}}\PY{l+s+s2}{np.round(p,4)\PYZcb{}}\PY{l+s+s2}{\PYZdq{}} \PY{o}{+} \PY{l+s+s2}{\PYZdq{}}\PY{l+s+s2}{, }\PY{l+s+s2}{\PYZdq{}} \PY{o}{+} \PY{n}{test\PYZus{}result}\PY{p}{)}
\end{Verbatim}
\end{tcolorbox}

    \begin{tcolorbox}[breakable, size=fbox, boxrule=1pt, pad at break*=1mm,colback=cellbackground, colframe=cellborder]
\prompt{In}{incolor}{163}{\boxspacing}
\begin{Verbatim}[commandchars=\\\{\}]
\PY{n}{wilcoxon\PYZus{}sport}\PY{p}{(}\PY{n}{boxing}\PY{p}{)}
\end{Verbatim}
\end{tcolorbox}

    \begin{Verbatim}[commandchars=\\\{\}]
Wilcoxon statistic = 13311.0, p-value = 0.3979, insufficient evidence to reject
H0.
    \end{Verbatim}

    \begin{tcolorbox}[breakable, size=fbox, boxrule=1pt, pad at break*=1mm,colback=cellbackground, colframe=cellborder]
\prompt{In}{incolor}{164}{\boxspacing}
\begin{Verbatim}[commandchars=\\\{\}]
\PY{n}{wilcoxon\PYZus{}sport}\PY{p}{(}\PY{n}{freestyle}\PY{p}{)}
\end{Verbatim}
\end{tcolorbox}

    \begin{Verbatim}[commandchars=\\\{\}]
Wilcoxon statistic = 835.5, p-value = 0.2105, insufficient evidence to reject
H0.
    \end{Verbatim}

    \begin{tcolorbox}[breakable, size=fbox, boxrule=1pt, pad at break*=1mm,colback=cellbackground, colframe=cellborder]
\prompt{In}{incolor}{165}{\boxspacing}
\begin{Verbatim}[commandchars=\\\{\}]
\PY{n}{wilcoxon\PYZus{}sport}\PY{p}{(}\PY{n}{grwrestling}\PY{p}{)}
\end{Verbatim}
\end{tcolorbox}

    \begin{Verbatim}[commandchars=\\\{\}]
Wilcoxon statistic = 580.5, p-value = 0.781, insufficient evidence to reject H0.
    \end{Verbatim}

    \begin{tcolorbox}[breakable, size=fbox, boxrule=1pt, pad at break*=1mm,colback=cellbackground, colframe=cellborder]
\prompt{In}{incolor}{166}{\boxspacing}
\begin{Verbatim}[commandchars=\\\{\}]
\PY{n}{wilcoxon\PYZus{}sport}\PY{p}{(}\PY{n}{taikwondo}\PY{p}{)}
\end{Verbatim}
\end{tcolorbox}

    \begin{Verbatim}[commandchars=\\\{\}]
Wilcoxon statistic = 1057.0, p-value = 0.4546, insufficient evidence to reject
H0.
    \end{Verbatim}

    \emph{For all sports, there is insufficient evidence to conclude that
color has effect on the difference of points. Seems like the data does
not agree with the conclusion.}

\textbf{Remark:} 1. As pointed out by other classmates, this is not a
good way to access whether color has \emph{greatest} effect in close
contest, because there are no other effects to be compared with. Perhaps
we can access whether color has \emph{significant} effect instead. In
fact, according to \texttt{red-blue.txt}:

\begin{quote}
For each sport, every contest was categorized on the basis of the
quartile of the final points difference in the bout, where the first
quartile of points difference represents symmetrical contests between
competitors of similar ability and the fourth quartile represents
contests between competitors with large asymmetries in ability. Contests
stopped early (such as by knockout) were scored as highly asymmetric
contests and coded in the fourth quartile.
\end{quote}

There is no clear indication of quartiles. (The followed cell listed all
labels in the datasets.) It is unclear whether we actually have to
define the quantiles ourselves, so we adopt the simplest approach of
dropping all \texttt{NaN} entries and do a Wilcoxon rank sum test (we
did try to isolate the first quantile but has no progress). Appreciate
your further guidance.

    \begin{tcolorbox}[breakable, size=fbox, boxrule=1pt, pad at break*=1mm,colback=cellbackground, colframe=cellborder]
\prompt{In}{incolor}{145}{\boxspacing}
\begin{Verbatim}[commandchars=\\\{\}]
\PY{n+nb}{set}\PY{p}{(}\PY{n}{boxing}\PY{p}{[}\PY{l+s+s2}{\PYZdq{}}\PY{l+s+s2}{Method of Win}\PY{l+s+s2}{\PYZdq{}}\PY{p}{]}\PY{o}{.}\PY{n}{values}\PY{p}{)}
\end{Verbatim}
\end{tcolorbox}

            \begin{tcolorbox}[breakable, size=fbox, boxrule=.5pt, pad at break*=1mm, opacityfill=0]
\prompt{Out}{outcolor}{145}{\boxspacing}
\begin{Verbatim}[commandchars=\\\{\}]
\{'Disqualified',
 'Points',
 'Referee Stopped Contest',
 'Referee Stopped Contest - Outscored',
 'Walk Over'\}
\end{Verbatim}
\end{tcolorbox}
        
    \begin{enumerate}
\def\labelenumi{\arabic{enumi}.}
\setcounter{enumi}{1}
\tightlist
\item
  It is a little bit late to ask this, but it is not clear whether we
  need to manually code up all the testing algorithms instead of using
  built-in algorithms. Please clarify in lectures/recitations.
\end{enumerate}


    % Add a bibliography block to the postdoc
    
    
    
\end{document}
