\documentclass[11pt]{article}

    \usepackage[breakable]{tcolorbox}
    \usepackage{parskip} % Stop auto-indenting (to mimic markdown behaviour)
    \renewcommand{\familydefault}{\sfdefault}
    \usepackage{helvet}

    % Basic figure setup, for now with no caption control since it's done
    % automatically by Pandoc (which extracts ![](path) syntax from Markdown).
    \usepackage{graphicx}
    % Maintain compatibility with old templates. Remove in nbconvert 6.0
    \let\Oldincludegraphics\includegraphics
    % Ensure that by default, figures have no caption (until we provide a
    % proper Figure object with a Caption API and a way to capture that
    % in the conversion process - todo).
    \usepackage{caption}
    \DeclareCaptionFormat{nocaption}{}
    \captionsetup{format=nocaption,aboveskip=0pt,belowskip=0pt}

    \usepackage[Export]{adjustbox} % Used to constrain images to a maximum size
    \adjustboxset{max size={0.9\linewidth}{0.9\paperheight}}
    \usepackage{float}
    \floatplacement{figure}{H} % forces figures to be placed at the correct location
    \usepackage{xcolor} % Allow colors to be defined
    \usepackage{enumerate} % Needed for markdown enumerations to work
    \usepackage{geometry} % Used to adjust the document margins
    \usepackage{amsmath} % Equations
    \usepackage{amssymb} % Equations
    \usepackage{textcomp} % defines textquotesingle
    % Hack from http://tex.stackexchange.com/a/47451/13684:
    \AtBeginDocument{%
        \def\PYZsq{\textquotesingle}% Upright quotes in Pygmentized code
    }
    \usepackage{upquote} % Upright quotes for verbatim code
    \usepackage{eurosym} % defines \euro
    \usepackage[mathletters]{ucs} % Extended unicode (utf-8) support
    \usepackage{fancyvrb} % verbatim replacement that allows latex
    \usepackage{grffile} % extends the file name processing of package graphics 
                         % to support a larger range
    \makeatletter % fix for grffile with XeLaTeX
    \def\Gread@@xetex#1{%
      \IfFileExists{"\Gin@base".bb}%
      {\Gread@eps{\Gin@base.bb}}%
      {\Gread@@xetex@aux#1}%
    }
    \makeatother

    % The hyperref package gives us a pdf with properly built
    % internal navigation ('pdf bookmarks' for the table of contents,
    % internal cross-reference links, web links for URLs, etc.)
    \usepackage{hyperref}
    % The default LaTeX title has an obnoxious amount of whitespace. By default,
    % titling removes some of it. It also provides customization options.
    \usepackage{titling}
    \usepackage{longtable} % longtable support required by pandoc >1.10
    \usepackage{booktabs}  % table support for pandoc > 1.12.2
    \usepackage[inline]{enumitem} % IRkernel/repr support (it uses the enumerate* environment)
    \usepackage[normalem]{ulem} % ulem is needed to support strikethroughs (\sout)
                                % normalem makes italics be italics, not underlines
    \usepackage{mathrsfs}
    

    
    % Colors for the hyperref package
    \definecolor{urlcolor}{rgb}{0,.145,.698}
    \definecolor{linkcolor}{rgb}{.71,0.21,0.01}
    \definecolor{citecolor}{rgb}{.12,.54,.11}

    % ANSI colors
    \definecolor{ansi-black}{HTML}{3E424D}
    \definecolor{ansi-black-intense}{HTML}{282C36}
    \definecolor{ansi-red}{HTML}{E75C58}
    \definecolor{ansi-red-intense}{HTML}{B22B31}
    \definecolor{ansi-green}{HTML}{00A250}
    \definecolor{ansi-green-intense}{HTML}{007427}
    \definecolor{ansi-yellow}{HTML}{DDB62B}
    \definecolor{ansi-yellow-intense}{HTML}{B27D12}
    \definecolor{ansi-blue}{HTML}{208FFB}
    \definecolor{ansi-blue-intense}{HTML}{0065CA}
    \definecolor{ansi-magenta}{HTML}{D160C4}
    \definecolor{ansi-magenta-intense}{HTML}{A03196}
    \definecolor{ansi-cyan}{HTML}{60C6C8}
    \definecolor{ansi-cyan-intense}{HTML}{258F8F}
    \definecolor{ansi-white}{HTML}{C5C1B4}
    \definecolor{ansi-white-intense}{HTML}{A1A6B2}
    \definecolor{ansi-default-inverse-fg}{HTML}{FFFFFF}
    \definecolor{ansi-default-inverse-bg}{HTML}{000000}

    % commands and environments needed by pandoc snippets
    % extracted from the output of `pandoc -s`
    \providecommand{\tightlist}{%
      \setlength{\itemsep}{0pt}\setlength{\parskip}{0pt}}
    \DefineVerbatimEnvironment{Highlighting}{Verbatim}{commandchars=\\\{\}}
    % Add ',fontsize=\small' for more characters per line
    \newenvironment{Shaded}{}{}
    \newcommand{\KeywordTok}[1]{\textcolor[rgb]{0.00,0.44,0.13}{\textbf{{#1}}}}
    \newcommand{\DataTypeTok}[1]{\textcolor[rgb]{0.56,0.13,0.00}{{#1}}}
    \newcommand{\DecValTok}[1]{\textcolor[rgb]{0.25,0.63,0.44}{{#1}}}
    \newcommand{\BaseNTok}[1]{\textcolor[rgb]{0.25,0.63,0.44}{{#1}}}
    \newcommand{\FloatTok}[1]{\textcolor[rgb]{0.25,0.63,0.44}{{#1}}}
    \newcommand{\CharTok}[1]{\textcolor[rgb]{0.25,0.44,0.63}{{#1}}}
    \newcommand{\StringTok}[1]{\textcolor[rgb]{0.25,0.44,0.63}{{#1}}}
    \newcommand{\CommentTok}[1]{\textcolor[rgb]{0.38,0.63,0.69}{\textit{{#1}}}}
    \newcommand{\OtherTok}[1]{\textcolor[rgb]{0.00,0.44,0.13}{{#1}}}
    \newcommand{\AlertTok}[1]{\textcolor[rgb]{1.00,0.00,0.00}{\textbf{{#1}}}}
    \newcommand{\FunctionTok}[1]{\textcolor[rgb]{0.02,0.16,0.49}{{#1}}}
    \newcommand{\RegionMarkerTok}[1]{{#1}}
    \newcommand{\ErrorTok}[1]{\textcolor[rgb]{1.00,0.00,0.00}{\textbf{{#1}}}}
    \newcommand{\NormalTok}[1]{{#1}}
    
    % Additional commands for more recent versions of Pandoc
    \newcommand{\ConstantTok}[1]{\textcolor[rgb]{0.53,0.00,0.00}{{#1}}}
    \newcommand{\SpecialCharTok}[1]{\textcolor[rgb]{0.25,0.44,0.63}{{#1}}}
    \newcommand{\VerbatimStringTok}[1]{\textcolor[rgb]{0.25,0.44,0.63}{{#1}}}
    \newcommand{\SpecialStringTok}[1]{\textcolor[rgb]{0.73,0.40,0.53}{{#1}}}
    \newcommand{\ImportTok}[1]{{#1}}
    \newcommand{\DocumentationTok}[1]{\textcolor[rgb]{0.73,0.13,0.13}{\textit{{#1}}}}
    \newcommand{\AnnotationTok}[1]{\textcolor[rgb]{0.38,0.63,0.69}{\textbf{\textit{{#1}}}}}
    \newcommand{\CommentVarTok}[1]{\textcolor[rgb]{0.38,0.63,0.69}{\textbf{\textit{{#1}}}}}
    \newcommand{\VariableTok}[1]{\textcolor[rgb]{0.10,0.09,0.49}{{#1}}}
    \newcommand{\ControlFlowTok}[1]{\textcolor[rgb]{0.00,0.44,0.13}{\textbf{{#1}}}}
    \newcommand{\OperatorTok}[1]{\textcolor[rgb]{0.40,0.40,0.40}{{#1}}}
    \newcommand{\BuiltInTok}[1]{{#1}}
    \newcommand{\ExtensionTok}[1]{{#1}}
    \newcommand{\PreprocessorTok}[1]{\textcolor[rgb]{0.74,0.48,0.00}{{#1}}}
    \newcommand{\AttributeTok}[1]{\textcolor[rgb]{0.49,0.56,0.16}{{#1}}}
    \newcommand{\InformationTok}[1]{\textcolor[rgb]{0.38,0.63,0.69}{\textbf{\textit{{#1}}}}}
    \newcommand{\WarningTok}[1]{\textcolor[rgb]{0.38,0.63,0.69}{\textbf{\textit{{#1}}}}}
    
    
    % Define a nice break command that doesn't care if a line doesn't already
    % exist.
    \def\br{\hspace*{\fill} \\* }
    % Math Jax compatibility definitions
    \def\gt{>}
    \def\lt{<}
    \let\Oldtex\TeX
    \let\Oldlatex\LaTeX
    \renewcommand{\TeX}{\textrm{\Oldtex}}
    \renewcommand{\LaTeX}{\textrm{\Oldlatex}}
    % Document parameters
    
    
    
    
    
% Pygments definitions
\makeatletter
\def\PY@reset{\let\PY@it=\relax \let\PY@bf=\relax%
    \let\PY@ul=\relax \let\PY@tc=\relax%
    \let\PY@bc=\relax \let\PY@ff=\relax}
\def\PY@tok#1{\csname PY@tok@#1\endcsname}
\def\PY@toks#1+{\ifx\relax#1\empty\else%
    \PY@tok{#1}\expandafter\PY@toks\fi}
\def\PY@do#1{\PY@bc{\PY@tc{\PY@ul{%
    \PY@it{\PY@bf{\PY@ff{#1}}}}}}}
\def\PY#1#2{\PY@reset\PY@toks#1+\relax+\PY@do{#2}}

\expandafter\def\csname PY@tok@w\endcsname{\def\PY@tc##1{\textcolor[rgb]{0.73,0.73,0.73}{##1}}}
\expandafter\def\csname PY@tok@c\endcsname{\let\PY@it=\textit\def\PY@tc##1{\textcolor[rgb]{0.25,0.50,0.50}{##1}}}
\expandafter\def\csname PY@tok@cp\endcsname{\def\PY@tc##1{\textcolor[rgb]{0.74,0.48,0.00}{##1}}}
\expandafter\def\csname PY@tok@k\endcsname{\let\PY@bf=\textbf\def\PY@tc##1{\textcolor[rgb]{0.00,0.50,0.00}{##1}}}
\expandafter\def\csname PY@tok@kp\endcsname{\def\PY@tc##1{\textcolor[rgb]{0.00,0.50,0.00}{##1}}}
\expandafter\def\csname PY@tok@kt\endcsname{\def\PY@tc##1{\textcolor[rgb]{0.69,0.00,0.25}{##1}}}
\expandafter\def\csname PY@tok@o\endcsname{\def\PY@tc##1{\textcolor[rgb]{0.40,0.40,0.40}{##1}}}
\expandafter\def\csname PY@tok@ow\endcsname{\let\PY@bf=\textbf\def\PY@tc##1{\textcolor[rgb]{0.67,0.13,1.00}{##1}}}
\expandafter\def\csname PY@tok@nb\endcsname{\def\PY@tc##1{\textcolor[rgb]{0.00,0.50,0.00}{##1}}}
\expandafter\def\csname PY@tok@nf\endcsname{\def\PY@tc##1{\textcolor[rgb]{0.00,0.00,1.00}{##1}}}
\expandafter\def\csname PY@tok@nc\endcsname{\let\PY@bf=\textbf\def\PY@tc##1{\textcolor[rgb]{0.00,0.00,1.00}{##1}}}
\expandafter\def\csname PY@tok@nn\endcsname{\let\PY@bf=\textbf\def\PY@tc##1{\textcolor[rgb]{0.00,0.00,1.00}{##1}}}
\expandafter\def\csname PY@tok@ne\endcsname{\let\PY@bf=\textbf\def\PY@tc##1{\textcolor[rgb]{0.82,0.25,0.23}{##1}}}
\expandafter\def\csname PY@tok@nv\endcsname{\def\PY@tc##1{\textcolor[rgb]{0.10,0.09,0.49}{##1}}}
\expandafter\def\csname PY@tok@no\endcsname{\def\PY@tc##1{\textcolor[rgb]{0.53,0.00,0.00}{##1}}}
\expandafter\def\csname PY@tok@nl\endcsname{\def\PY@tc##1{\textcolor[rgb]{0.63,0.63,0.00}{##1}}}
\expandafter\def\csname PY@tok@ni\endcsname{\let\PY@bf=\textbf\def\PY@tc##1{\textcolor[rgb]{0.60,0.60,0.60}{##1}}}
\expandafter\def\csname PY@tok@na\endcsname{\def\PY@tc##1{\textcolor[rgb]{0.49,0.56,0.16}{##1}}}
\expandafter\def\csname PY@tok@nt\endcsname{\let\PY@bf=\textbf\def\PY@tc##1{\textcolor[rgb]{0.00,0.50,0.00}{##1}}}
\expandafter\def\csname PY@tok@nd\endcsname{\def\PY@tc##1{\textcolor[rgb]{0.67,0.13,1.00}{##1}}}
\expandafter\def\csname PY@tok@s\endcsname{\def\PY@tc##1{\textcolor[rgb]{0.73,0.13,0.13}{##1}}}
\expandafter\def\csname PY@tok@sd\endcsname{\let\PY@it=\textit\def\PY@tc##1{\textcolor[rgb]{0.73,0.13,0.13}{##1}}}
\expandafter\def\csname PY@tok@si\endcsname{\let\PY@bf=\textbf\def\PY@tc##1{\textcolor[rgb]{0.73,0.40,0.53}{##1}}}
\expandafter\def\csname PY@tok@se\endcsname{\let\PY@bf=\textbf\def\PY@tc##1{\textcolor[rgb]{0.73,0.40,0.13}{##1}}}
\expandafter\def\csname PY@tok@sr\endcsname{\def\PY@tc##1{\textcolor[rgb]{0.73,0.40,0.53}{##1}}}
\expandafter\def\csname PY@tok@ss\endcsname{\def\PY@tc##1{\textcolor[rgb]{0.10,0.09,0.49}{##1}}}
\expandafter\def\csname PY@tok@sx\endcsname{\def\PY@tc##1{\textcolor[rgb]{0.00,0.50,0.00}{##1}}}
\expandafter\def\csname PY@tok@m\endcsname{\def\PY@tc##1{\textcolor[rgb]{0.40,0.40,0.40}{##1}}}
\expandafter\def\csname PY@tok@gh\endcsname{\let\PY@bf=\textbf\def\PY@tc##1{\textcolor[rgb]{0.00,0.00,0.50}{##1}}}
\expandafter\def\csname PY@tok@gu\endcsname{\let\PY@bf=\textbf\def\PY@tc##1{\textcolor[rgb]{0.50,0.00,0.50}{##1}}}
\expandafter\def\csname PY@tok@gd\endcsname{\def\PY@tc##1{\textcolor[rgb]{0.63,0.00,0.00}{##1}}}
\expandafter\def\csname PY@tok@gi\endcsname{\def\PY@tc##1{\textcolor[rgb]{0.00,0.63,0.00}{##1}}}
\expandafter\def\csname PY@tok@gr\endcsname{\def\PY@tc##1{\textcolor[rgb]{1.00,0.00,0.00}{##1}}}
\expandafter\def\csname PY@tok@ge\endcsname{\let\PY@it=\textit}
\expandafter\def\csname PY@tok@gs\endcsname{\let\PY@bf=\textbf}
\expandafter\def\csname PY@tok@gp\endcsname{\let\PY@bf=\textbf\def\PY@tc##1{\textcolor[rgb]{0.00,0.00,0.50}{##1}}}
\expandafter\def\csname PY@tok@go\endcsname{\def\PY@tc##1{\textcolor[rgb]{0.53,0.53,0.53}{##1}}}
\expandafter\def\csname PY@tok@gt\endcsname{\def\PY@tc##1{\textcolor[rgb]{0.00,0.27,0.87}{##1}}}
\expandafter\def\csname PY@tok@err\endcsname{\def\PY@bc##1{\setlength{\fboxsep}{0pt}\fcolorbox[rgb]{1.00,0.00,0.00}{1,1,1}{\strut ##1}}}
\expandafter\def\csname PY@tok@kc\endcsname{\let\PY@bf=\textbf\def\PY@tc##1{\textcolor[rgb]{0.00,0.50,0.00}{##1}}}
\expandafter\def\csname PY@tok@kd\endcsname{\let\PY@bf=\textbf\def\PY@tc##1{\textcolor[rgb]{0.00,0.50,0.00}{##1}}}
\expandafter\def\csname PY@tok@kn\endcsname{\let\PY@bf=\textbf\def\PY@tc##1{\textcolor[rgb]{0.00,0.50,0.00}{##1}}}
\expandafter\def\csname PY@tok@kr\endcsname{\let\PY@bf=\textbf\def\PY@tc##1{\textcolor[rgb]{0.00,0.50,0.00}{##1}}}
\expandafter\def\csname PY@tok@bp\endcsname{\def\PY@tc##1{\textcolor[rgb]{0.00,0.50,0.00}{##1}}}
\expandafter\def\csname PY@tok@fm\endcsname{\def\PY@tc##1{\textcolor[rgb]{0.00,0.00,1.00}{##1}}}
\expandafter\def\csname PY@tok@vc\endcsname{\def\PY@tc##1{\textcolor[rgb]{0.10,0.09,0.49}{##1}}}
\expandafter\def\csname PY@tok@vg\endcsname{\def\PY@tc##1{\textcolor[rgb]{0.10,0.09,0.49}{##1}}}
\expandafter\def\csname PY@tok@vi\endcsname{\def\PY@tc##1{\textcolor[rgb]{0.10,0.09,0.49}{##1}}}
\expandafter\def\csname PY@tok@vm\endcsname{\def\PY@tc##1{\textcolor[rgb]{0.10,0.09,0.49}{##1}}}
\expandafter\def\csname PY@tok@sa\endcsname{\def\PY@tc##1{\textcolor[rgb]{0.73,0.13,0.13}{##1}}}
\expandafter\def\csname PY@tok@sb\endcsname{\def\PY@tc##1{\textcolor[rgb]{0.73,0.13,0.13}{##1}}}
\expandafter\def\csname PY@tok@sc\endcsname{\def\PY@tc##1{\textcolor[rgb]{0.73,0.13,0.13}{##1}}}
\expandafter\def\csname PY@tok@dl\endcsname{\def\PY@tc##1{\textcolor[rgb]{0.73,0.13,0.13}{##1}}}
\expandafter\def\csname PY@tok@s2\endcsname{\def\PY@tc##1{\textcolor[rgb]{0.73,0.13,0.13}{##1}}}
\expandafter\def\csname PY@tok@sh\endcsname{\def\PY@tc##1{\textcolor[rgb]{0.73,0.13,0.13}{##1}}}
\expandafter\def\csname PY@tok@s1\endcsname{\def\PY@tc##1{\textcolor[rgb]{0.73,0.13,0.13}{##1}}}
\expandafter\def\csname PY@tok@mb\endcsname{\def\PY@tc##1{\textcolor[rgb]{0.40,0.40,0.40}{##1}}}
\expandafter\def\csname PY@tok@mf\endcsname{\def\PY@tc##1{\textcolor[rgb]{0.40,0.40,0.40}{##1}}}
\expandafter\def\csname PY@tok@mh\endcsname{\def\PY@tc##1{\textcolor[rgb]{0.40,0.40,0.40}{##1}}}
\expandafter\def\csname PY@tok@mi\endcsname{\def\PY@tc##1{\textcolor[rgb]{0.40,0.40,0.40}{##1}}}
\expandafter\def\csname PY@tok@il\endcsname{\def\PY@tc##1{\textcolor[rgb]{0.40,0.40,0.40}{##1}}}
\expandafter\def\csname PY@tok@mo\endcsname{\def\PY@tc##1{\textcolor[rgb]{0.40,0.40,0.40}{##1}}}
\expandafter\def\csname PY@tok@ch\endcsname{\let\PY@it=\textit\def\PY@tc##1{\textcolor[rgb]{0.25,0.50,0.50}{##1}}}
\expandafter\def\csname PY@tok@cm\endcsname{\let\PY@it=\textit\def\PY@tc##1{\textcolor[rgb]{0.25,0.50,0.50}{##1}}}
\expandafter\def\csname PY@tok@cpf\endcsname{\let\PY@it=\textit\def\PY@tc##1{\textcolor[rgb]{0.25,0.50,0.50}{##1}}}
\expandafter\def\csname PY@tok@c1\endcsname{\let\PY@it=\textit\def\PY@tc##1{\textcolor[rgb]{0.25,0.50,0.50}{##1}}}
\expandafter\def\csname PY@tok@cs\endcsname{\let\PY@it=\textit\def\PY@tc##1{\textcolor[rgb]{0.25,0.50,0.50}{##1}}}

\def\PYZbs{\char`\\}
\def\PYZus{\char`\_}
\def\PYZob{\char`\{}
\def\PYZcb{\char`\}}
\def\PYZca{\char`\^}
\def\PYZam{\char`\&}
\def\PYZlt{\char`\<}
\def\PYZgt{\char`\>}
\def\PYZsh{\char`\#}
\def\PYZpc{\char`\%}
\def\PYZdl{\char`\$}
\def\PYZhy{\char`\-}
\def\PYZsq{\char`\'}
\def\PYZdq{\char`\"}
\def\PYZti{\char`\~}
% for compatibility with earlier versions
\def\PYZat{@}
\def\PYZlb{[}
\def\PYZrb{]}
\makeatother


    % For linebreaks inside Verbatim environment from package fancyvrb. 
    \makeatletter
        \newbox\Wrappedcontinuationbox 
        \newbox\Wrappedvisiblespacebox 
        \newcommand*\Wrappedvisiblespace {\textcolor{red}{\textvisiblespace}} 
        \newcommand*\Wrappedcontinuationsymbol {\textcolor{red}{\llap{\tiny$\m@th\hookrightarrow$}}} 
        \newcommand*\Wrappedcontinuationindent {3ex } 
        \newcommand*\Wrappedafterbreak {\kern\Wrappedcontinuationindent\copy\Wrappedcontinuationbox} 
        % Take advantage of the already applied Pygments mark-up to insert 
        % potential linebreaks for TeX processing. 
        %        {, <, #, %, $, ' and ": go to next line. 
        %        _, }, ^, &, >, - and ~: stay at end of broken line. 
        % Use of \textquotesingle for straight quote. 
        \newcommand*\Wrappedbreaksatspecials {% 
            \def\PYGZus{\discretionary{\char`\_}{\Wrappedafterbreak}{\char`\_}}% 
            \def\PYGZob{\discretionary{}{\Wrappedafterbreak\char`\{}{\char`\{}}% 
            \def\PYGZcb{\discretionary{\char`\}}{\Wrappedafterbreak}{\char`\}}}% 
            \def\PYGZca{\discretionary{\char`\^}{\Wrappedafterbreak}{\char`\^}}% 
            \def\PYGZam{\discretionary{\char`\&}{\Wrappedafterbreak}{\char`\&}}% 
            \def\PYGZlt{\discretionary{}{\Wrappedafterbreak\char`\<}{\char`\<}}% 
            \def\PYGZgt{\discretionary{\char`\>}{\Wrappedafterbreak}{\char`\>}}% 
            \def\PYGZsh{\discretionary{}{\Wrappedafterbreak\char`\#}{\char`\#}}% 
            \def\PYGZpc{\discretionary{}{\Wrappedafterbreak\char`\%}{\char`\%}}% 
            \def\PYGZdl{\discretionary{}{\Wrappedafterbreak\char`\$}{\char`\$}}% 
            \def\PYGZhy{\discretionary{\char`\-}{\Wrappedafterbreak}{\char`\-}}% 
            \def\PYGZsq{\discretionary{}{\Wrappedafterbreak\textquotesingle}{\textquotesingle}}% 
            \def\PYGZdq{\discretionary{}{\Wrappedafterbreak\char`\"}{\char`\"}}% 
            \def\PYGZti{\discretionary{\char`\~}{\Wrappedafterbreak}{\char`\~}}% 
        } 
        % Some characters . , ; ? ! / are not pygmentized. 
        % This macro makes them "active" and they will insert potential linebreaks 
        \newcommand*\Wrappedbreaksatpunct {% 
            \lccode`\~`\.\lowercase{\def~}{\discretionary{\hbox{\char`\.}}{\Wrappedafterbreak}{\hbox{\char`\.}}}% 
            \lccode`\~`\,\lowercase{\def~}{\discretionary{\hbox{\char`\,}}{\Wrappedafterbreak}{\hbox{\char`\,}}}% 
            \lccode`\~`\;\lowercase{\def~}{\discretionary{\hbox{\char`\;}}{\Wrappedafterbreak}{\hbox{\char`\;}}}% 
            \lccode`\~`\:\lowercase{\def~}{\discretionary{\hbox{\char`\:}}{\Wrappedafterbreak}{\hbox{\char`\:}}}% 
            \lccode`\~`\?\lowercase{\def~}{\discretionary{\hbox{\char`\?}}{\Wrappedafterbreak}{\hbox{\char`\?}}}% 
            \lccode`\~`\!\lowercase{\def~}{\discretionary{\hbox{\char`\!}}{\Wrappedafterbreak}{\hbox{\char`\!}}}% 
            \lccode`\~`\/\lowercase{\def~}{\discretionary{\hbox{\char`\/}}{\Wrappedafterbreak}{\hbox{\char`\/}}}% 
            \catcode`\.\active
            \catcode`\,\active 
            \catcode`\;\active
            \catcode`\:\active
            \catcode`\?\active
            \catcode`\!\active
            \catcode`\/\active 
            \lccode`\~`\~ 	
        }
    \makeatother

    \let\OriginalVerbatim=\Verbatim
    \makeatletter
    \renewcommand{\Verbatim}[1][1]{%
        %\parskip\z@skip
        \sbox\Wrappedcontinuationbox {\Wrappedcontinuationsymbol}%
        \sbox\Wrappedvisiblespacebox {\FV@SetupFont\Wrappedvisiblespace}%
        \def\FancyVerbFormatLine ##1{\hsize\linewidth
            \vtop{\raggedright\hyphenpenalty\z@\exhyphenpenalty\z@
                \doublehyphendemerits\z@\finalhyphendemerits\z@
                \strut ##1\strut}%
        }%
        % If the linebreak is at a space, the latter will be displayed as visible
        % space at end of first line, and a continuation symbol starts next line.
        % Stretch/shrink are however usually zero for typewriter font.
        \def\FV@Space {%
            \nobreak\hskip\z@ plus\fontdimen3\font minus\fontdimen4\font
            \discretionary{\copy\Wrappedvisiblespacebox}{\Wrappedafterbreak}
            {\kern\fontdimen2\font}%
        }%
        
        % Allow breaks at special characters using \PYG... macros.
        \Wrappedbreaksatspecials
        % Breaks at punctuation characters . , ; ? ! and / need catcode=\active 	
        \OriginalVerbatim[#1,codes*=\Wrappedbreaksatpunct]%
    }
    \makeatother

    % Exact colors from NB
    \definecolor{incolor}{HTML}{303F9F}
    \definecolor{outcolor}{HTML}{D84315}
    \definecolor{cellborder}{HTML}{CFCFCF}
    \definecolor{cellbackground}{HTML}{F7F7F7}
    
    % prompt
    \makeatletter
    \newcommand{\boxspacing}{\kern\kvtcb@left@rule\kern\kvtcb@boxsep}
    \makeatother
    \newcommand{\prompt}[4]{
        \ttfamily\llap{{\color{#2}[#3]:\hspace{3pt}#4}}\vspace{-\baselineskip}
    }
    

    
    % Prevent overflowing lines due to hard-to-break entities
    \sloppy 
    % Setup hyperref package
    \hypersetup{
      breaklinks=true,  % so long urls are correctly broken across lines
      colorlinks=true,
      urlcolor=urlcolor,
      linkcolor=linkcolor,
      citecolor=citecolor,
      }
    % Slightly bigger margins than the latex defaults
    
    \geometry{verbose,tmargin=1in,bmargin=1in,lmargin=1in,rmargin=1in}
    
    % FancyHDR
	\usepackage{fancyhdr}
	\pagestyle{fancyplain}
	\rhead{Chun-Hei (Samuel) Lam \\ ID: 928931321}
	\chead{15.077 Statistical ML and DS \\ Problem Sheet 4}
	\lhead{}
	\setlength{\headheight}{15pt}

    % Document title
    \title{15.077 Statistical ML and DS}

\begin{document}
    \begin{figure}
\includegraphics[scale=0.5]{MIT-Logo-(2).jpg}
\end{figure}

\begin{flushright}
\begin{tabular}{ll}
Module Code:     & 15.077                \\
Lecturer:        & Roy Welsch         \\
Coursework:      & PS.04                     \\
Due Date:        & 5 April, 2021      \\ \\
Student Name:    & Chun Hei (Samuel) LAM          \\
MIT ID:          & 928931321              
\end{tabular}
\end{flushright}

\vspace{2cm}

\begin{center}
    \huge 15.077 Statistical Learning and Data Science \\
    \vspace{0.7cm}
    \Large Problem Sheet 4 \\
    \vspace{0.5cm}
    \Large Experimental Design, Statistical Process Control
    \normalsize
\end{center}

\vspace{1.5cm}

\noindent \textbf{Declaration:}
\begin{tcolorbox}
I pledge that the work submitted for this coursework is my own unassisted work unless stated otherwise.
\end{tcolorbox}
    
\noindent \textbf{Notes from Student:}
\begin{tcolorbox}
Acknowledgement to Harry Yu. Please look at the python notebook attached if there are places that cannot be displayed properly.
\end{tcolorbox}
\newpage

    \begin{tcolorbox}[breakable, size=fbox, boxrule=1pt, pad at break*=1mm,colback=cellbackground, colframe=cellborder]
\prompt{In}{incolor}{1}{\boxspacing}
\begin{Verbatim}[commandchars=\\\{\}]
\PY{k+kn}{import} \PY{n+nn}{numpy} \PY{k}{as} \PY{n+nn}{np}
\PY{k+kn}{from} \PY{n+nn}{scipy} \PY{k+kn}{import} \PY{n}{stats}
\PY{k+kn}{import} \PY{n+nn}{pandas} \PY{k}{as} \PY{n+nn}{pd}
\PY{k+kn}{import} \PY{n+nn}{matplotlib}\PY{n+nn}{.}\PY{n+nn}{pyplot} \PY{k}{as} \PY{n+nn}{plt}
\PY{k+kn}{import} \PY{n+nn}{matplotlib}\PY{n+nn}{.}\PY{n+nn}{cm} \PY{k}{as} \PY{n+nn}{cm}
\PY{k+kn}{import} \PY{n+nn}{seaborn} \PY{k}{as} \PY{n+nn}{sns}
\PY{k+kn}{import} \PY{n+nn}{statsmodels}\PY{n+nn}{.}\PY{n+nn}{api} \PY{k}{as} \PY{n+nn}{sm}
\PY{k+kn}{import} \PY{n+nn}{plotly}\PY{n+nn}{.}\PY{n+nn}{express} \PY{k}{as} \PY{n+nn}{px}
\PY{k+kn}{import} \PY{n+nn}{itertools}
\end{Verbatim}
\end{tcolorbox}

    \hypertarget{rice-12.29-semiconductor}{%
\section{Rice 12.29 Semiconductor}\label{rice-12.29-semiconductor}}

    The performance of a semiconductor depends upon the thickness of a layer
of silicon dioxide. In an experiment (Czitrom and Reece, 1997), layer
thicknesses were measured at three furnace locations for three types of
wafers (virgin wafers, recycled in-house wafers, and recycled wafers
from an external source). The data are contained in the file
\texttt{waferlayers}.

    \begin{tcolorbox}[breakable, size=fbox, boxrule=1pt, pad at break*=1mm,colback=cellbackground, colframe=cellborder]
\prompt{In}{incolor}{20}{\boxspacing}
\begin{Verbatim}[commandchars=\\\{\}]
\PY{n}{Waferlayers} \PY{o}{=} \PY{n}{pd}\PY{o}{.}\PY{n}{read\PYZus{}csv}\PY{p}{(}\PY{l+s+s2}{\PYZdq{}}\PY{l+s+s2}{waferlayers.txt}\PY{l+s+s2}{\PYZdq{}}\PY{p}{,} \PY{n}{quotechar}\PY{o}{=}\PY{l+s+s2}{\PYZdq{}}\PY{l+s+s2}{\PYZsq{}}\PY{l+s+s2}{\PYZdq{}}\PY{p}{)}
\PY{n}{Waferlayers} \PY{o}{=} \PY{n}{Waferlayers}\PY{o}{.}\PY{n}{rename}\PY{p}{(}\PY{n}{columns}\PY{o}{=}\PY{p}{\PYZob{}}\PY{l+s+s2}{\PYZdq{}}\PY{l+s+s2}{Wafer Type}\PY{l+s+s2}{\PYZdq{}}\PY{p}{:} \PY{l+s+s2}{\PYZdq{}}\PY{l+s+s2}{WaferType}\PY{l+s+s2}{\PYZdq{}}\PY{p}{\PYZcb{}}\PY{p}{)}
\end{Verbatim}
\end{tcolorbox}

    Conduct a two-way analysis of variance and test for significance of main
effects and interactions.

    \emph{Solution: We first fit the ANOVA model.}

    \begin{tcolorbox}[breakable, size=fbox, boxrule=1pt, pad at break*=1mm,colback=cellbackground, colframe=cellborder]
\prompt{In}{incolor}{23}{\boxspacing}
\begin{Verbatim}[commandchars=\\\{\}]
\PY{n}{model} \PY{o}{=} \PY{n}{sm}\PY{o}{.}\PY{n}{formula}\PY{o}{.}\PY{n}{ols}\PY{p}{(}\PY{l+s+s1}{\PYZsq{}}\PY{l+s+s1}{Thickness \PYZti{} C(Furnace) + C(WaferType) + C(Furnace):C(WaferType)}\PY{l+s+s1}{\PYZsq{}}\PY{p}{,} \PY{n}{data}\PY{o}{=}\PY{n}{Waferlayers}\PY{p}{)}\PY{o}{.}\PY{n}{fit}\PY{p}{(}\PY{p}{)}
\end{Verbatim}
\end{tcolorbox}

    \emph{We then look at the estimated parameters:}

    \begin{tcolorbox}[breakable, size=fbox, boxrule=1pt, pad at break*=1mm,colback=cellbackground, colframe=cellborder]
\prompt{In}{incolor}{29}{\boxspacing}
\begin{Verbatim}[commandchars=\\\{\}]
\PY{n}{model}\PY{o}{.}\PY{n}{params}
\end{Verbatim}
\end{tcolorbox}

            \begin{tcolorbox}[breakable, size=fbox, boxrule=.5pt, pad at break*=1mm, opacityfill=0]
\prompt{Out}{outcolor}{29}{\boxspacing}
\begin{Verbatim}[commandchars=\\\{\}]
Intercept                                   91.966667
C(Furnace)[T.2]                             -3.333333
C(Furnace)[T.3]                             -1.600000
C(WaferType)[T.In-House]                    -2.033333
C(WaferType)[T.Virgin]                      -1.900000
C(Furnace)[T.2]:C(WaferType)[T.In-House]     4.633333
C(Furnace)[T.3]:C(WaferType)[T.In-House]     2.200000
C(Furnace)[T.2]:C(WaferType)[T.Virgin]       3.333333
C(Furnace)[T.3]:C(WaferType)[T.Virgin]      -0.166667
dtype: float64
\end{Verbatim}
\end{tcolorbox}
        
    \emph{Note: The base case is
\texttt{Furnace\ =\ 1,\ WaferType\ =\ External}. The coefficient
\texttt{C(Furnace){[}T.2{]}} refers to effect of \texttt{Furnace\ =\ 2},
while the coefficient
\texttt{C(Furnace){[}T.3{]}:C(WaferType){[}T.Virgin{]}} refers to joint
effect of \texttt{Furnace\ =\ 3} and
\texttt{WaferType\ =\ \textquotesingle{}Virgin\textquotesingle{}}, so on
and so forth.}

    \emph{We then perform F-test for main effects and interactions.}

    \begin{tcolorbox}[breakable, size=fbox, boxrule=1pt, pad at break*=1mm,colback=cellbackground, colframe=cellborder]
\prompt{In}{incolor}{24}{\boxspacing}
\begin{Verbatim}[commandchars=\\\{\}]
\PY{n}{sm}\PY{o}{.}\PY{n}{stats}\PY{o}{.}\PY{n}{anova\PYZus{}lm}\PY{p}{(}\PY{n}{model}\PY{p}{,} \PY{n}{typ}\PY{o}{=}\PY{l+m+mi}{2}\PY{p}{)}
\end{Verbatim}
\end{tcolorbox}

            \begin{tcolorbox}[breakable, size=fbox, boxrule=.5pt, pad at break*=1mm, opacityfill=0]
\prompt{Out}{outcolor}{24}{\boxspacing}
\begin{Verbatim}[commandchars=\\\{\}]
                            sum\_sq    df         F    PR(>F)
C(Furnace)                4.108889   2.0  1.446038  0.261588
C(WaferType)              5.875556   2.0  2.067779  0.155467
C(Furnace):C(WaferType)  21.348889   4.0  3.756648  0.021618
Residual                 25.573333  18.0       NaN       NaN
\end{Verbatim}
\end{tcolorbox}
        
    \emph{If we use \(\alpha = 0.05\), then we have the following}

\begin{longtable}[]{@{}lll@{}}
\toprule
\begin{minipage}[b]{0.30\columnwidth}\raggedright
\(H_0\)\strut
\end{minipage} & \begin{minipage}[b]{0.30\columnwidth}\raggedright
\(H_1\)\strut
\end{minipage} & \begin{minipage}[b]{0.30\columnwidth}\raggedright
Result\strut
\end{minipage}\tabularnewline
\midrule
\endhead
\begin{minipage}[t]{0.30\columnwidth}\raggedright
\texttt{Furnance} has no effect.\strut
\end{minipage} & \begin{minipage}[t]{0.30\columnwidth}\raggedright
\texttt{Furnance} has effect.\strut
\end{minipage} & \begin{minipage}[t]{0.30\columnwidth}\raggedright
Insufficient evidence to reject \(H_0\).\strut
\end{minipage}\tabularnewline
\begin{minipage}[t]{0.30\columnwidth}\raggedright
\texttt{WaferType} has no effect.\strut
\end{minipage} & \begin{minipage}[t]{0.30\columnwidth}\raggedright
\texttt{WaferType} has effect.\strut
\end{minipage} & \begin{minipage}[t]{0.30\columnwidth}\raggedright
Insufficient evidence to reject \(H_0\).\strut
\end{minipage}\tabularnewline
\begin{minipage}[t]{0.30\columnwidth}\raggedright
Interaction of \texttt{Furnance} and \texttt{WaferType} has no
effect.\strut
\end{minipage} & \begin{minipage}[t]{0.30\columnwidth}\raggedright
Interaction of \texttt{Furnance} and \texttt{WaferType} has
effect.\strut
\end{minipage} & \begin{minipage}[t]{0.30\columnwidth}\raggedright
Sufficient evidence to reject \(H_0\).\strut
\end{minipage}\tabularnewline
\bottomrule
\end{longtable}

Only interaction of \texttt{Furnance} and \texttt{WaferType} has
significant effect.

    Construct a graph such as that shown in Figure 12.3 of textbook. Does
the comparison of layer thicknesses depend on furnace location?

    \emph{Solution: We plot \texttt{Thickness} vs. \texttt{Furnace}.}

    \begin{tcolorbox}[breakable, size=fbox, boxrule=1pt, pad at break*=1mm,colback=cellbackground, colframe=cellborder]
\prompt{In}{incolor}{44}{\boxspacing}
\begin{Verbatim}[commandchars=\\\{\}]
\PY{k+kn}{from} \PY{n+nn}{statsmodels}\PY{n+nn}{.}\PY{n+nn}{graphics}\PY{n+nn}{.}\PY{n+nn}{factorplots} \PY{k+kn}{import} \PY{n}{interaction\PYZus{}plot}
\PY{n}{fig} \PY{o}{=} \PY{n}{interaction\PYZus{}plot}\PY{p}{(}\PY{n}{Waferlayers}\PY{o}{.}\PY{n}{Furnace}\PY{p}{,} \PY{n}{Waferlayers}\PY{o}{.}\PY{n}{WaferType}\PY{p}{,} \PY{n}{Waferlayers}\PY{o}{.}\PY{n}{Thickness}\PY{p}{,} 
                       \PY{n}{colors}\PY{o}{=}\PY{p}{[}\PY{l+s+s1}{\PYZsq{}}\PY{l+s+s1}{red}\PY{l+s+s1}{\PYZsq{}}\PY{p}{,}\PY{l+s+s1}{\PYZsq{}}\PY{l+s+s1}{blue}\PY{l+s+s1}{\PYZsq{}}\PY{p}{,}\PY{l+s+s1}{\PYZsq{}}\PY{l+s+s1}{green}\PY{l+s+s1}{\PYZsq{}}\PY{p}{]}\PY{p}{,} \PY{n}{markers}\PY{o}{=}\PY{p}{[}\PY{l+s+s1}{\PYZsq{}}\PY{l+s+s1}{D}\PY{l+s+s1}{\PYZsq{}}\PY{p}{,}\PY{l+s+s1}{\PYZsq{}}\PY{l+s+s1}{\PYZca{}}\PY{l+s+s1}{\PYZsq{}}\PY{p}{,}\PY{l+s+s1}{\PYZsq{}}\PY{l+s+s1}{o}\PY{l+s+s1}{\PYZsq{}}\PY{p}{]}\PY{p}{,} \PY{n}{ms}\PY{o}{=}\PY{l+m+mi}{8}\PY{p}{)}
\end{Verbatim}
\end{tcolorbox}

    \begin{center}
    \adjustimage{max size={0.9\linewidth}{0.9\paperheight}}{output_16_0.png}
    \end{center}
    { \hspace*{\fill} \\}
    
    \emph{As we can see the line segments are not parallel to each other. If
average thickness depends on solely \texttt{Furnace} then we expect the
line segments to be not horizontal and parallel to each other. The line
segments are indeed not horizontal, but are not parallel to each other
as well. Therefore there are interaction between \texttt{Furnace} and
\texttt{WaferType}, and the layer thickness depends more on the
interaction.}

    \hypertarget{adhesive-process}{%
\section{Adhesive Process}\label{adhesive-process}}

An experiment was conducted using an unreplicated \(2^4\) factorial
design to determine the effects of the amount of glue \texttt{x1},
predrying temperature \texttt{x2}, tunnel temperature \texttt{x3}, and
pressure \texttt{x4} on the adhesiveforce obtained in an adhesive
process.

    \begin{tcolorbox}[breakable, size=fbox, boxrule=1pt, pad at break*=1mm,colback=cellbackground, colframe=cellborder]
\prompt{In}{incolor}{4}{\boxspacing}
\begin{Verbatim}[commandchars=\\\{\}]
\PY{n}{Adhesive} \PY{o}{=} \PY{n}{pd}\PY{o}{.}\PY{n}{read\PYZus{}csv}\PY{p}{(}\PY{l+s+s2}{\PYZdq{}}\PY{l+s+s2}{Adhesive.csv}\PY{l+s+s2}{\PYZdq{}}\PY{p}{)}
\end{Verbatim}
\end{tcolorbox}

    \textbf{Part (a):} Estimate all the main effects and interactions.

    \emph{Solution: the following tabulates all effects and interactions of
variables.}

    \begin{tcolorbox}[breakable, size=fbox, boxrule=1pt, pad at break*=1mm,colback=cellbackground, colframe=cellborder]
\prompt{In}{incolor}{5}{\boxspacing}
\begin{Verbatim}[commandchars=\\\{\}]
\PY{n}{model} \PY{o}{=} \PY{n}{sm}\PY{o}{.}\PY{n}{formula}\PY{o}{.}\PY{n}{ols}\PY{p}{(}\PY{l+s+s1}{\PYZsq{}}\PY{l+s+s1}{y \PYZti{} (x1+x2+x3+x4)**4}\PY{l+s+s1}{\PYZsq{}}\PY{p}{,} \PY{n}{data}\PY{o}{=}\PY{n}{Adhesive}\PY{p}{)}\PY{o}{.}\PY{n}{fit}\PY{p}{(}\PY{p}{)}
\PY{n}{model}\PY{o}{.}\PY{n}{params}
\end{Verbatim}
\end{tcolorbox}

            \begin{tcolorbox}[breakable, size=fbox, boxrule=.5pt, pad at break*=1mm, opacityfill=0]
\prompt{Out}{outcolor}{5}{\boxspacing}
\begin{Verbatim}[commandchars=\\\{\}]
Intercept      4.301875
x1             0.311875
x2             0.543125
x3             0.345625
x4            -0.096875
x1:x2         -0.004375
x1:x3          0.160625
x1:x4         -0.049375
x2:x3          0.149375
x2:x4          0.374375
x3:x4          0.014375
x1:x2:x3      -0.138125
x1:x2:x4      -0.010625
x1:x3:x4       0.196875
x2:x3:x4      -0.021875
x1:x2:x3:x4   -0.016875
dtype: float64
\end{Verbatim}
\end{tcolorbox}
        
    \textbf{Part (b):} Plot the estimated effects on a normal probability
plot.

    \begin{tcolorbox}[breakable, size=fbox, boxrule=1pt, pad at break*=1mm,colback=cellbackground, colframe=cellborder]
\prompt{In}{incolor}{6}{\boxspacing}
\begin{Verbatim}[commandchars=\\\{\}]
\PY{n}{fit\PYZus{}val} \PY{o}{=} \PY{n}{model}\PY{o}{.}\PY{n}{params}\PY{o}{.}\PY{n}{sort\PYZus{}values}\PY{p}{(}\PY{p}{)}\PY{o}{.}\PY{n}{values}\PY{p}{[}\PY{p}{:}\PY{n+nb}{len}\PY{p}{(}\PY{n}{model}\PY{o}{.}\PY{n}{params}\PY{p}{)}\PY{o}{\PYZhy{}}\PY{l+m+mi}{1}\PY{p}{]}
\PY{n}{name} \PY{o}{=} \PY{n}{model}\PY{o}{.}\PY{n}{params}\PY{o}{.}\PY{n}{sort\PYZus{}values}\PY{p}{(}\PY{p}{)}\PY{o}{.}\PY{n}{\PYZus{}index}\PY{p}{[}\PY{p}{:}\PY{n+nb}{len}\PY{p}{(}\PY{n}{model}\PY{o}{.}\PY{n}{params}\PY{p}{)}\PY{o}{\PYZhy{}}\PY{l+m+mi}{1}\PY{p}{]}
\end{Verbatim}
\end{tcolorbox}

    \begin{tcolorbox}[breakable, size=fbox, boxrule=1pt, pad at break*=1mm,colback=cellbackground, colframe=cellborder]
\prompt{In}{incolor}{7}{\boxspacing}
\begin{Verbatim}[commandchars=\\\{\}]
\PY{n}{mu}\PY{p}{,} \PY{n}{std} \PY{o}{=} \PY{n}{stats}\PY{o}{.}\PY{n}{norm}\PY{o}{.}\PY{n}{fit}\PY{p}{(}\PY{n}{fit\PYZus{}val}\PY{p}{,} \PY{n}{floc}\PY{o}{=}\PY{l+m+mi}{0}\PY{p}{)}
\PY{n}{osm}\PY{p}{,} \PY{n}{osr} \PY{o}{=} \PY{n}{stats}\PY{o}{.}\PY{n}{probplot}\PY{p}{(}\PY{n}{fit\PYZus{}val}\PY{p}{,} \PY{n}{dist}\PY{o}{=}\PY{n}{stats}\PY{o}{.}\PY{n}{norm}\PY{p}{(}\PY{n}{loc}\PY{o}{=}\PY{l+m+mi}{0}\PY{p}{,} \PY{n}{scale}\PY{o}{=}\PY{n}{std}\PY{p}{)}\PY{p}{)}
\end{Verbatim}
\end{tcolorbox}

    \begin{tcolorbox}[breakable, size=fbox, boxrule=1pt, pad at break*=1mm,colback=cellbackground, colframe=cellborder]
\prompt{In}{incolor}{51}{\boxspacing}
\begin{Verbatim}[commandchars=\\\{\}]
\PY{n}{colors} \PY{o}{=} \PY{n}{cm}\PY{o}{.}\PY{n}{rainbow}\PY{p}{(}\PY{n}{np}\PY{o}{.}\PY{n}{linspace}\PY{p}{(}\PY{l+m+mi}{0}\PY{p}{,} \PY{l+m+mi}{1}\PY{p}{,} \PY{n+nb}{len}\PY{p}{(}\PY{n}{fit\PYZus{}val}\PY{p}{)}\PY{p}{)}\PY{p}{)}
\PY{n}{xvec} \PY{o}{=} \PY{n}{np}\PY{o}{.}\PY{n}{linspace}\PY{p}{(}\PY{o}{\PYZhy{}}\PY{l+m+mf}{0.4}\PY{p}{,} \PY{l+m+mf}{0.6}\PY{p}{,} \PY{l+m+mi}{2}\PY{p}{)}

\PY{n}{fig}\PY{p}{,} \PY{n}{ax} \PY{o}{=} \PY{n}{plt}\PY{o}{.}\PY{n}{subplots}\PY{p}{(}\PY{n}{nrows} \PY{o}{=} \PY{l+m+mi}{1}\PY{p}{,} \PY{n}{ncols} \PY{o}{=} \PY{l+m+mi}{1}\PY{p}{,} \PY{n}{figsize}\PY{o}{=}\PY{p}{(}\PY{l+m+mi}{8}\PY{p}{,}\PY{l+m+mi}{6}\PY{p}{)}\PY{p}{)}
\PY{c+c1}{\PYZsh{} ax.scatter(osm[0], osm[1])}

\PY{k}{for} \PY{n}{x}\PY{p}{,}\PY{n}{y}\PY{p}{,}\PY{n}{c}\PY{p}{,}\PY{n}{txt} \PY{o+ow}{in} \PY{n+nb}{zip}\PY{p}{(}\PY{n}{osm}\PY{p}{[}\PY{l+m+mi}{0}\PY{p}{]}\PY{p}{,} \PY{n}{osm}\PY{p}{[}\PY{l+m+mi}{1}\PY{p}{]}\PY{p}{,} \PY{n}{colors}\PY{p}{,} \PY{n}{name}\PY{p}{)}\PY{p}{:}
    \PY{n}{ax}\PY{o}{.}\PY{n}{scatter}\PY{p}{(}\PY{n}{x}\PY{p}{,} \PY{n}{y}\PY{p}{,} \PY{n}{color}\PY{o}{=}\PY{n}{c}\PY{p}{,} \PY{n}{label}\PY{o}{=}\PY{n}{txt}\PY{p}{)}
    \PY{n}{ax}\PY{o}{.}\PY{n}{plot}\PY{p}{(}\PY{n}{xvec}\PY{p}{,} \PY{n}{xvec}\PY{p}{)}
    \PY{c+c1}{\PYZsh{} ax.annotate(txt, (x,y))}

\PY{n}{ax}\PY{o}{.}\PY{n}{plot}\PY{p}{(}\PY{n}{xvec}\PY{p}{,}\PY{n}{xvec}\PY{p}{,}\PY{n}{color}\PY{o}{=}\PY{l+s+s2}{\PYZdq{}}\PY{l+s+s2}{red}\PY{l+s+s2}{\PYZdq{}}\PY{p}{)}
\PY{n}{ax}\PY{o}{.}\PY{n}{legend}\PY{p}{(}\PY{n}{bbox\PYZus{}to\PYZus{}anchor}\PY{o}{=}\PY{p}{(}\PY{l+m+mf}{1.05}\PY{p}{,} \PY{l+m+mf}{1.01}\PY{p}{)}\PY{p}{,} \PY{n}{loc}\PY{o}{=}\PY{l+s+s1}{\PYZsq{}}\PY{l+s+s1}{upper left}\PY{l+s+s1}{\PYZsq{}}\PY{p}{)}
\PY{n}{ax}\PY{o}{.}\PY{n}{set\PYZus{}xlabel}\PY{p}{(}\PY{l+s+s2}{\PYZdq{}}\PY{l+s+s2}{Theoretical Quantiles}\PY{l+s+s2}{\PYZdq{}}\PY{p}{)}
\PY{n}{ax}\PY{o}{.}\PY{n}{set\PYZus{}ylabel}\PY{p}{(}\PY{l+s+s2}{\PYZdq{}}\PY{l+s+s2}{Sample Quantiles}\PY{l+s+s2}{\PYZdq{}}\PY{p}{)}
\end{Verbatim}
\end{tcolorbox}

            \begin{tcolorbox}[breakable, size=fbox, boxrule=.5pt, pad at break*=1mm, opacityfill=0]
\prompt{Out}{outcolor}{51}{\boxspacing}
\begin{Verbatim}[commandchars=\\\{\}]
Text(0, 0.5, 'Sample Quantiles')
\end{Verbatim}
\end{tcolorbox}
        
    \begin{center}
    \adjustimage{max size={0.9\linewidth}{0.9\paperheight}}{output_26_1.png}
    \end{center}
    { \hspace*{\fill} \\}
    
    \textbf{Part (c):} Interpret your results.

    \emph{Solution: As we can see from the figure, most of the points are
far from the line \texttt{x=x}. The points there are closed to
\texttt{x=x} corresponds to the factors which are not significant, which
are \texttt{x1:x2} and \texttt{x3:x4} in our case.}

    \hypertarget{plutonium}{%
\section{Plutonium}\label{plutonium}}

A \(2^{8-4}_{\text{IV}}\) fractional factorial design was run to
identify sources of plutonium contamination in the radioactivity
material analysis of dried shellfish at the National Institute of
Standards and Technology (NIST). The data are in the table below. No
contamination occurred at runs 1,4, and 9.

    \begin{tcolorbox}[breakable, size=fbox, boxrule=1pt, pad at break*=1mm,colback=cellbackground, colframe=cellborder]
\prompt{In}{incolor}{2}{\boxspacing}
\begin{Verbatim}[commandchars=\\\{\}]
\PY{n}{Plutonium} \PY{o}{=} \PY{n}{pd}\PY{o}{.}\PY{n}{read\PYZus{}csv}\PY{p}{(}\PY{l+s+s2}{\PYZdq{}}\PY{l+s+s2}{Plutonium.csv}\PY{l+s+s2}{\PYZdq{}}\PY{p}{)}
\end{Verbatim}
\end{tcolorbox}

    The factors and levels are shown in the following table.

\begin{longtable}[]{@{}llll@{}}
\toprule
Factor & Code & -1 & +1\tabularnewline
\midrule
\endhead
Glassware & \texttt{x1} & Distilled Water & Soap, acid,
stored\tabularnewline
Reagent & \texttt{x2} & New & Old\tabularnewline
Sample Prep & \texttt{x3} & Co-precipitation &
Electrodeposition\tabularnewline
Tracer & \texttt{x4} & Stock & Fresh\tabularnewline
Dissolution & \texttt{x5} & Without & With\tabularnewline
Hood & \texttt{x6} & B & A\tabularnewline
Chemistry & \texttt{x7} & Without & With\tabularnewline
Ashing & \texttt{x8} & Without & With\tabularnewline
\bottomrule
\end{longtable}

    \textbf{Part (a):} Write down the alias relationships.

    \emph{Solution:} We first identify the 4 design generators in our
experiment. From observation we have the generators:
\texttt{x5=x2*x3*x4}, \texttt{x6=x1*x3*x4}, \texttt{x7=x1*x2*x3} and
\texttt{x8=x1*x2*x4}. We may check by looking at the columns.

    \begin{tcolorbox}[breakable, size=fbox, boxrule=1pt, pad at break*=1mm,colback=cellbackground, colframe=cellborder]
\prompt{In}{incolor}{3}{\boxspacing}
\begin{Verbatim}[commandchars=\\\{\}]
\PY{n+nb}{print}\PY{p}{(}\PY{p}{(}\PY{n+nb}{all}\PY{p}{(}\PY{n}{Plutonium}\PY{p}{[}\PY{l+s+s2}{\PYZdq{}}\PY{l+s+s2}{x5}\PY{l+s+s2}{\PYZdq{}}\PY{p}{]} \PY{o}{==} \PY{n}{Plutonium}\PY{p}{[}\PY{l+s+s2}{\PYZdq{}}\PY{l+s+s2}{x2}\PY{l+s+s2}{\PYZdq{}}\PY{p}{]} \PY{o}{*} \PY{n}{Plutonium}\PY{p}{[}\PY{l+s+s2}{\PYZdq{}}\PY{l+s+s2}{x3}\PY{l+s+s2}{\PYZdq{}}\PY{p}{]} \PY{o}{*} \PY{n}{Plutonium}\PY{p}{[}\PY{l+s+s2}{\PYZdq{}}\PY{l+s+s2}{x4}\PY{l+s+s2}{\PYZdq{}}\PY{p}{]}\PY{p}{)}\PY{p}{,} 
    \PY{n+nb}{all}\PY{p}{(}\PY{n}{Plutonium}\PY{p}{[}\PY{l+s+s2}{\PYZdq{}}\PY{l+s+s2}{x6}\PY{l+s+s2}{\PYZdq{}}\PY{p}{]} \PY{o}{==} \PY{n}{Plutonium}\PY{p}{[}\PY{l+s+s2}{\PYZdq{}}\PY{l+s+s2}{x1}\PY{l+s+s2}{\PYZdq{}}\PY{p}{]} \PY{o}{*} \PY{n}{Plutonium}\PY{p}{[}\PY{l+s+s2}{\PYZdq{}}\PY{l+s+s2}{x3}\PY{l+s+s2}{\PYZdq{}}\PY{p}{]} \PY{o}{*} \PY{n}{Plutonium}\PY{p}{[}\PY{l+s+s2}{\PYZdq{}}\PY{l+s+s2}{x4}\PY{l+s+s2}{\PYZdq{}}\PY{p}{]}\PY{p}{)}\PY{p}{,}
    \PY{n+nb}{all}\PY{p}{(}\PY{n}{Plutonium}\PY{p}{[}\PY{l+s+s2}{\PYZdq{}}\PY{l+s+s2}{x7}\PY{l+s+s2}{\PYZdq{}}\PY{p}{]} \PY{o}{==} \PY{n}{Plutonium}\PY{p}{[}\PY{l+s+s2}{\PYZdq{}}\PY{l+s+s2}{x1}\PY{l+s+s2}{\PYZdq{}}\PY{p}{]} \PY{o}{*} \PY{n}{Plutonium}\PY{p}{[}\PY{l+s+s2}{\PYZdq{}}\PY{l+s+s2}{x2}\PY{l+s+s2}{\PYZdq{}}\PY{p}{]} \PY{o}{*} \PY{n}{Plutonium}\PY{p}{[}\PY{l+s+s2}{\PYZdq{}}\PY{l+s+s2}{x3}\PY{l+s+s2}{\PYZdq{}}\PY{p}{]}\PY{p}{)}\PY{p}{,}
    \PY{n+nb}{all}\PY{p}{(}\PY{n}{Plutonium}\PY{p}{[}\PY{l+s+s2}{\PYZdq{}}\PY{l+s+s2}{x8}\PY{l+s+s2}{\PYZdq{}}\PY{p}{]} \PY{o}{==} \PY{n}{Plutonium}\PY{p}{[}\PY{l+s+s2}{\PYZdq{}}\PY{l+s+s2}{x1}\PY{l+s+s2}{\PYZdq{}}\PY{p}{]} \PY{o}{*} \PY{n}{Plutonium}\PY{p}{[}\PY{l+s+s2}{\PYZdq{}}\PY{l+s+s2}{x2}\PY{l+s+s2}{\PYZdq{}}\PY{p}{]} \PY{o}{*} \PY{n}{Plutonium}\PY{p}{[}\PY{l+s+s2}{\PYZdq{}}\PY{l+s+s2}{x4}\PY{l+s+s2}{\PYZdq{}}\PY{p}{]}\PY{p}{)}\PY{p}{)}\PY{p}{)}
\end{Verbatim}
\end{tcolorbox}

    \begin{Verbatim}[commandchars=\\\{\}]
(True, True, True, True)
    \end{Verbatim}

    We then write the generators in standard form:
\texttt{I\ =\ x2*x3*x4*x5\ =\ x1*x3*x4*x6\ =\ x1*x2*x3*x7\ =\ x1*x2*x4*x8}.

    Finally, we may multiply the generators by other effects to give a
complete list of aliasing effects.

    \begin{tcolorbox}[breakable, size=fbox, boxrule=1pt, pad at break*=1mm,colback=cellbackground, colframe=cellborder]
\prompt{In}{incolor}{140}{\boxspacing}
\begin{Verbatim}[commandchars=\\\{\}]
\PY{n}{design} \PY{o}{=} \PY{n}{Plutonium}\PY{o}{.}\PY{n}{drop}\PY{p}{(}\PY{n}{columns}\PY{o}{=}\PY{p}{[}\PY{l+s+s2}{\PYZdq{}}\PY{l+s+s2}{Run}\PY{l+s+s2}{\PYZdq{}}\PY{p}{,}\PY{l+s+s2}{\PYZdq{}}\PY{l+s+s2}{y}\PY{l+s+s2}{\PYZdq{}}\PY{p}{]}\PY{p}{)}
\PY{n}{design2} \PY{o}{=} \PY{n}{design}\PY{o}{.}\PY{n}{copy}\PY{p}{(}\PY{p}{)} \PY{o}{*} \PY{o}{\PYZhy{}}\PY{l+m+mi}{1}
\PY{n}{design2}\PY{o}{.}\PY{n}{iloc}\PY{p}{[}\PY{p}{:}\PY{p}{,}\PY{o}{\PYZhy{}}\PY{l+m+mi}{4}\PY{p}{:}\PY{p}{]} \PY{o}{=} \PY{l+m+mi}{1}
\PY{n}{design2\PYZus{}copy} \PY{o}{=} \PY{n}{np}\PY{o}{.}\PY{n}{array}\PY{p}{(}\PY{n+nb}{list}\PY{p}{(}\PY{n}{itertools}\PY{o}{.}\PY{n}{repeat}\PY{p}{(}\PY{n}{design2}\PY{o}{.}\PY{n}{values}\PY{p}{,} \PY{l+m+mi}{16}\PY{p}{)}\PY{p}{)}\PY{p}{)}\PY{o}{.}\PY{n}{reshape}\PY{p}{(}\PY{p}{(}\PY{o}{\PYZhy{}}\PY{l+m+mi}{1}\PY{p}{,}\PY{l+m+mi}{8}\PY{p}{)}\PY{p}{)}
\PY{n}{ortho} \PY{o}{=} \PY{n}{pd}\PY{o}{.}\PY{n}{DataFrame}\PY{p}{(}\PY{n}{design2\PYZus{}copy}\PY{p}{,} \PY{n}{columns}\PY{o}{=}\PY{n}{design2}\PY{o}{.}\PY{n}{columns}\PY{p}{,} 
                     \PY{n}{index}\PY{o}{=}\PY{n}{pd}\PY{o}{.}\PY{n}{MultiIndex}\PY{o}{.}\PY{n}{from\PYZus{}product}\PY{p}{(}\PY{p}{[}\PY{n}{design}\PY{o}{.}\PY{n}{index}\PY{p}{,} \PY{n}{np}\PY{o}{.}\PY{n}{arange}\PY{p}{(}\PY{l+m+mi}{16}\PY{p}{)}\PY{p}{]}\PY{p}{,} \PY{n}{names}\PY{o}{=}\PY{p}{[}\PY{l+s+s2}{\PYZdq{}}\PY{l+s+s2}{X}\PY{l+s+s2}{\PYZdq{}}\PY{p}{,}\PY{l+s+s2}{\PYZdq{}}\PY{l+s+s2}{Y}\PY{l+s+s2}{\PYZdq{}}\PY{p}{]}\PY{p}{)}\PY{p}{)}
\end{Verbatim}
\end{tcolorbox}

    \begin{tcolorbox}[breakable, size=fbox, boxrule=1pt, pad at break*=1mm,colback=cellbackground, colframe=cellborder]
\prompt{In}{incolor}{141}{\boxspacing}
\begin{Verbatim}[commandchars=\\\{\}]
\PY{n}{prod} \PY{o}{=} \PY{n}{design}\PY{o}{.}\PY{n}{mul}\PY{p}{(}\PY{n}{ortho}\PY{p}{,} \PY{n}{level}\PY{o}{=}\PY{l+s+s2}{\PYZdq{}}\PY{l+s+s2}{X}\PY{l+s+s2}{\PYZdq{}}\PY{p}{)}
\end{Verbatim}
\end{tcolorbox}

    \begin{tcolorbox}[breakable, size=fbox, boxrule=1pt, pad at break*=1mm,colback=cellbackground, colframe=cellborder]
\prompt{In}{incolor}{160}{\boxspacing}
\begin{Verbatim}[commandchars=\\\{\}]
\PY{n}{equiv\PYZus{}class} \PY{o}{=} \PY{n}{prod}\PY{o}{.}\PY{n}{apply}\PY{p}{(}\PY{k}{lambda} \PY{n}{row}\PY{p}{:} \PY{l+s+sa}{f}\PY{l+s+s2}{\PYZdq{}}\PY{l+s+s2}{(}\PY{l+s+s2}{\PYZob{}}\PY{l+s+s2}{((row == 1)*row.index.str[1]).sum()\PYZcb{})}\PY{l+s+s2}{\PYZdq{}}\PY{p}{,} \PY{n}{axis} \PY{o}{=} \PY{l+m+mi}{1}\PY{p}{)}
\PY{n}{equiv\PYZus{}class}\PY{o}{.}\PY{n}{iloc}\PY{p}{[}\PY{l+m+mi}{0}\PY{p}{]} \PY{o}{=} \PY{l+s+s2}{\PYZdq{}}\PY{l+s+s2}{(I)}\PY{l+s+s2}{\PYZdq{}}
\PY{n}{equiv\PYZus{}class} \PY{o}{=} \PY{n}{equiv\PYZus{}class}\PY{o}{.}\PY{n}{groupby}\PY{p}{(}\PY{l+s+s2}{\PYZdq{}}\PY{l+s+s2}{Y}\PY{l+s+s2}{\PYZdq{}}\PY{p}{)}\PY{o}{.}\PY{n}{apply}\PY{p}{(}\PY{l+s+s2}{\PYZdq{}}\PY{l+s+s2}{=}\PY{l+s+s2}{\PYZdq{}}\PY{o}{.}\PY{n}{join}\PY{p}{)}
\PY{n+nb}{print}\PY{p}{(}\PY{l+s+s2}{\PYZdq{}}\PY{l+s+se}{\PYZbs{}n}\PY{l+s+s2}{\PYZdq{}}\PY{o}{.}\PY{n}{join}\PY{p}{(}\PY{n}{equiv\PYZus{}class}\PY{p}{)}\PY{p}{)}
\end{Verbatim}
\end{tcolorbox}

    \begin{Verbatim}[commandchars=\\\{\}]
(I)=(1678)=(2578)=(1256)=(3567)=(1358)=(2368)=(1237)=(4568)=(1457)=(2467)=(1248)
=(3478)=(1346)=(2345)=(12345678)
(1)=(678)=(12578)=(256)=(13567)=(358)=(12368)=(237)=(14568)=(457)=(12467)=(248)=
(13478)=(346)=(12345)=(2345678)
(2)=(12678)=(578)=(156)=(23567)=(12358)=(368)=(137)=(24568)=(12457)=(467)=(148)=
(23478)=(12346)=(345)=(1345678)
(12)=(2678)=(1578)=(56)=(123567)=(2358)=(1368)=(37)=(124568)=(2457)=(1467)=(48)=
(123478)=(2346)=(1345)=(345678)
(3)=(13678)=(23578)=(12356)=(567)=(158)=(268)=(127)=(34568)=(13457)=(23467)=(123
48)=(478)=(146)=(245)=(1245678)
(13)=(3678)=(123578)=(2356)=(1567)=(58)=(1268)=(27)=(134568)=(3457)=(123467)=(23
48)=(1478)=(46)=(1245)=(245678)
(23)=(123678)=(3578)=(1356)=(2567)=(1258)=(68)=(17)=(234568)=(123457)=(3467)=(13
48)=(2478)=(1246)=(45)=(145678)
(123)=(23678)=(13578)=(356)=(12567)=(258)=(168)=(7)=(1234568)=(23457)=(13467)=(3
48)=(12478)=(246)=(145)=(45678)
(4)=(14678)=(24578)=(12456)=(34567)=(13458)=(23468)=(12347)=(568)=(157)=(267)=(1
28)=(378)=(136)=(235)=(1235678)
(14)=(4678)=(124578)=(2456)=(134567)=(3458)=(123468)=(2347)=(1568)=(57)=(1267)=(
28)=(1378)=(36)=(1235)=(235678)
(24)=(124678)=(4578)=(1456)=(234567)=(123458)=(3468)=(1347)=(2568)=(1257)=(67)=(
18)=(2378)=(1236)=(35)=(135678)
(124)=(24678)=(14578)=(456)=(1234567)=(23458)=(13468)=(347)=(12568)=(257)=(167)=
(8)=(12378)=(236)=(135)=(35678)
(34)=(134678)=(234578)=(123456)=(4567)=(1458)=(2468)=(1247)=(3568)=(1357)=(2367)
=(1238)=(78)=(16)=(25)=(125678)
(134)=(34678)=(1234578)=(23456)=(14567)=(458)=(12468)=(247)=(13568)=(357)=(12367
)=(238)=(178)=(6)=(125)=(25678)
(234)=(1234678)=(34578)=(13456)=(24567)=(12458)=(468)=(147)=(23568)=(12357)=(367
)=(138)=(278)=(126)=(5)=(15678)
(1234)=(234678)=(134578)=(3456)=(124567)=(2458)=(1468)=(47)=(123568)=(2357)=(136
7)=(38)=(1278)=(26)=(15)=(5678)
    \end{Verbatim}

    \textbf{Part (b):} Estimate the main effects.

    \begin{tcolorbox}[breakable, size=fbox, boxrule=1pt, pad at break*=1mm,colback=cellbackground, colframe=cellborder]
\prompt{In}{incolor}{9}{\boxspacing}
\begin{Verbatim}[commandchars=\\\{\}]
\PY{n}{model} \PY{o}{=} \PY{n}{sm}\PY{o}{.}\PY{n}{formula}\PY{o}{.}\PY{n}{ols}\PY{p}{(}\PY{l+s+s1}{\PYZsq{}}\PY{l+s+s1}{y \PYZti{} x1+x2+x3+x4+x5+x6+x7+x8+x1*x2+x1*x3+x1*x4+x1*x5+x1*x6+x1*x7+x1*x8}\PY{l+s+s1}{\PYZsq{}}\PY{p}{,} \PY{n}{data}\PY{o}{=}\PY{n}{Plutonium}\PY{p}{)}\PY{o}{.}\PY{n}{fit}\PY{p}{(}\PY{p}{)}
\PY{n}{model}\PY{o}{.}\PY{n}{params}
\end{Verbatim}
\end{tcolorbox}

            \begin{tcolorbox}[breakable, size=fbox, boxrule=.5pt, pad at break*=1mm, opacityfill=0]
\prompt{Out}{outcolor}{9}{\boxspacing}
\begin{Verbatim}[commandchars=\\\{\}]
Intercept    0.778556
x1           0.724856
x2          -0.431181
x3           0.301694
x4           0.325931
x5          -0.402619
x6           0.693194
x7           0.029556
x8          -0.006431
x1:x2       -0.435406
x1:x3        0.290569
x1:x4        0.309306
x1:x5       -0.378294
x1:x6        0.685894
x1:x7       -0.008794
x1:x8       -0.006831
dtype: float64
\end{Verbatim}
\end{tcolorbox}
        
    \textbf{Part (c):} Prepare a normal probability plot for the effects and
interpret the results.

    \begin{tcolorbox}[breakable, size=fbox, boxrule=1pt, pad at break*=1mm,colback=cellbackground, colframe=cellborder]
\prompt{In}{incolor}{26}{\boxspacing}
\begin{Verbatim}[commandchars=\\\{\}]
\PY{n}{fit\PYZus{}val} \PY{o}{=} \PY{n}{model}\PY{o}{.}\PY{n}{params}\PY{o}{.}\PY{n}{sort\PYZus{}values}\PY{p}{(}\PY{p}{)}\PY{o}{.}\PY{n}{values}\PY{p}{[}\PY{p}{:}\PY{n+nb}{len}\PY{p}{(}\PY{n}{model}\PY{o}{.}\PY{n}{params}\PY{p}{)}\PY{o}{\PYZhy{}}\PY{l+m+mi}{1}\PY{p}{]}
\PY{n}{name} \PY{o}{=} \PY{n}{model}\PY{o}{.}\PY{n}{params}\PY{o}{.}\PY{n}{sort\PYZus{}values}\PY{p}{(}\PY{p}{)}\PY{o}{.}\PY{n}{\PYZus{}index}\PY{p}{[}\PY{p}{:}\PY{n+nb}{len}\PY{p}{(}\PY{n}{model}\PY{o}{.}\PY{n}{params}\PY{p}{)}\PY{o}{\PYZhy{}}\PY{l+m+mi}{1}\PY{p}{]}
\PY{n}{mu}\PY{p}{,} \PY{n}{std} \PY{o}{=} \PY{n}{stats}\PY{o}{.}\PY{n}{norm}\PY{o}{.}\PY{n}{fit}\PY{p}{(}\PY{n}{fit\PYZus{}val}\PY{p}{,} \PY{n}{floc}\PY{o}{=}\PY{l+m+mi}{0}\PY{p}{)}
\PY{n}{osm}\PY{p}{,} \PY{n}{osr} \PY{o}{=} \PY{n}{stats}\PY{o}{.}\PY{n}{probplot}\PY{p}{(}\PY{n}{fit\PYZus{}val}\PY{p}{,} \PY{n}{dist}\PY{o}{=}\PY{n}{stats}\PY{o}{.}\PY{n}{norm}\PY{p}{(}\PY{n}{loc}\PY{o}{=}\PY{l+m+mi}{0}\PY{p}{,} \PY{n}{scale}\PY{o}{=}\PY{n}{std}\PY{p}{)}\PY{p}{)}
\PY{n}{colors} \PY{o}{=} \PY{n}{cm}\PY{o}{.}\PY{n}{rainbow}\PY{p}{(}\PY{n}{np}\PY{o}{.}\PY{n}{linspace}\PY{p}{(}\PY{l+m+mi}{0}\PY{p}{,} \PY{l+m+mi}{1}\PY{p}{,} \PY{n+nb}{len}\PY{p}{(}\PY{n}{fit\PYZus{}val}\PY{p}{)}\PY{p}{)}\PY{p}{)}
\PY{n}{xvec} \PY{o}{=} \PY{n}{np}\PY{o}{.}\PY{n}{linspace}\PY{p}{(}\PY{o}{\PYZhy{}}\PY{l+m+mf}{0.8}\PY{p}{,} \PY{l+m+mf}{0.8}\PY{p}{,} \PY{l+m+mi}{2}\PY{p}{)}

\PY{n}{fig}\PY{p}{,} \PY{n}{ax} \PY{o}{=} \PY{n}{plt}\PY{o}{.}\PY{n}{subplots}\PY{p}{(}\PY{n}{nrows} \PY{o}{=} \PY{l+m+mi}{1}\PY{p}{,} \PY{n}{ncols} \PY{o}{=} \PY{l+m+mi}{1}\PY{p}{,} \PY{n}{figsize}\PY{o}{=}\PY{p}{(}\PY{l+m+mi}{8}\PY{p}{,}\PY{l+m+mi}{6}\PY{p}{)}\PY{p}{)}
\PY{c+c1}{\PYZsh{} ax.scatter(osm[0], osm[1])}

\PY{k}{for} \PY{n}{x}\PY{p}{,}\PY{n}{y}\PY{p}{,}\PY{n}{c}\PY{p}{,}\PY{n}{txt} \PY{o+ow}{in} \PY{n+nb}{zip}\PY{p}{(}\PY{n}{osm}\PY{p}{[}\PY{l+m+mi}{0}\PY{p}{]}\PY{p}{,} \PY{n}{osm}\PY{p}{[}\PY{l+m+mi}{1}\PY{p}{]}\PY{p}{,} \PY{n}{colors}\PY{p}{,} \PY{n}{name}\PY{p}{)}\PY{p}{:}
    \PY{n}{ax}\PY{o}{.}\PY{n}{scatter}\PY{p}{(}\PY{n}{x}\PY{p}{,} \PY{n}{y}\PY{p}{,} \PY{n}{color}\PY{o}{=}\PY{n}{c}\PY{p}{,} \PY{n}{label}\PY{o}{=}\PY{n}{txt}\PY{p}{)}
    \PY{c+c1}{\PYZsh{} ax.annotate(txt, (x,y))}

\PY{n}{ax}\PY{o}{.}\PY{n}{plot}\PY{p}{(}\PY{n}{xvec}\PY{p}{,}\PY{n}{xvec}\PY{p}{,}\PY{n}{color}\PY{o}{=}\PY{l+s+s2}{\PYZdq{}}\PY{l+s+s2}{red}\PY{l+s+s2}{\PYZdq{}}\PY{p}{)}
\PY{n}{ax}\PY{o}{.}\PY{n}{legend}\PY{p}{(}\PY{n}{bbox\PYZus{}to\PYZus{}anchor}\PY{o}{=}\PY{p}{(}\PY{l+m+mf}{1.05}\PY{p}{,} \PY{l+m+mf}{1.01}\PY{p}{)}\PY{p}{,} \PY{n}{loc}\PY{o}{=}\PY{l+s+s1}{\PYZsq{}}\PY{l+s+s1}{upper left}\PY{l+s+s1}{\PYZsq{}}\PY{p}{)}
\PY{n}{ax}\PY{o}{.}\PY{n}{set\PYZus{}xlabel}\PY{p}{(}\PY{l+s+s2}{\PYZdq{}}\PY{l+s+s2}{Theoretical Quantiles}\PY{l+s+s2}{\PYZdq{}}\PY{p}{)}
\PY{n}{ax}\PY{o}{.}\PY{n}{set\PYZus{}ylabel}\PY{p}{(}\PY{l+s+s2}{\PYZdq{}}\PY{l+s+s2}{Sample Quantiles}\PY{l+s+s2}{\PYZdq{}}\PY{p}{)}
\end{Verbatim}
\end{tcolorbox}

            \begin{tcolorbox}[breakable, size=fbox, boxrule=.5pt, pad at break*=1mm, opacityfill=0]
\prompt{Out}{outcolor}{26}{\boxspacing}
\begin{Verbatim}[commandchars=\\\{\}]
Text(0, 0.5, 'Sample Quantiles')
\end{Verbatim}
\end{tcolorbox}
        
    \begin{center}
    \adjustimage{max size={0.9\linewidth}{0.9\paperheight}}{output_43_1.png}
    \end{center}
    { \hspace*{\fill} \\}
    
    \emph{Comment: The outliers seems to be \texttt{x1:x2}, \texttt{x1:x7},
\texttt{x1:x3}, \texttt{x1:x6}. One thing worth noticing is that the
data \(y\) itself is not normal - it is dominated by some outliers above
1. That might be the reason why the estimated effects are ``grouped''
together. To conclude, we should have applied variance-stablising
transformations to the response before analysing our data\ldots{}}

    \begin{tcolorbox}[breakable, size=fbox, boxrule=1pt, pad at break*=1mm,colback=cellbackground, colframe=cellborder]
\prompt{In}{incolor}{174}{\boxspacing}
\begin{Verbatim}[commandchars=\\\{\}]
\PY{n}{fig}\PY{p}{,} \PY{n}{ax} \PY{o}{=} \PY{n}{plt}\PY{o}{.}\PY{n}{subplots}\PY{p}{(}\PY{n}{nrows} \PY{o}{=} \PY{l+m+mi}{1}\PY{p}{,} \PY{n}{ncols} \PY{o}{=} \PY{l+m+mi}{1}\PY{p}{,} \PY{n}{figsize}\PY{o}{=}\PY{p}{(}\PY{l+m+mi}{8}\PY{p}{,}\PY{l+m+mi}{6}\PY{p}{)}\PY{p}{)}
\PY{n}{Plutonium}\PY{p}{[}\PY{l+s+s2}{\PYZdq{}}\PY{l+s+s2}{y}\PY{l+s+s2}{\PYZdq{}}\PY{p}{]}\PY{o}{.}\PY{n}{hist}\PY{p}{(}\PY{n}{bins}\PY{o}{=}\PY{l+m+mi}{16}\PY{p}{,} \PY{n}{grid}\PY{o}{=}\PY{k+kc}{False}\PY{p}{,} \PY{n}{ax}\PY{o}{=}\PY{n}{ax}\PY{p}{)}
\PY{n}{ax}\PY{o}{.}\PY{n}{set\PYZus{}xlabel}\PY{p}{(}\PY{l+s+s2}{\PYZdq{}}\PY{l+s+s2}{Responses}\PY{l+s+s2}{\PYZdq{}}\PY{p}{)}
\PY{n}{ax}\PY{o}{.}\PY{n}{set\PYZus{}ylabel}\PY{p}{(}\PY{l+s+s2}{\PYZdq{}}\PY{l+s+s2}{Frequencies}\PY{l+s+s2}{\PYZdq{}}\PY{p}{)}
\end{Verbatim}
\end{tcolorbox}

            \begin{tcolorbox}[breakable, size=fbox, boxrule=.5pt, pad at break*=1mm, opacityfill=0]
\prompt{Out}{outcolor}{174}{\boxspacing}
\begin{Verbatim}[commandchars=\\\{\}]
Text(0, 0.5, 'Frequencies')
\end{Verbatim}
\end{tcolorbox}
        
    \begin{center}
    \adjustimage{max size={0.9\linewidth}{0.9\paperheight}}{output_45_1.png}
    \end{center}
    { \hspace*{\fill} \\}
    
    \hypertarget{inorganic-impurities}{%
\section{Inorganic Impurities}\label{inorganic-impurities}}

    This dataset is from an experiment to investigate inorganic impurities
(ash) in paper. Two variables, temperature \(T\) in degree Celsius and
time \(t\) in hours, were studied. The coded predictor variables shown
in the following table are \begin{equation}
x_1 = \frac{T-775}{115}, \quad x_2 = \frac{t-3}{1.5}
\end{equation} and the response \(y\) is impurity percentage times
\(10^3\).

    \begin{tcolorbox}[breakable, size=fbox, boxrule=1pt, pad at break*=1mm,colback=cellbackground, colframe=cellborder]
\prompt{In}{incolor}{2}{\boxspacing}
\begin{Verbatim}[commandchars=\\\{\}]
\PY{n}{Impurities} \PY{o}{=} \PY{n}{pd}\PY{o}{.}\PY{n}{read\PYZus{}csv}\PY{p}{(}\PY{l+s+s2}{\PYZdq{}}\PY{l+s+s2}{Impurities.csv}\PY{l+s+s2}{\PYZdq{}}\PY{p}{)}
\end{Verbatim}
\end{tcolorbox}

    \begin{tcolorbox}[breakable, size=fbox, boxrule=1pt, pad at break*=1mm,colback=cellbackground, colframe=cellborder]
\prompt{In}{incolor}{3}{\boxspacing}
\begin{Verbatim}[commandchars=\\\{\}]
\PY{n}{Impurities}
\end{Verbatim}
\end{tcolorbox}

            \begin{tcolorbox}[breakable, size=fbox, boxrule=.5pt, pad at break*=1mm, opacityfill=0]
\prompt{Out}{outcolor}{3}{\boxspacing}
\begin{Verbatim}[commandchars=\\\{\}]
     x1   x2    y
0  -1.0 -1.0  211
1   1.0 -1.0   92
2  -1.0  1.0  216
3   1.0  1.0   99
4  -1.5  0.0  222
5   1.5  0.0   48
6   0.0 -1.5  168
7   0.0  1.5  179
8   0.0  0.0  122
9   0.0  0.0  175
10  0.0  0.0  157
11  0.0  0.0  146
\end{Verbatim}
\end{tcolorbox}
        
    \textbf{Part (a)} What type of design has been used in the study? Can
the design be rotated.

    \emph{Solution: this is a Box-Wilson Central Composite Design. The
design cannot be rotated.}

    \begin{tcolorbox}[breakable, size=fbox, boxrule=1pt, pad at break*=1mm,colback=cellbackground, colframe=cellborder]
\prompt{In}{incolor}{4}{\boxspacing}
\begin{Verbatim}[commandchars=\\\{\}]
\PY{n}{plt}\PY{o}{.}\PY{n}{scatter}\PY{p}{(}\PY{n}{Impurities}\PY{p}{[}\PY{l+s+s2}{\PYZdq{}}\PY{l+s+s2}{x1}\PY{l+s+s2}{\PYZdq{}}\PY{p}{]}\PY{p}{,} \PY{n}{Impurities}\PY{p}{[}\PY{l+s+s2}{\PYZdq{}}\PY{l+s+s2}{x2}\PY{l+s+s2}{\PYZdq{}}\PY{p}{]}\PY{p}{)}
\end{Verbatim}
\end{tcolorbox}

            \begin{tcolorbox}[breakable, size=fbox, boxrule=.5pt, pad at break*=1mm, opacityfill=0]
\prompt{Out}{outcolor}{4}{\boxspacing}
\begin{Verbatim}[commandchars=\\\{\}]
<matplotlib.collections.PathCollection at 0x1460c084c88>
\end{Verbatim}
\end{tcolorbox}
        
    \begin{center}
    \adjustimage{max size={0.9\linewidth}{0.9\paperheight}}{output_52_1.png}
    \end{center}
    { \hspace*{\fill} \\}
    
    \textbf{Part (b)} Fit a quadratic model to the data. Is this model
satisfactory?

    \begin{tcolorbox}[breakable, size=fbox, boxrule=1pt, pad at break*=1mm,colback=cellbackground, colframe=cellborder]
\prompt{In}{incolor}{6}{\boxspacing}
\begin{Verbatim}[commandchars=\\\{\}]
\PY{n}{model} \PY{o}{=} \PY{n}{sm}\PY{o}{.}\PY{n}{formula}\PY{o}{.}\PY{n}{ols}\PY{p}{(}\PY{l+s+s1}{\PYZsq{}}\PY{l+s+s1}{y \PYZti{} x1+x2+I(x1**2)+I(x2**2)+x1*x2}\PY{l+s+s1}{\PYZsq{}}\PY{p}{,} \PY{n}{data}\PY{o}{=}\PY{n}{Impurities}\PY{p}{)}\PY{o}{.}\PY{n}{fit}\PY{p}{(}\PY{p}{)}
\PY{n}{model}\PY{o}{.}\PY{n}{summary}\PY{p}{(}\PY{p}{)}
\end{Verbatim}
\end{tcolorbox}

    \begin{Verbatim}[commandchars=\\\{\}]
C:\textbackslash{}Users\textbackslash{}godin\textbackslash{}anaconda3\textbackslash{}lib\textbackslash{}site-packages\textbackslash{}scipy\textbackslash{}stats\textbackslash{}stats.py:1604:
UserWarning: kurtosistest only valid for n>=20 {\ldots} continuing anyway, n=12
  "anyway, n=\%i" \% int(n))
    \end{Verbatim}

            \begin{tcolorbox}[breakable, size=fbox, boxrule=.5pt, pad at break*=1mm, opacityfill=0]
\prompt{Out}{outcolor}{6}{\boxspacing}
\begin{Verbatim}[commandchars=\\\{\}]
<class 'statsmodels.iolib.summary.Summary'>
"""
                            OLS Regression Results
==============================================================================
Dep. Variable:                      y   R-squared:                       0.954
Model:                            OLS   Adj. R-squared:                  0.916
Method:                 Least Squares   F-statistic:                     24.91
Date:                Sat, 03 Apr 2021   Prob (F-statistic):           0.000604
Time:                        22:46:01   Log-Likelihood:                -45.909
No. Observations:                  12   AIC:                             103.8
Df Residuals:                       6   BIC:                             106.7
Df Model:                           5
Covariance Type:            nonrobust
==============================================================================
                 coef    std err          t      P>|t|      [0.025      0.975]
------------------------------------------------------------------------------
Intercept    150.0445      7.821     19.184      0.000     130.907     169.182
x1           -58.4706      5.384    -10.861      0.000     -71.644     -45.297
x2             3.3529      5.384      0.623      0.556      -9.821      16.526
I(x1 ** 2)    -6.5282      5.693     -1.147      0.295     -20.460       7.403
I(x2 ** 2)    10.5830      5.693      1.859      0.112      -3.349      24.514
x1:x2          0.5000      7.848      0.064      0.951     -18.704      19.704
==============================================================================
Omnibus:                        6.900   Durbin-Watson:                   2.742
Prob(Omnibus):                  0.032   Jarque-Bera (JB):                3.519
Skew:                          -0.381   Prob(JB):                        0.172
Kurtosis:                       5.541   Cond. No.                         3.21
==============================================================================

Notes:
[1] Standard Errors assume that the covariance matrix of the errors is correctly
specified.
"""
\end{Verbatim}
\end{tcolorbox}
        
    \emph{I am not happy with the model, because the quadratic model is
useless (look at the extremely high t values).}

    \textbf{Part (c)} If it is important to minimise the ash value, where
would you run the process?

    \emph{We would run the process at the place with highest temperature.}

    \hypertarget{copper-manufacturing}{%
\section{Copper Manufacturing}\label{copper-manufacturing}}

The copper content of a manufacturing process is measured three times
per day and the results are reportedas parts per million. The values for
25 days are given in the table below.

    \begin{tcolorbox}[breakable, size=fbox, boxrule=1pt, pad at break*=1mm,colback=cellbackground, colframe=cellborder]
\prompt{In}{incolor}{7}{\boxspacing}
\begin{Verbatim}[commandchars=\\\{\}]
\PY{n}{Copper} \PY{o}{=} \PY{n}{pd}\PY{o}{.}\PY{n}{read\PYZus{}csv}\PY{p}{(}\PY{l+s+s2}{\PYZdq{}}\PY{l+s+s2}{Copper.csv}\PY{l+s+s2}{\PYZdq{}}\PY{p}{)}
\end{Verbatim}
\end{tcolorbox}

    \textbf{Part (a)} Using all the data, find trial control limits for
\(\bar{X}\) and \(R\) charts, construct the chart, and plot the data. Is
the process in statistical control.

    \emph{Solution: we first obtain the mean and range of each observation.}

    \begin{tcolorbox}[breakable, size=fbox, boxrule=1pt, pad at break*=1mm,colback=cellbackground, colframe=cellborder]
\prompt{In}{incolor}{8}{\boxspacing}
\begin{Verbatim}[commandchars=\\\{\}]
\PY{n}{Copper}\PY{p}{[}\PY{l+s+s2}{\PYZdq{}}\PY{l+s+s2}{Omean}\PY{l+s+s2}{\PYZdq{}}\PY{p}{]} \PY{o}{=} \PY{n}{Copper}\PY{p}{[}\PY{p}{[}\PY{l+s+s2}{\PYZdq{}}\PY{l+s+s2}{O1}\PY{l+s+s2}{\PYZdq{}}\PY{p}{,}\PY{l+s+s2}{\PYZdq{}}\PY{l+s+s2}{O2}\PY{l+s+s2}{\PYZdq{}}\PY{p}{,}\PY{l+s+s2}{\PYZdq{}}\PY{l+s+s2}{O3}\PY{l+s+s2}{\PYZdq{}}\PY{p}{]}\PY{p}{]}\PY{o}{.}\PY{n}{mean}\PY{p}{(}\PY{n}{axis}\PY{o}{=}\PY{l+m+mi}{1}\PY{p}{)}
\PY{n}{Copper}\PY{p}{[}\PY{l+s+s2}{\PYZdq{}}\PY{l+s+s2}{Range}\PY{l+s+s2}{\PYZdq{}}\PY{p}{]} \PY{o}{=} \PY{n}{Copper}\PY{p}{[}\PY{p}{[}\PY{l+s+s2}{\PYZdq{}}\PY{l+s+s2}{O1}\PY{l+s+s2}{\PYZdq{}}\PY{p}{,}\PY{l+s+s2}{\PYZdq{}}\PY{l+s+s2}{O2}\PY{l+s+s2}{\PYZdq{}}\PY{p}{,}\PY{l+s+s2}{\PYZdq{}}\PY{l+s+s2}{O3}\PY{l+s+s2}{\PYZdq{}}\PY{p}{]}\PY{p}{]}\PY{o}{.}\PY{n}{max}\PY{p}{(}\PY{n}{axis}\PY{o}{=}\PY{l+m+mi}{1}\PY{p}{)} \PY{o}{\PYZhy{}} \PY{n}{Copper}\PY{p}{[}\PY{p}{[}\PY{l+s+s2}{\PYZdq{}}\PY{l+s+s2}{O1}\PY{l+s+s2}{\PYZdq{}}\PY{p}{,}\PY{l+s+s2}{\PYZdq{}}\PY{l+s+s2}{O2}\PY{l+s+s2}{\PYZdq{}}\PY{p}{,}\PY{l+s+s2}{\PYZdq{}}\PY{l+s+s2}{O3}\PY{l+s+s2}{\PYZdq{}}\PY{p}{]}\PY{p}{]}\PY{o}{.}\PY{n}{min}\PY{p}{(}\PY{n}{axis}\PY{o}{=}\PY{l+m+mi}{1}\PY{p}{)}
\end{Verbatim}
\end{tcolorbox}

    \emph{We then write a function which input vector of mean
\texttt{mean\_vec} and vector of ranges \texttt{range\_vec} and return
limits for \(\bar{X}\) and \(R\) charts as well as actual control
chart.}

    \begin{tcolorbox}[breakable, size=fbox, boxrule=1pt, pad at break*=1mm,colback=cellbackground, colframe=cellborder]
\prompt{In}{incolor}{9}{\boxspacing}
\begin{Verbatim}[commandchars=\\\{\}]
\PY{n}{np}\PY{o}{.}\PY{n}{array}\PY{p}{(}\PY{p}{[}\PY{l+m+mi}{1}\PY{p}{,}\PY{l+m+mi}{2}\PY{p}{,}\PY{l+m+mi}{3}\PY{p}{,}\PY{l+m+mi}{4}\PY{p}{,}\PY{l+m+mi}{5}\PY{p}{]}\PY{p}{)}\PY{o}{.}\PY{n}{mean}\PY{p}{(}\PY{p}{)}
\end{Verbatim}
\end{tcolorbox}

            \begin{tcolorbox}[breakable, size=fbox, boxrule=.5pt, pad at break*=1mm, opacityfill=0]
\prompt{Out}{outcolor}{9}{\boxspacing}
\begin{Verbatim}[commandchars=\\\{\}]
3.0
\end{Verbatim}
\end{tcolorbox}
        
    \begin{tcolorbox}[breakable, size=fbox, boxrule=1pt, pad at break*=1mm,colback=cellbackground, colframe=cellborder]
\prompt{In}{incolor}{14}{\boxspacing}
\begin{Verbatim}[commandchars=\\\{\}]
\PY{k}{def} \PY{n+nf}{XbarRchart}\PY{p}{(}\PY{n}{mean\PYZus{}vec}\PY{p}{,} \PY{n}{range\PYZus{}vec}\PY{p}{,} \PY{n}{Xbar\PYZus{}plot}\PY{o}{=}\PY{k+kc}{None}\PY{p}{,} \PY{n}{R\PYZus{}plot}\PY{o}{=}\PY{k+kc}{None}\PY{p}{)}\PY{p}{:} \PY{c+c1}{\PYZsh{} input array of mean and range}
    
    \PY{c+c1}{\PYZsh{} Grand Mean and Grand Range}
    \PY{n}{grand\PYZus{}mean} \PY{o}{=} \PY{n}{mean\PYZus{}vec}\PY{o}{.}\PY{n}{mean}\PY{p}{(}\PY{p}{)}\PY{p}{;} \PY{n}{mean\PYZus{}range} \PY{o}{=} \PY{n}{range\PYZus{}vec}\PY{o}{.}\PY{n}{mean}\PY{p}{(}\PY{p}{)}
    
    \PY{c+c1}{\PYZsh{} Hard Code the multipliers for now}
    \PY{n}{n} \PY{o}{=} \PY{l+m+mi}{3}\PY{p}{;} \PY{n}{A2} \PY{o}{=} \PY{l+m+mf}{1.023}\PY{p}{;} \PY{n}{D3} \PY{o}{=} \PY{l+m+mi}{0}\PY{p}{;} \PY{n}{D4} \PY{o}{=} \PY{l+m+mf}{2.575}\PY{p}{;}
    
    \PY{c+c1}{\PYZsh{} Obtain the control limits in control charts.}
    \PY{n}{Xbar\PYZus{}UCL} \PY{o}{=} \PY{n}{grand\PYZus{}mean} \PY{o}{+} \PY{n}{A2}\PY{o}{*}\PY{n}{mean\PYZus{}range}\PY{p}{;} \PY{n}{Xbar\PYZus{}LCL} \PY{o}{=} \PY{n}{grand\PYZus{}mean} \PY{o}{\PYZhy{}} \PY{n}{A2}\PY{o}{*}\PY{n}{mean\PYZus{}range}
    \PY{n}{R\PYZus{}UCL} \PY{o}{=} \PY{n}{D4}\PY{o}{*}\PY{n}{mean\PYZus{}range}\PY{p}{;} \PY{n}{R\PYZus{}LCL} \PY{o}{=} \PY{n}{D3}\PY{o}{*}\PY{n}{mean\PYZus{}range}
    
    \PY{c+c1}{\PYZsh{} Return Xbar plot.}
    \PY{k}{if} \PY{n}{Xbar\PYZus{}plot} \PY{o+ow}{is} \PY{o+ow}{not} \PY{k+kc}{None}\PY{p}{:}
        \PY{n}{Xbar\PYZus{}plot}\PY{o}{.}\PY{n}{plot}\PY{p}{(}\PY{n}{mean\PYZus{}vec}\PY{p}{)}
        \PY{n}{Xbar\PYZus{}plot}\PY{o}{.}\PY{n}{axhline}\PY{p}{(}\PY{n}{y}\PY{o}{=}\PY{n}{Xbar\PYZus{}UCL}\PY{p}{,} \PY{n}{color}\PY{o}{=}\PY{l+s+s1}{\PYZsq{}}\PY{l+s+s1}{r}\PY{l+s+s1}{\PYZsq{}}\PY{p}{,} \PY{n}{linestyle}\PY{o}{=}\PY{l+s+s1}{\PYZsq{}}\PY{l+s+s1}{\PYZhy{}}\PY{l+s+s1}{\PYZsq{}}\PY{p}{)}
        \PY{n}{Xbar\PYZus{}plot}\PY{o}{.}\PY{n}{axhline}\PY{p}{(}\PY{n}{y}\PY{o}{=}\PY{n}{Xbar\PYZus{}LCL}\PY{p}{,} \PY{n}{color}\PY{o}{=}\PY{l+s+s1}{\PYZsq{}}\PY{l+s+s1}{r}\PY{l+s+s1}{\PYZsq{}}\PY{p}{,} \PY{n}{linestyle}\PY{o}{=}\PY{l+s+s1}{\PYZsq{}}\PY{l+s+s1}{\PYZhy{}}\PY{l+s+s1}{\PYZsq{}}\PY{p}{)}
        \PY{n}{Xbar\PYZus{}plot}\PY{o}{.}\PY{n}{axhline}\PY{p}{(}\PY{n}{y}\PY{o}{=}\PY{n}{grand\PYZus{}mean}\PY{p}{,} \PY{n}{color}\PY{o}{=}\PY{l+s+s1}{\PYZsq{}}\PY{l+s+s1}{b}\PY{l+s+s1}{\PYZsq{}}\PY{p}{,} \PY{n}{linestyle}\PY{o}{=}\PY{l+s+s1}{\PYZsq{}}\PY{l+s+s1}{\PYZhy{}}\PY{l+s+s1}{\PYZsq{}}\PY{p}{)}
        \PY{n}{Xbar\PYZus{}plot}\PY{o}{.}\PY{n}{set\PYZus{}xlabel}\PY{p}{(}\PY{l+s+s1}{\PYZsq{}}\PY{l+s+s1}{Run}\PY{l+s+s1}{\PYZsq{}}\PY{p}{)}
        \PY{n}{Xbar\PYZus{}plot}\PY{o}{.}\PY{n}{set\PYZus{}ylabel}\PY{p}{(}\PY{l+s+s1}{\PYZsq{}}\PY{l+s+s1}{Mean of Observation}\PY{l+s+s1}{\PYZsq{}}\PY{p}{)}
        \PY{n}{Xbar\PYZus{}plot}\PY{o}{.}\PY{n}{set\PYZus{}title}\PY{p}{(}\PY{l+s+sa}{f}\PY{l+s+s1}{\PYZsq{}}\PY{l+s+s1}{Xbar Chart (mean = }\PY{l+s+s1}{\PYZob{}}\PY{l+s+s1}{np.round(grand\PYZus{}mean,3)\PYZcb{}, limits = (}\PY{l+s+s1}{\PYZob{}}\PY{l+s+s1}{np.round(Xbar\PYZus{}LCL, 3)\PYZcb{}, }\PY{l+s+s1}{\PYZob{}}\PY{l+s+s1}{np.round(Xbar\PYZus{}UCL, 3)\PYZcb{}))}\PY{l+s+s1}{\PYZsq{}}\PY{p}{)}

    \PY{c+c1}{\PYZsh{} Return R plot.}
    \PY{k}{if} \PY{n}{R\PYZus{}plot} \PY{o+ow}{is} \PY{o+ow}{not} \PY{k+kc}{None}\PY{p}{:}
        \PY{n}{R\PYZus{}plot}\PY{o}{.}\PY{n}{plot}\PY{p}{(}\PY{n}{range\PYZus{}vec}\PY{p}{)}
        \PY{n}{R\PYZus{}plot}\PY{o}{.}\PY{n}{axhline}\PY{p}{(}\PY{n}{y}\PY{o}{=}\PY{n}{R\PYZus{}UCL}\PY{p}{,} \PY{n}{color}\PY{o}{=}\PY{l+s+s1}{\PYZsq{}}\PY{l+s+s1}{r}\PY{l+s+s1}{\PYZsq{}}\PY{p}{,} \PY{n}{linestyle}\PY{o}{=}\PY{l+s+s1}{\PYZsq{}}\PY{l+s+s1}{\PYZhy{}}\PY{l+s+s1}{\PYZsq{}}\PY{p}{)}
        \PY{n}{R\PYZus{}plot}\PY{o}{.}\PY{n}{axhline}\PY{p}{(}\PY{n}{y}\PY{o}{=}\PY{n}{R\PYZus{}LCL}\PY{p}{,} \PY{n}{color}\PY{o}{=}\PY{l+s+s1}{\PYZsq{}}\PY{l+s+s1}{r}\PY{l+s+s1}{\PYZsq{}}\PY{p}{,} \PY{n}{linestyle}\PY{o}{=}\PY{l+s+s1}{\PYZsq{}}\PY{l+s+s1}{\PYZhy{}}\PY{l+s+s1}{\PYZsq{}}\PY{p}{)}
        \PY{n}{R\PYZus{}plot}\PY{o}{.}\PY{n}{axhline}\PY{p}{(}\PY{n}{y}\PY{o}{=}\PY{n}{mean\PYZus{}range}\PY{p}{,} \PY{n}{color}\PY{o}{=}\PY{l+s+s1}{\PYZsq{}}\PY{l+s+s1}{b}\PY{l+s+s1}{\PYZsq{}}\PY{p}{,} \PY{n}{linestyle}\PY{o}{=}\PY{l+s+s1}{\PYZsq{}}\PY{l+s+s1}{\PYZhy{}}\PY{l+s+s1}{\PYZsq{}}\PY{p}{)}
        \PY{n}{R\PYZus{}plot}\PY{o}{.}\PY{n}{set\PYZus{}xlabel}\PY{p}{(}\PY{l+s+s1}{\PYZsq{}}\PY{l+s+s1}{Run}\PY{l+s+s1}{\PYZsq{}}\PY{p}{)}
        \PY{n}{R\PYZus{}plot}\PY{o}{.}\PY{n}{set\PYZus{}ylabel}\PY{p}{(}\PY{l+s+s1}{\PYZsq{}}\PY{l+s+s1}{Range of Observation}\PY{l+s+s1}{\PYZsq{}}\PY{p}{)}
        \PY{n}{R\PYZus{}plot}\PY{o}{.}\PY{n}{set\PYZus{}title}\PY{p}{(}\PY{l+s+sa}{f}\PY{l+s+s1}{\PYZsq{}}\PY{l+s+s1}{Xbar Chart (mean = }\PY{l+s+s1}{\PYZob{}}\PY{l+s+s1}{np.round(mean\PYZus{}range,3)\PYZcb{}, limits = (}\PY{l+s+s1}{\PYZob{}}\PY{l+s+s1}{np.round(R\PYZus{}LCL, 3)\PYZcb{}, }\PY{l+s+s1}{\PYZob{}}\PY{l+s+s1}{np.round(R\PYZus{}UCL, 3)\PYZcb{})}\PY{l+s+s1}{\PYZsq{}}\PY{p}{)}
    
    \PY{n}{result} \PY{o}{=} \PY{n+nb}{dict}\PY{p}{(}\PY{p}{)}
    \PY{n}{result}\PY{p}{[}\PY{l+s+s2}{\PYZdq{}}\PY{l+s+s2}{grand\PYZus{}mean}\PY{l+s+s2}{\PYZdq{}}\PY{p}{]} \PY{o}{=} \PY{n}{grand\PYZus{}mean}
    \PY{n}{result}\PY{p}{[}\PY{l+s+s2}{\PYZdq{}}\PY{l+s+s2}{mean\PYZus{}range}\PY{l+s+s2}{\PYZdq{}}\PY{p}{]} \PY{o}{=} \PY{n}{mean\PYZus{}range}
    \PY{n}{result}\PY{p}{[}\PY{l+s+s2}{\PYZdq{}}\PY{l+s+s2}{Xbar\PYZus{}lim}\PY{l+s+s2}{\PYZdq{}}\PY{p}{]} \PY{o}{=} \PY{p}{(}\PY{n}{Xbar\PYZus{}LCL}\PY{p}{,} \PY{n}{Xbar\PYZus{}UCL}\PY{p}{)}
    \PY{n}{result}\PY{p}{[}\PY{l+s+s2}{\PYZdq{}}\PY{l+s+s2}{R\PYZus{}lim}\PY{l+s+s2}{\PYZdq{}}\PY{p}{]} \PY{o}{=} \PY{p}{(}\PY{n}{R\PYZus{}LCL}\PY{p}{,} \PY{n}{R\PYZus{}UCL}\PY{p}{)}
    
    \PY{k}{return} \PY{n}{result}
\end{Verbatim}
\end{tcolorbox}

    \begin{tcolorbox}[breakable, size=fbox, boxrule=1pt, pad at break*=1mm,colback=cellbackground, colframe=cellborder]
\prompt{In}{incolor}{15}{\boxspacing}
\begin{Verbatim}[commandchars=\\\{\}]
\PY{n}{fig}\PY{p}{,} \PY{p}{(}\PY{n}{ax1}\PY{p}{,} \PY{n}{ax2}\PY{p}{)} \PY{o}{=} \PY{n}{plt}\PY{o}{.}\PY{n}{subplots}\PY{p}{(}\PY{n}{nrows}\PY{o}{=}\PY{l+m+mi}{1}\PY{p}{,} \PY{n}{ncols}\PY{o}{=}\PY{l+m+mi}{2}\PY{p}{,} \PY{n}{figsize}\PY{o}{=}\PY{p}{(}\PY{l+m+mi}{14}\PY{p}{,}\PY{l+m+mi}{5}\PY{p}{)}\PY{p}{)}
\PY{n}{result} \PY{o}{=} \PY{n}{XbarRchart}\PY{p}{(}\PY{n}{Copper}\PY{p}{[}\PY{l+s+s2}{\PYZdq{}}\PY{l+s+s2}{Omean}\PY{l+s+s2}{\PYZdq{}}\PY{p}{]}\PY{p}{,} \PY{n}{Copper}\PY{p}{[}\PY{l+s+s2}{\PYZdq{}}\PY{l+s+s2}{Range}\PY{l+s+s2}{\PYZdq{}}\PY{p}{]}\PY{p}{,} \PY{n}{Xbar\PYZus{}plot}\PY{o}{=}\PY{n}{ax1}\PY{p}{,} \PY{n}{R\PYZus{}plot}\PY{o}{=}\PY{n}{ax2}\PY{p}{)}
\end{Verbatim}
\end{tcolorbox}

    \begin{center}
    \adjustimage{max size={0.9\linewidth}{0.9\paperheight}}{output_66_0.png}
    \end{center}
    { \hspace*{\fill} \\}
    
    \emph{All the ranges in \(R\) charts falls in between the control
limits, so we can assume the process has constant variations. We can see
that there are points in \(\bar{X}\) beyond the control limits, so the
process is statistically out-of-control.}

    \textbf{Part (b)} If necessary, revise the control limits computed in
part (a), assuming any samples that plot outside the control limits can
be eliminated. Continue to eliminate points outside the control limits
and revise, until all points plot between control limits.

    \emph{Solution: we repeatedly filter out the points and revise control
limits. Here I do not use while loop since we should be able to make all
points fall into control limit within two iterations.}

    \begin{tcolorbox}[breakable, size=fbox, boxrule=1pt, pad at break*=1mm,colback=cellbackground, colframe=cellborder]
\prompt{In}{incolor}{19}{\boxspacing}
\begin{Verbatim}[commandchars=\\\{\}]
\PY{n}{filtered} \PY{o}{=} \PY{n}{Copper}\PY{p}{[}\PY{p}{(}\PY{n}{Copper}\PY{o}{.}\PY{n}{Omean} \PY{o}{\PYZlt{}} \PY{n}{result}\PY{p}{[}\PY{l+s+s2}{\PYZdq{}}\PY{l+s+s2}{Xbar\PYZus{}lim}\PY{l+s+s2}{\PYZdq{}}\PY{p}{]}\PY{p}{[}\PY{l+m+mi}{1}\PY{p}{]}\PY{p}{)} \PY{o}{\PYZam{}} \PY{p}{(}\PY{n}{Copper}\PY{o}{.}\PY{n}{Omean} \PY{o}{\PYZgt{}} \PY{n}{result}\PY{p}{[}\PY{l+s+s2}{\PYZdq{}}\PY{l+s+s2}{Xbar\PYZus{}lim}\PY{l+s+s2}{\PYZdq{}}\PY{p}{]}\PY{p}{[}\PY{l+m+mi}{0}\PY{p}{]}\PY{p}{)}\PY{p}{]}
\PY{n}{fig}\PY{p}{,} \PY{p}{(}\PY{n}{ax1}\PY{p}{,} \PY{n}{ax2}\PY{p}{)} \PY{o}{=} \PY{n}{plt}\PY{o}{.}\PY{n}{subplots}\PY{p}{(}\PY{n}{nrows}\PY{o}{=}\PY{l+m+mi}{1}\PY{p}{,} \PY{n}{ncols}\PY{o}{=}\PY{l+m+mi}{2}\PY{p}{,} \PY{n}{figsize}\PY{o}{=}\PY{p}{(}\PY{l+m+mi}{14}\PY{p}{,}\PY{l+m+mi}{5}\PY{p}{)}\PY{p}{)}
\PY{n}{result2} \PY{o}{=} \PY{n}{XbarRchart}\PY{p}{(}\PY{n}{filtered}\PY{p}{[}\PY{l+s+s2}{\PYZdq{}}\PY{l+s+s2}{Omean}\PY{l+s+s2}{\PYZdq{}}\PY{p}{]}\PY{p}{,} \PY{n}{filtered}\PY{p}{[}\PY{l+s+s2}{\PYZdq{}}\PY{l+s+s2}{Range}\PY{l+s+s2}{\PYZdq{}}\PY{p}{]}\PY{p}{,} \PY{n}{Xbar\PYZus{}plot}\PY{o}{=}\PY{n}{ax1}\PY{p}{,} \PY{n}{R\PYZus{}plot}\PY{o}{=}\PY{n}{ax2}\PY{p}{)}
\end{Verbatim}
\end{tcolorbox}

    \begin{center}
    \adjustimage{max size={0.9\linewidth}{0.9\paperheight}}{output_70_0.png}
    \end{center}
    { \hspace*{\fill} \\}
    
    \emph{Turns out one iteration is enough - we can check if all points
fall in between the control limits.}

    \begin{tcolorbox}[breakable, size=fbox, boxrule=1pt, pad at break*=1mm,colback=cellbackground, colframe=cellborder]
\prompt{In}{incolor}{21}{\boxspacing}
\begin{Verbatim}[commandchars=\\\{\}]
\PY{n}{filtered2} \PY{o}{=} \PY{n}{filtered}\PY{p}{[}\PY{p}{(}\PY{n}{filtered}\PY{o}{.}\PY{n}{Omean} \PY{o}{\PYZlt{}} \PY{n}{result2}\PY{p}{[}\PY{l+s+s2}{\PYZdq{}}\PY{l+s+s2}{Xbar\PYZus{}lim}\PY{l+s+s2}{\PYZdq{}}\PY{p}{]}\PY{p}{[}\PY{l+m+mi}{1}\PY{p}{]}\PY{p}{)} \PY{o}{\PYZam{}} \PY{p}{(}\PY{n}{filtered}\PY{o}{.}\PY{n}{Omean} \PY{o}{\PYZgt{}} \PY{n}{result2}\PY{p}{[}\PY{l+s+s2}{\PYZdq{}}\PY{l+s+s2}{Xbar\PYZus{}lim}\PY{l+s+s2}{\PYZdq{}}\PY{p}{]}\PY{p}{[}\PY{l+m+mi}{0}\PY{p}{]}\PY{p}{)}\PY{p}{]}
\PY{n}{fig}\PY{p}{,} \PY{p}{(}\PY{n}{ax1}\PY{p}{,} \PY{n}{ax2}\PY{p}{)} \PY{o}{=} \PY{n}{plt}\PY{o}{.}\PY{n}{subplots}\PY{p}{(}\PY{n}{nrows}\PY{o}{=}\PY{l+m+mi}{1}\PY{p}{,} \PY{n}{ncols}\PY{o}{=}\PY{l+m+mi}{2}\PY{p}{,} \PY{n}{figsize}\PY{o}{=}\PY{p}{(}\PY{l+m+mi}{14}\PY{p}{,}\PY{l+m+mi}{5}\PY{p}{)}\PY{p}{)}
\PY{n}{result3} \PY{o}{=} \PY{n}{XbarRchart}\PY{p}{(}\PY{n}{filtered2}\PY{p}{[}\PY{l+s+s2}{\PYZdq{}}\PY{l+s+s2}{Omean}\PY{l+s+s2}{\PYZdq{}}\PY{p}{]}\PY{p}{,} \PY{n}{filtered2}\PY{p}{[}\PY{l+s+s2}{\PYZdq{}}\PY{l+s+s2}{Range}\PY{l+s+s2}{\PYZdq{}}\PY{p}{]}\PY{p}{,} \PY{n}{Xbar\PYZus{}plot}\PY{o}{=}\PY{n}{ax1}\PY{p}{,} \PY{n}{R\PYZus{}plot}\PY{o}{=}\PY{n}{ax2}\PY{p}{)}
\end{Verbatim}
\end{tcolorbox}

    \begin{center}
    \adjustimage{max size={0.9\linewidth}{0.9\paperheight}}{output_72_0.png}
    \end{center}
    { \hspace*{\fill} \\}
    
    \textbf{Part (c)} Given that the specifications are at \(6.0 \pm 0.5\),
estimate \(C_p\) and \(C_{pk}\) for the in-control process and interpret
these ratios.

    \begin{tcolorbox}[breakable, size=fbox, boxrule=1pt, pad at break*=1mm,colback=cellbackground, colframe=cellborder]
\prompt{In}{incolor}{22}{\boxspacing}
\begin{Verbatim}[commandchars=\\\{\}]
\PY{n}{USL} \PY{o}{=} \PY{l+m+mf}{6.5}\PY{p}{;} \PY{n}{LSL} \PY{o}{=} \PY{l+m+mf}{5.5}\PY{p}{;} \PY{n}{d2} \PY{o}{=} \PY{l+m+mf}{1.693}\PY{p}{;}
\PY{n}{mu} \PY{o}{=} \PY{n}{result3}\PY{p}{[}\PY{l+s+s2}{\PYZdq{}}\PY{l+s+s2}{grand\PYZus{}mean}\PY{l+s+s2}{\PYZdq{}}\PY{p}{]}\PY{p}{;} \PY{n}{sigma\PYZus{}hat} \PY{o}{=} \PY{n}{result3}\PY{p}{[}\PY{l+s+s2}{\PYZdq{}}\PY{l+s+s2}{mean\PYZus{}range}\PY{l+s+s2}{\PYZdq{}}\PY{p}{]} \PY{o}{/} \PY{n}{d2}
\PY{n}{Cp} \PY{o}{=} \PY{p}{(}\PY{n}{USL} \PY{o}{\PYZhy{}} \PY{n}{LSL}\PY{p}{)} \PY{o}{/} \PY{p}{(}\PY{l+m+mi}{6} \PY{o}{*} \PY{n}{sigma\PYZus{}hat}\PY{p}{)}
\PY{n}{Cpk} \PY{o}{=} \PY{n+nb}{min}\PY{p}{(}\PY{p}{(}\PY{n}{USL}\PY{o}{\PYZhy{}}\PY{n}{mu}\PY{p}{)}\PY{p}{,} \PY{p}{(}\PY{n}{mu}\PY{o}{\PYZhy{}}\PY{n}{LSL}\PY{p}{)}\PY{p}{)} \PY{o}{/} \PY{p}{(}\PY{l+m+mi}{3} \PY{o}{*} \PY{n}{sigma\PYZus{}hat}\PY{p}{)}
\PY{n+nb}{print}\PY{p}{(}\PY{l+s+sa}{f}\PY{l+s+s2}{\PYZdq{}}\PY{l+s+s2}{Cp = }\PY{l+s+s2}{\PYZob{}}\PY{l+s+s2}{np.round(Cp,5)\PYZcb{}, Cpk = }\PY{l+s+s2}{\PYZob{}}\PY{l+s+s2}{np.round(Cpk,5)\PYZcb{}}\PY{l+s+s2}{\PYZdq{}}\PY{p}{)}
\end{Verbatim}
\end{tcolorbox}

    \begin{Verbatim}[commandchars=\\\{\}]
Cp = 0.24845, Cpk = 0.13763
    \end{Verbatim}

    \emph{Comment: both \texttt{Cp} and \texttt{Cpk} are very small, so it
is not capable of producing acceptable product. Under normal assumption,
the average probability of making no mistake is approximately:}

    \begin{tcolorbox}[breakable, size=fbox, boxrule=1pt, pad at break*=1mm,colback=cellbackground, colframe=cellborder]
\prompt{In}{incolor}{23}{\boxspacing}
\begin{Verbatim}[commandchars=\\\{\}]
\PY{l+m+mi}{1} \PY{o}{\PYZhy{}} \PY{l+m+mi}{2}\PY{o}{*}\PY{p}{(}\PY{l+m+mi}{1} \PY{o}{\PYZhy{}} \PY{n}{stats}\PY{o}{.}\PY{n}{norm}\PY{o}{.}\PY{n}{cdf}\PY{p}{(}\PY{l+m+mi}{3}\PY{o}{*}\PY{n}{Cp}\PY{p}{)}\PY{p}{)}
\end{Verbatim}
\end{tcolorbox}

            \begin{tcolorbox}[breakable, size=fbox, boxrule=.5pt, pad at break*=1mm, opacityfill=0]
\prompt{Out}{outcolor}{23}{\boxspacing}
\begin{Verbatim}[commandchars=\\\{\}]
0.5439373673474139
\end{Verbatim}
\end{tcolorbox}
        
    \begin{tcolorbox}[breakable, size=fbox, boxrule=1pt, pad at break*=1mm,colback=cellbackground, colframe=cellborder]
\prompt{In}{incolor}{ }{\boxspacing}
\begin{Verbatim}[commandchars=\\\{\}]

\end{Verbatim}
\end{tcolorbox}


    % Add a bibliography block to the postdoc
    
    
    
\end{document}
